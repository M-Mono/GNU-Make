\chapter{执行make}
一般描述整个工程编译规则的Makefile可以通过不止一种方式来执行。最简单直接的方
法就是使用不带任何参数的“make”命令来重新编译所有过时的文件。通常我们的
Makefile就书写为这种方式。

在某些情况下:
\begin{enumerate}
\itemsep=4pt \parskip=0pt
\item  可能需要使用make更新一部分过时文件而不是全部。

\item 需要使用另外的编译器或者重新定义编译选项。

\item 只需要察看那些文件被修改,而不需要重新编译。
\end{enumerate}

为了达到这些特殊的目的,需要使用make的命令行参数来实现。Make的命令行参数能实
现的功能不仅限于这些,通过make的命令行参数可以实现很多特殊功能。

另外,make的退出状态有三种:
\begin{enumerate}
\itemsep=4pt \parskip=0pt
\item[]  0 — 状态为0时,表示执行成功。

\item[] 2 — 执行过程出现错误,同时会提示错误信息。

\item[] 1 — 在执行make时使用了“-q”参数,而且当前工程中存在过时的目标文件。
\end{enumerate}

本章的内容主要讲述如何使用make的命令参数选项来实现一些特殊的目的。在本章最后
会对make的命令行参数选项进行比较详细的讨论。

\section{指定makefile文件}
本文以前的部分对如何指定makefile文件已经有过介绍。当需要将一个普通命名的文件
作为makefile文件时,需要使用make的“-f”、“--file”或者“--makefile”选项来指定。
例如:“make –f altmake”,它的意思是告诉make将文件“altmake”作为makefile文件来
解析执行。

当在make的命令行选项中出现多个“-f”参数时,所有通过“-f”参数指定的文件都被作为
make解析执行的makefile文件。

默认情况,在没有使用“-f”(“--file”或者“--makefile”)指定文件时。make会在工作
目录(当前目录)依次搜索命名为“GNUmakefile”、“makefile”和“Makefile”的文件,最
终解析执行的是这三个文件中首先搜索到的哪一个。

\section{指定终极目标}
“终极目标”的基本概念我们在前面已经提到过。所谓终极目标就是make最终所要重建的
Makefile某个规则的目标(也可以称之为“最终规则”)。为了完成对终极目标的重建,
可能会触发它的依赖或者依赖的依赖文件被重建的过程。

默认情况下,终极目标就是出现在Makefile中,除以点号“.”开始的第一个规则中的第一
个目标(如果第一个规则存在多个目标)。因此Makefile书写时需要保证:第一个目标
的编译规则就描述了整个工程或者程序的编译过程和规则。如果Makefile的第一个规则
有多个目标,那么默认的终极目标是多个目标中的第一个。我们在Makefile所在的目录
下执行“make”时,将完成对默认终极目标的重建。

另外,也可以通过命令行将一个Makefile中的目标指定为此次make过程的终极目标,替
代默认的终极目标。使用Makefile中目标名作为参数来执行“make”(格式为 “make
TARGET\_NAME”,如:“make clean”),可以把这个目标指定为终极目标。使用这种方
式,我们也可以同时指定多个终极目标。

任何出现在Makefile规则中的目标都可以被指定为终极目标(不包含以“-”开始的和包含
“=”的赋值语句,一般它们也不会作为一个目标出现),甚至可以指定一个在Makefile中
不存在的目标作为终极目标,前提是存在一个对应的隐含规则能够实现对这个目标的重
建。例如:目录“src”下存在一个.c的源文件“foo.c”,我们的Makefile中不存在对目标
“foo”的描述规则,或者当前目录下就没有默认的makefile文件,为了编译“foo.c”生成
可执行的“foo”。只需要将“foo”作为make的参数执行:“make foo”就可以实现编译“foo”
的目的。

make在执行时设置一个特殊变量“MAKECMDGOALS”,此变量记录了命令行参数指定的终极
目标列表,没有通过参数指定终极目标时此变量为空。注意:此变量仅用在特殊的场合
(比如判断),在Makeifle中不要对它进行重新定义!例如:

\begin{Verbatim}[]
sources = foo.c bar.c
ifneq ($(MAKECMDGOALS),clean)
include $(sources:.c=.d)
endif
\end{Verbatim}

例子中使用了变量“MAKECMDGOALS”来判断命令行参数是否指定终极目标为“clean”,如果
不是才包含所有源文件对应的依赖关系描述文件。上例的功能是避免“make clean”时
make试图重建所有.d文件的过程。

这种方式主要用在以下几个方面:

1. 对程序的一部分进行编译,或者仅对某几个程序进行编译而不是完整编译这个工程
(也可以在命令行参数中明确给出原本的默认的终极目标,例如:make all,没有人会
说它是错误的。但是大家都会认为多此一举)。例如以下一个Makefile的片段,其中各
个文件都有自己的描述规则:
\begin{Verbatim}[]
.PHONY: all all: size nm ld ar as
\end{Verbatim}
\noindent 仅需要重建“size”文件时,执行“make size”就可以了。其它的程序就不会被
重建。

2. 指定编译或者创建那些正常编译过程不能生成的文件(例如重建一个调试输出文件、
或者编译一个调试版本的程序等),这些文件在Makefile中存在重建规则,但是它们没
有出现在默认终极目标目标的依赖中。

3. 指定执行一个由伪目标定义的若干条命令或者一个空目标文件。如绝大多数Makefile
中都会包含一个“clean”伪目标,这个伪目标定义了删除make过程生成的所有文件的命
令,需要删除这些文件时执行“make clean”就可以了。本节以下列出了一些典型的伪目
标和空目标的名字。

部分标准的伪目标和空目标命名:
\begin{dinglist}{71}
\itemsep=4pt \parskip=0pt
\item \textbf{all}

\item \textbf{clean}

这个伪目标定义了一组命令,这些命令的功能是删除所有由make创建的文件。
\item \textbf{mostlyclean}

和“clean”伪目标功能相似。区别在于它所定义的删除命令不会全部删除由make生成的文件。比如说不需要删除某些库文件。
\item \textbf{distclean}

\item \textbf{realclean}

\item \textbf{clobber}

同样类似于伪目标“clean”,但它们所定义的删除命令所删除的文件更多。可以包含非make创建的文件。例如:编译之前系统的配置文件、链接文件等。
\item \textbf{install}

将make成功创建的可执行文件拷贝到shell 环境变量“PATH”指定的某个目录。典型的,应用可执行文件被拷贝到目录“/usr/local/bin”,库文件拷贝到目录“/usr/local/lib”目录下。
\item \textbf{print}

打印出所有被更改的源文件列表。
\item \textbf{tar}

创建一个tar文件(归档文件包)。
\item \textbf{shar}

创建一个源代码的shell文档(shar文件)。
\item \textbf{dist}

为源文件创建发布的压缩包,可以使各种压缩方式的发布包。
\item \textbf{TAGS}

创建当前目录下所有源文件的符号信息(“tags”)文件,这个文件可被vim使用。
\item \textbf{check}
\item \textbf{test}

\end{dinglist}

对Makefile最后生成的文件进行检查。

这些功能和目标的对照关系并不是GNU make规定的。你可以在Makefile中定义任何命名的伪目标。但是以上这些都被作约定,所有开源的工程中这些特殊的目标的命名都是按照这种约定来的。既然绝大多数程序员都遵循这种约定,自然我们也应该按照这种约定来做。否则在别人看来这样Makefile只能算一个样例,不能作为正式版本。


\section{替代命令的执行}
书写Makefile的目的就是为了告诉make一个目标是否过期,以及如果重建一个过期的目
标。但是在某些时候,我们并不希望真正更新那些已经过期的目标文件(比如:只是检
查更新目标的命令是否正确,或者察看那些目标需要更新)。要实现这样的目的,可以
使用一些特定的参数来限定make执行的动作。通过指定参数,替代make默认动作的执
行。因此我们把这种方式称为:替代命令的执行。这些参数包括:

\begin{Verbatim}[]
-n
--just-print
--dry-run
--recon
\end{Verbatim}
指定make执行空操作(不执行规则的命令),只打印出需要重建目标使用的命令(只打
印过期的目标的重建命令),而不对目标进行重建。

\begin{Verbatim}[]
-t
--touch
\end{Verbatim}
类似于shell下的“touch”命令的功能。更新所有目标文件的时间戳(对于过时的目标文
件不进行内容更新,只更新时间戳)。

\begin{Verbatim}[]
-q
--question
\end{Verbatim}
不执行任何命令并且不打印任何输出信息,只检查所指定的目标是否已经是最新的。如
果是则返回0,否则返回1。使用“-q”(“--question”)的目的只是让make返回给定(没
有指定则时终极目标)的目标是否是最新的。可以根据它的返回值来判断是否须要真正
的执行更新目标的动作。

\begin{Verbatim}[]
-W FILE
--what-if= FILE
--assume-new= FILE
--new-file= FILE
\end{Verbatim}
这个参数需要指定一个文件名。通常是一个存在源文件。make将当前系统时间作为这个
文件的时间戳(假设这个文件被修改过,但不真正的更改文件本身的时间戳)。因此这
个文件的时间戳被认为最新的,在执行时依赖于这个文件的目标将会被重建。通过这种
方式并结合“-n”参数,我们可以查看那些目标依赖于这个文件(修改这个文件以后执行
make那些目标会被更新)。通常“-W”参数和“-n”参数一同使用,可以在修改一个文件后
来检查修改会造成那些目标需要被更新,但并不执行更新的命令,只是打印命令。“-W”
和“-t”参数配合使用时,make将忽略其它规则的命令。只对依赖于“-W”指定文件的目标
执行“touch”命令,在没有使用“-s”时,可以看到那些文件执行了“touch”。需要说明的
是,make在对文件执行“touch”时不是调用shell的命令,而是由make直接操作。

“-W”和“-q”参数配合使用时。由于将当前时间作为指定文件的时间戳(目标文件相对于
系统当前时间是过时的),所以make的返回状态在没有错误发生时为1,存在错误时为
2。

\begin{quote}\kaishu
\textbf{注意:}以上三个参数同时使用时可能会出现错误。
\end{quote}

\noindent \textbf{总结:}

参数“-n”、“-t”和“-q”不影响之前带“+”号和包含“\$(MAKE)”的命令行的执行。就是说如
果在规则的命令行中命令之前使用了“+”或者此命令行是递归地make调用时,无论是否使
用了这三个参数之一,这些命令都得到执行。

“-W”参数有两个特点:

\begin{enumerate}
\itemsep=4pt \parskip=0pt
  \item 可以和“-n”或者“-q”参数配合使用来查看修改所带来的影响(导致那些目标
      会被重建)。
  \item 在不指定“-n”和“-q”参数、只使用“-W”指定一个文件时,可以模拟这个文件
      被修改的状态。make就会重建依赖于此文件的所有目标。
\end{enumerate}

另外“-p”和“-v”参数可以允许我们输出Makefille被执行的过程信息,相信这一点在很多
场合,特别是调试Makefile时非常有用(本章的最后一节有详细讨论)。


\section{防止特定文件重建}
时当修改了工程中的某一个文件后,并不希望重建那些依赖于这个文件的目标。比如说
我们在给一个头文件中加入了一个宏定义、或者一个增加的函数声明,这些修改不会对
已经编译完成的程序产生任何影响。但在执行make时,因为头文件的改变会导致所有包
含它的源文件被重新编译,当然了终极目标肯定也会被重建(除非你的模块时独立的,
或者你的Makefile的规则链的定义本身就存在缺陷)。这种情况时,为了避免重新编译
整个工程,我们可以按照下边的过程来处理:

第一种:

\begin{enumerate}
\itemsep=4pt \parskip=0pt
  \item 使用“make”命令对所有需要更新的目标进行重建。保证修改某个文件之前所
      有的目标已经是最新的。
  \item 编辑需要修改的那个源文件(修改的头文件的内容不能对之前的编译的程序
      有影响,比如:更改了头文件中的宏定义。这样会造成已经存在的程序和实现
      不相符!这里所说的修改指:不改变已经存在的任何东西,除非你有特殊的要
      求)。
  \item 使用“make -t”命令来改变已存在的所有的目标文件的时间戳,将其最后修改
      时间修改到当前时间。
\end{enumerate}

第一种方式的现实基于这样的前提:需要对未更改头文件之前的编译程序进行保存(有
点像备份)。通常这种需求还是比较少见的。更多的情况是:修改这个头文件之前已经
修改了很多源文件或者其它的头文件,并且也没有执行make(缺少了上边的第一步)。
这种方式,有点像有计划的修改。没有普遍性,修改通常是不可预知的。因此这种方式
在实际应用中几乎没有参考的意义。对于不可预知的修改所带来的问题,我们需要另外
一种方式。

第二种方式在很大程度上可以满足我们的需求:

\begin{enumerate}
\itemsep=4pt \parskip=0pt
  \item 执行编译,使用“make –o HEADERFILE”,“HEADERFILE”为需要忽略更改的头
      文件,防止那些依赖于这个头文件的目标被重建。忽略多个头文件的修改可使
      用多个“-o HEADERFILE”。这样,头文件“HEADERFILE”的修改就不会触发依赖它
      的目标被重建(通过“-o”告诉make,这个头文件的时间戳比它的依赖晚)。%
      \textbf{需要注意的是:“-o”参数的这种使用方式仅限于头文件(.h文件),
  不
      能使用“-o”来指定源文件。}
  \item 执行“make -t”命令。
\end{enumerate}


\section{替换变量定义}
执行make时,一个含有“=”的命令行参数“V=X”的含义是定义变量“V”的值为“X”,并将这
个比变量作为make的参数。这种方式定义的变量会替代Makefile中的同名变量定义(如
果存在,并且在Makefile中没有使用指示符“override”对这个变量进行说明),这个过
程被称之命令行参数定义覆盖普通变量定义。

通常用这种方式来传递一个公共变量给make。例如:在Makefile中,使用变量“CFLAGS”
来指定编译参数,在Makefile规则的命令一般都是这么写的:

\begin{Verbatim}[]
cc -c $(CFLAGS) foo.c
\end{Verbatim}

这样就可以通过改变“CFLAGS”的值来控制编译选项。我们可以在Makefile中为它指定
值。例如:
\begin{Verbatim}[]
CFLAGS=-g
\end{Verbatim}
当直接执行“make”时,编译命令是“cc –c –g foo.c”。如果需要改变“CFLAGS”的定义,
可以在命令行中指定这个变量的值,像这样“make CFLAGS=’-g –O2’”,此时所执行的编
译命令将是“cc –c –g –O2 foo.c”(在参数中如果包含空格或者shell的特殊字符,则需
要将参数放在引号中)。对变量“CFLAGS”定义追加的功能就是使用这种方式来实现的。

变量“CFLAGS”可以通过这种方式来实现,它是make的隐含变量之一。对于普通变量的定
义,也可以通过这种方式来对它进行重新定义(覆盖Makefile中的定义)、或者实现变
量值的追加功能。

通过命令行定义变量时,也存在两种风格的变量定义:递归展式定义和直接展开式定
义。上例中使用递归展开式的定义(使用“=”),也可以是直接展开式的(使用“:=”)。
除非在命令行定义的变量值中包含了对其他变量或者函数的引用,否则这两种方式在此
是等价的。

为了防止命令行参数的变量定义覆盖Makefile中的同名变量定义,可以在Makefile中使
用指示符“override”声明这个变量。这一点前边的章节已经有详细的讨论!


\section{使用make进行编译测试}

正常情况make在执行Makefile时,如果出现命令执行的错误,会立即放弃继续执行并返
回一个非0的状态。就是说错误发生点之后的命令将不会被执行。一个错误的发生就表明
了终极目标将不能被重建,make一旦检查到错误就会立刻终止执行。

假如我们在修改了一些源文件之后重新编译工程,当然了,我们所希望的是在某一个文
件编译出错以后能够继续进行后续文件的编译。直到最后出现链接错误时才退出。这样
的目的是为了了解所修改的文件中那些文件没有修改正确。在下一次编译之前能够对出
现错误的所有文件进行改正。而不是编译一次改正一个文件,或者改正一个文件再编译
一次。

我了实现我们这个目的,需要使用make的“-k”或者“--keep-going”命令行选项。这个参
数的功能是告诉make当出现错误时继续执行,直到最后出现致命错误(无法重建终极目
标)才返回非0并退出。例如:当编译一个.o目标文件时出现错误,如果使用“make -k”
执行,make将不会在检测到这个错误时退出(虽然已经知道终极目标是不可能会重建成
功的);只是给出一个错误消息,make将继续重建其它需要重建的目标文件;直到最后
出现致命错误才退出。在没有使用“-k”或者“—keep-going”时,make在检测到错误时会立
刻退出。

总之:在通常情况下,make的目的是重建终极目标。当它在执行过程中一旦发现无法重
建终极目标,就立刻以非0状态退出。当使用“-k”或者“--keep-going”参数时,执行的目
的是为了测试重建过程,需要发现存在的所有问题,以便在下一次make之前进行修正。
这也是调试Makefile或者查找源文件错误的一种非常有效的手段。


\section{make的命令行选项}
本节罗列了make 所支持的命令行参数(这些参数可以通过man手册查看):


\begin{dinglist}{47}
\itemsep=4pt
  \item \begin{minipage}[t]{\linewidth}
          \textbf{-b} \\
          \textbf{-m}
        \end{minipage}

忽略,提供其它版本make兼容性。

  \item \begin{minipage}[t]{\linewidth}
          \textbf{-B} \\
          \textbf{-{}-always-make}
        \end{minipage}

强制重建所有规则的目标,不根据规则的依赖描述决定是否重建目标文件。

  \item \begin{minipage}[t]{\linewidth}
          \textbf{-C DIR} \\
          \textbf{-{}-directory=DIR}
        \end{minipage}

在读取Makefile之前,进入目录“DIR”,就是切换工作目录到“DIR”之后执行make。存在
多个“-C”选项时,make的最终工作目录是第一个目录的相对路径。如:“make -C / -C
etc”等价于“make -C /etc”。一般此选项被用在递归地make调用中。

  \item \begin{minipage}[t]{\linewidth}
          \textbf{-d}
        \end{minipage}

make在执行过程中打印出所有的调试信息。包括:make认为那些文件需要重建;那些文
件需要比较它们的最后修改时间、比较的结果;重建目标所要执行的命令;使用的隐含
规则等。使用“-d”选项我们可以看到make构造依赖关系链、重建目标过程的所有信息,
它等效于“-{}-debug=a”。

  \item \begin{minipage}[t]{\linewidth}
          \textbf{-{}-debug[=OPTIONS]}
        \end{minipage}

make执行时输出调试信息。可以使用“OPTIONS”控制调试信息级别。默认是%
“OPTIONS=\\%
b”,“OPTIONS”的可能值为以下这些,首字母有效(all 和 aw等效)。

\begin{itemize}
\item a(all)

输出所有类型的调试信息,等效于“-d”选项。

\item b(basic)

输出基本调试信息。包括:那些目标过期、是否重建成功过期目标文件。

\item v(verbose)

“basic”级别之上的输出信息。包括:解析的makefile文件名,不需要重建文件等。此选项目默认打开“basic”级别的调试信息。

\item i(implicit)

输出所有使用到的隐含规则描述。此选项目默认打开“basic”级别的调试信息。

\item j(jobs)

输出所有执行命令的子进程,包括命令执行的PID等。

\item m(makefile)

也就是makefile,输出make读取makefile,更新makefile,执行makefile的信息。

\end{itemize}

  \item \begin{minipage}[t]{\linewidth}
          \textbf{-e} \\
          \textbf{-{}-environment-overrides}
        \end{minipage}

使用系统环境变量的定义覆盖Makefile中的同名变量定义。

  \item \begin{minipage}[t]{\linewidth}
          \textbf{-f=FILE} \\
          \textbf{-{}-file=FILE} \\
          \textbf{-{}-makefile=FILE}
        \end{minipage}

指定“FILE”为make执行的makefile文件。

  \item \begin{minipage}[t]{\linewidth}
          \textbf{-h} \\
          \textbf{-{}-help}
        \end{minipage}

打印帮助信息。

  \item \begin{minipage}[t]{\linewidth}
          \textbf{-i} \\
          \textbf{-{}-ignore-errors}
        \end{minipage}

执行过程中忽略规则命令执行的错误。

  \item \begin{minipage}[t]{\linewidth}
          \textbf{-I DIR} \\
          \textbf{-{}-include-dir=DIR}
        \end{minipage}

指定被包含makefile文件的搜索目录。在Makefile中出现“include”另外一个文件时,将
在“DIR”目录下搜索。多个“-I”指定目录时,搜索目录按照指定顺序进行。

  \item \begin{minipage}[t]{\linewidth}
          \textbf{-j  [JOBS]} \\
          \textbf{-{}-jobs[=JOBS]}
        \end{minipage}

指定可同时执行的命令数目。在没有指定“-j”参数的情况下,执行的命令数目将是系统
允许的最大可能数目。存在多个“-j”参数时,尽最后一个“-j”指定的数目(“JOBS”)有
效。

  \item \begin{minipage}[t]{\linewidth}
          \textbf{-k} \\
          \textbf{-{}-keep-going}
        \end{minipage}

执行命令错误时不终止make的执行,make尽最大可能的执行所有的命令,直到出现致命
错误才终止。

  \item \begin{minipage}[t]{\linewidth}
          \textbf{-l LOAD} \\
          \textbf{-{}-load-average[=LOAD]}\\
          \textbf{-{}-max-load[=LOAD]}
        \end{minipage}

告诉make当存在其它任务在执行时,如果系统负荷超过“LOAD”(浮点数表示的),不再
启动新任务。没有指定“LOAD”的“-I”选项将取消之前“-I”指定的限制。

  \item \begin{minipage}[t]{\linewidth}
          \textbf{-n} \\
          \textbf{-{}-just-print}\\
          \textbf{-{}-dry-run}\\
          \textbf{-{}-recon}
        \end{minipage}

只打印出所要执行的命令,但不执行命令。

  \item \begin{minipage}[t]{\linewidth}
          \textbf{-o FILE} \\
          \textbf{-{}-old-file=FILE}\\
          \textbf{-{}-assume-old=FILE}
        \end{minipage}

指定文件“FILE”不需要重建,即使相对于它的依赖已经过期;同时也不重建依赖于此文
件任何文件(目标文件)。注意:此参数不会通过变量“MAKEFLAGS”传递给子make进程。

  \item \begin{minipage}[t]{\linewidth}
          \textbf{-p} \\
          \textbf{-{}-print-data-base}
        \end{minipage}

命令执行之前,打印出make读取的Makefile的所有数据(包括规则和变量的值),同时
打印出make的版本信息。如果只需要打印这些数据信息(不执行命令)可以使用“make
-qp”命令。查看make执行前的预设规则和变量,可使用命令“make –p -f /dev/null”。

  \item \begin{minipage}[t]{\linewidth}
          \textbf{-q} \\
          \textbf{-{}-question}
        \end{minipage}

称为“询问模式”;不运行任何命令,并且无输出。make只是返回一个查询状态。返回状
态为0表示没有目标需要重建,1表示存在需要重建的目标,2表示有错误发生。

  \item \begin{minipage}[t]{\linewidth}
          \textbf{-r} \\
          \textbf{-{}-no-builtin-rules}
        \end{minipage}

取消所有内嵌的隐含规则,不过你可以在Makefile中使用模式规则来定义规则。同时选
项“-r”会取消所有支持后追规则的隐含后缀列表,同样我们也可以在Makefile中使用
“.SUFFIXES”定义我们自己的后缀规则。“-r”选项不会取消make内嵌的隐含变量。

  \item \begin{minipage}[t]{\linewidth}
          \textbf{-R} \\
          \textbf{-{}-no-builtin-variabes}
        \end{minipage}

取消make内嵌的隐含变量,不过我们可以在Makefile中明确定义某些变量。注意,“-R”
选项同时打开“-r”选项。因为没有了隐含变量,隐含规则将失去意义(隐含规则是以内
嵌的隐含变量为基础的)。

  \item \begin{minipage}[t]{\linewidth}
          \textbf{-s} \\
          \textbf{-{}-silent}\\
            \textbf{-{}-quiet}
        \end{minipage}

取消命令执行过程的打印。

  \item \begin{minipage}[t]{\linewidth}
          \textbf{-S} \\
          \textbf{-{}-no-keep-going}\\
            \textbf{-{}-stop}
        \end{minipage}

取消“-k”选项。在递归的make过程中子make通过“MAKEFLAGS”变量继承了上层的命令行选
项。我们可以在子make中使用“-S”选项取消上层传递的“-k”选项,或者取消系统环境变
量“MAKEFLAGS”中的“-k”选项。

  \item \begin{minipage}[t]{\linewidth}
          \textbf{-t} \\
          \textbf{-{}-touch}
        \end{minipage}

和Linux的touch命令实现功能相同,更新所有目标文件的时间戳到当前系统时间。防止
make对所有过时目标文件的重建。

  \item \begin{minipage}[t]{\linewidth}
          \textbf{-v} \\
          \textbf{-{}-version}
        \end{minipage}

查看make版本信息。

  \item \begin{minipage}[t]{\linewidth}
          \textbf{-w} \\
          \textbf{-{}-print-directory}
        \end{minipage}

在make进入一个目录读取Makefile之前打印工作目录。这个选项可以帮助我们调试
Makefile,跟踪定位错误。使用“-C”选项时默认打开这个选项。参考本节前半部分“-C”
选项的描述。

  \item \begin{minipage}[t]{\linewidth}
          \textbf{-{}-no-print-directory}
        \end{minipage}

取消“-w”选项。可以是用在递归的make调用过程中,取消“-C”参数的默认打开“-w”功
能。

  \item \begin{minipage}[t]{\linewidth}
          \textbf{-W FILE} \\
          \textbf{-{}-what-if=FILE}\\
          \textbf{-{}-new-file=FILE}\\
          \textbf{-{}-assume-file=FILE}
        \end{minipage}

设定文件“FILE”的时间戳为当前时间,但不改变文件实际的最后修改时间。此选项主要
是为实现了对所有依赖于文件“FILE”的目标的强制重建。

  \item \begin{minipage}[t]{\linewidth}
          \textbf{-{}-warn-undefined-variables}
        \end{minipage}

在发现Makefile中存在对没有定义的变量进行引用时给出告警信息。此功能可以帮助我
们调试一个存在多级套嵌变量引用的复杂Makefile。但是:我们建议在书写Makefile时
尽量避免超过三级以上的变量套嵌引用。

\end{dinglist}
