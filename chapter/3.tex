\chapter{Makefile 总述}

\section{Makefile的内容}
在一个完整的Makefile中,包含了5个东西:显式规则、隐含规
则、变量定义、指示符和注释。关于“规则”、“变量”和“Makefile指示符”将在后续的章
节进行详细的讨论。本章讨论的是一些基本概念。

\begin{itemize}\itemsep=0pt \parskip=0pt
  \item 显式规则:它描述了在何种情况下如何更新一个或者多个被称为目标的文件
      (Makefile的目标文件)。书写Makefile时需要明确地给出目标文件、目标的
      依赖文件列表以及更新目标文件所需要的命令(有些规则没有命令,这样的规
      则只是纯粹的描述了文件之间的依赖关系)。
  \item 隐含规则:它是make根据一类目标文件(典型的是根据文件名的后缀)而自
      动推导出来的规则。make根据目标文件的名,自动产生目标的依赖文件并使用
      默认的命令来对目标进行更新(建立一个规则)。
  \item 变量定义:使用一个字符或字符串代表一段文本串,当定义了一个变量以
      后,Makefile后续在需要使用此文本串的地方,通过引用这个变量来实现对文
      本串的使用。第一章的例子中,我们就定义了一个变量“objects”来表示一个.o
      文件列表。
  \item Makefile指示符:指示符指明在make程序读取makefile文件过程中所要执行
      的一个动作。其中包括:
  \begin{itemize}
    \item 读取一个文件,读取给定文件名的文件,将其内容作为makefile 文件的
        一部分。
    \item 决定(通常是根据一个变量的得值)处理或者忽略Makefile中的某一特
        定部分。
    \item 定义一个多行变量。
  \end{itemize}
  \item 注释:Makefile中“\#”字符后的内容被作为是注释内容(和shell 脚本一
      样)处理。如果此行的第一个非空字符为“\#”,那么此行为注释行。注释行的
      结尾如果存在反斜线(\verb"\"),那么下一行也被作为注释行。一般在书写
      Makefile 时推荐将注释作为一个独立的行,而不要和Makefile的有效行放在一
      行中书写。当在Makefile中需要使用字符“\#” 时,可以使用反斜线加“\#”
      (\#)来实现(对特殊字符“\#” 的转义),其表示将“\#”作为一字符而不是注
      释的开始标志。
\end{itemize}


\textbf{需要注意的地方:}

Makefile中第一个规则之后的所有以[Tab]字符开始的的行,make程序都会将其交给系统
shell程序去解释执行。因此,以[Tab]字符开始的注释行也会被交给shell来处理,此命
令行是否需要被执行(shell执行或者忽略)是由系统shell程序来判决的。

另外,在使用指示符“define”定义一个多行的变量或者命令包时,其定义体(“define”
和“endef”之间的内容)会被完整的展开到Makefile中引用此变量的地方(包含定义体中
的注释行);make在引用此变量的地方对所有的定义体进行处理,决定是注释还是有效
内容。Makefile中变量的引用和C语言中的宏类似(但是其实质并不相同,后续将会详细
讨论)。对一个变量引用的地方make所做的就是将这个变量根据定义进行基于文本的展
开,展开变量的过程不涉及到任何变量的具体含义和功能分析。


\section{makefile文件的命名}

默认的情况下,make会在工作目录(执行make的目录)下按照文件名顺序寻找makefile
文件读取并执行,查找的文件名顺序为:“GNUmakefile”、“makefile”、“Makefile”。

通常应该使用“makefile”或者“Makefile”作为一个makefile的文件名(我们推荐使用
“Makefile”,首字母大写而比较显著,一般在一个目录中和当前目录的一些重要文件(R
EADME,Chagelist等)靠近,在寻找时会比较容易的发现它)。而“GNUmakefile”是我们
不推荐使用的文件名,因为以此命名的文件只有“GNU make”才可以识别,而其他版本的
make程序只会在工作目录下“makefile”和“Makefile”这两个文件。

如果make程序在工作目录下无法找到以上三个文件中的任何一个,它将不读取任何其他
文件作为解析对象。但是根据make隐含规则的特性,我们可以通过命令行指定一个目
标,如果当前目录下存在符合此目标的依赖文件,那么这个命令行所指定的目标将会被
创建或者更新,参见注释。

当makefile文件的命名不是这三个任何一个时,需要通过make的“-f”或者“--file”选项
来指定make读取的makefile文件。给make指定makefile文件的格式为:“-f NAME”或者
“—file=NAME”,它指定文件“NAME”作为执行make时读取的makefile文件。也可以通过多
个“-f”或者“--file”选项来指定多个需要读取的makefile文件,多个makefile文件将会
被按照指定的顺序进行链接并被make解析执行。当通过“-f”或者“--file”指定make读取
makefile的文件时,make就不再自动查找这三个标准命名的makefile文件。

\begin{quote}\kaishu

\textbf{注释:}通过命令指定目标使用make的隐含规则:

当前目录下不存在以“GNUmakefile”、“makefile”、“Makefile”命名的任何文件,

1.当前目录下存在一个源文件foo.c的,我们可以使用“make foo.o”来使用make的隐含规
则自动生成foo.o。当执行“make foo.o”时。我们可以看到其执行的命令为:cc –c –o
foo.o foo.c之后,foo.o将会被创建或者更新。

2.        如果当前目录下没有foo.c文
件时,就是make对.o文件目标的隐含规则中依赖文件不存在。如果使用命令“make
foo.o”时,将回到到如下提示:make: *** No rule to make target ‘foo.o’. Stop.

3.        如果直接使用命令“make”时,得到的提示信息如下:make: *** No targets
specified and no makefile found. Stop.
\end{quote}

\section{包含其它makefile文件}
本节我们讨论如何在一个Makefile中包含其它的makefile文件。Makefile中包含其它文件所需要使用的关键字是“include”,和c语言对头文件的包含方式一致。
“include”指示符告诉make暂停读取当前的Makefile,而转去读取“include”指定的一个或者多个文件,完成以后再继续当前Makefile的读取。Makefile中指示符“include”书写在独立的一行,其形式如下:

\begin{Verbatim}[]
include FILENAMES...
\end{Verbatim}

FILENAMES是shell所支持的文件名(可以使用通配符)。
指示符“include”所在的行可以一个或者多个空格(make程序在处理时将忽略这些空格)开始,切忌不能以[Tab]字符开始(如果一行以[Tab]字符开始make程序将此行作为一个命令行来处理)。指示符“include”和文件名之间、多个文件之间使用空格或者[Tab]键隔开。行尾的空白字符在处理时被忽略。使用指示符包含进来的Makefile中,如果存在变量或者函数的引用。它们将会在包含它们的Makefile中被展开(详细可参考 第六章 Makefile中的变量)。
来看一个例子,存在三个.mk文件a.mk、b.mk、c.mk,“\$(bar)”被扩展为“bish bash”。则
\begin{Verbatim}[]
include foo *.mk $(bar)
\end{Verbatim}
等价于
\begin{Verbatim}[]
include foo a.mk b.mk c.mk bish bash
\end{Verbatim}

之前已经提到过make程序在处理指示符include时,将暂停对当前使用指示符“include”
的makefile文件的读取,而转去依此读取由“include”指示符指定的文件列表。直到完成
所有这些文件以后再回过头继续读取指示符“include”所在的makefile文件。

通常指示符“include”用在以下场合:

\begin{enumerate}
\itemsep=0pt \parskip=0pt
\item 有多个不同的程序,由不同目录下的几个独立的Makefile来描述其重建规则。
    它们需要使用一组通用的变量定义或者模式规则。通用的做法是将这些共同使用
    的变量或者模式规则定义在一个文件中(没有具体的文件命名限制),在需要使
    用的Makefile中使用指示符“include”来包含此文件。
\item 当根据源文件自动产生依赖文件时;我们可以将自动产生的依赖关系保存在另
    外一个文件中,主Makefile 使用指示符“include”包含这些文件。这样的做法比
    直接在主Makefile 中追加依赖文件的方法要明智的多。其它版本的make已经使用
    这种方式来处理。
\end{enumerate}

如果指示符“include”指定的文件不是以斜线开始(绝对路径,如/
usr/src/Makefile...),而且当前目录下也不存在此文件;make将根据文件名试图在以
下几个目录下查找:首先,查找使用命令行选项“-I”或者“--include-dir”指定的目录,
如果找到指定的文件,则使用这个文件;否则继续依此搜索以下几个目录(如果其存
在):“/usr/gnu/include”、“/usr/local/include”和“/usr/include”。

当在这些目录下都没有找到“include”指定的文件时,make将会提示一个包含文件未找到
的告警提示,但是不会立刻退出。而是继续处理Makefile的后续内容。当完成读取整个
Makefile后,make将试图使用规则来创建通过指示符“include”指定的但未找到的文件,
当不能创建它时(没有创建这个文件的规则),make将提示致命错误并退出。会输出类
似如下错误提示:

\begin{Verbatim}[]
Makefile:错误的行数:未找到文件名:提示信息(No such file or directory)
Make: *** No rule to make target ‘<filename>’. Stop
\end{Verbatim}

通常我们在Makefile中可使用“-include”来代替“include”,来忽略由于包含文件不存在
或者无法创建时的错误提示(“-”的意思是告诉make,忽略此操作的错误。make继续执
行)。像下边那样:
\begin{Verbatim}[]
-include FILENAMES...
\end{Verbatim}

使用这种方式时,当所要包含的文件不存在时不会有错误提示、make也不会退出;除此
之外,和第一种方式效果相同。以下是这两种方式的比较:

使用“include FILENAMES...”,make程序处理时,如果“FILENAMES”列表中的任何一个文
件不能正常读取而且不存在一个创建此文件的规则时make程序将会提示错误并退出。

使用“-include FILENAMES...”的情况是,当所包含的文件不存在或者不存在一个规则去
创建它,make程序会继续执行,只有真正由于不能正确完成终极目标的重建时(某些必
需的目标无法在当前已读取的makefile文件内容中找到正确的重建规则),才会提示致
命错误并退出。为了和其它的make程序进行兼容。也可以使用“sinclude”来代替
“-include”(GNU所支持的方式)。

\section{变量 MAKEFILES}
如果在当前环境定义了一个“MAKEFILES”环境变量,make执行时
首先将此变量的值作为需要读入的Makefile文件,多个文件之间使用空格分开。类似使
用指示符“include”包含其它Makefile文件一样,如果文件名非绝对路径而且当前目录也
不存在此文件,make会在一些默认的目录去寻找。它和使用“include”的区别:

\begin{enumerate}
\itemsep=0pt \parskip=0pt
\item 环境变量指定的makefile文件中的“目标”不会被作为make执行的“终极目标”。
    就是说,这些文件中所定义规则的目标,make不会将其作为“终极目标”来看待。
    如果在make的工作目录下没有一个名为“Makefile”、“makefile”或者
    “GNUmakefile”的文件,make同样会提示“make: *** No targets specified and
    no makefile found. Stop.”;而在make的工作目录下存在这样一个文件(“M
    akefile”、“makefile”或者“GNUmakefile”),那么make执行时的“终极目标”就是
    当前目录下这个文件中所定义的“终极目标”。
\item 环境变量所定义的文件列表,在执行make时,如果不能找到其中某一个文件
    (不存在或者无法创建)。make不会提示错误,也不退出。就是说环境变量
    “MAKEFILES”定义的包含文件是否存在不会导致make错误(这是比较隐蔽的地
    方)。
\item make在执行时,首先读取的是环境变量“MAKEFILES”所指定的文件列表,之后才
    是工作目录下的makefile文件,“include”所指定的文件是在make发现此关键字的
    时、暂停正在读取的文件而转去读取“include”所指定的文件。
\end{enumerate}

变量“MAKEFILES”主要用在“make”的递归调用过程中的的通信。实际应用中很少设置此变
量。因为一旦设置了此变量,在多级make调用时;由于每一级make都会读取“MAKEFILES”
变量所指定的文件,将导致执行出现混乱(这可能不是你想看到的执行结果)。不过,
我们可以使用此环境变量来指定一个定义了通用“隐含规则”和变量的文件,比如设置默
认搜索路径;通过这种方式设置的“隐含规则”和定义的变量可以被任何make进程使用
(有点象C语言中的全局变量)。

也有人想让login程序自动的在自己的工作环境中设置此环境变量,编写的Makefile建立
在此环境变量的基础上。此想法可以肯定地说不是一个好主意。规劝大家千万不要这么
干,否则你所编写的Makefile在其人的工作环境中肯定不能正常工作。因为别人的工作
环境中可能没有设置相同的环境变量“MAKEFILES”。推荐的做法实:在需要包含其它
makefile文件时使用指示符“include”来实现。


\section{变量 MAKEFILE\_LIST}

make程序在读取多个makefile文件时,包括由环境变量“MAKEFILES”指定、命令行指、当
前工作下的默认的以及使用指示符“include”指定包含的,在对这些文件进行解析执行之
前make读取的文件名将会被自动依次追加到变量“MAKEFILE\_LIST”的定义域中。

这样我们就可以通过测试此变量的最后一个字来获取当前make程序正在处理的makefile
文件名。具体地说就是在一个makefile文件中如果使用指示符“include”包含另外一个文
件之后,变量“MAKEFILE\_LIST”的最后一个字只可能是指示符“include”指定所要包含的
那个文件的名字。一个makefile的内容如下:
\begin{Verbatim}[]
name1 := $(word $(words $(MAKEFILE_LIST)),$(MAKEFILE_LIST))
include inc.mk
name2 := $(word $(words $(MAKEFILE_LIST)),$(MAKEFILE_LIST))
all:
    @echo name1 = $(name1)
    @echo name2 = $(name2)
\end{Verbatim}

执行make,则看到的将是如下的结果:
\begin{Verbatim}[]
name1 = Makefile
name2 = inc.mk
\end{Verbatim}

此例子中涉及到了make的函数的和变量定义的方式,这些将在后续的章节中有详细的讲述。


\section{其他特殊变量}
GNU make支持一个特殊的变量,此变量不能通过任何途经给它赋值。它被展开为一个特
定的值。一个重要的特殊的变量是“.VARIABLES”。它被展开以后是此引用点之前、
makefile文件中所定义的所有全局变量列表。包括:空变量(未赋值的变量)和make的
内嵌变量,但不包含目标指定的变量,目标指定变量值在特定目标的上下文有效。

\section{makefile文件的重建}

Makefile可由其它文件生成,例如RCS或SCCS文件。如果Makefile由其它文件重建,那么
在make在开始解析这个Makefile时需要重新读取更新后的Makefile、而不是之前的
Makefile。make的处理过程是这样的:

make在读入所有makefile文件之后,首先将所读取的每个makefile作为一个目标,寻找
更新它们的规则。如果存在一个更新某一个makefile文件明确规则或者隐含规则,就去
更新对应的makefile文件。完成对所有的makefile文件的更新之后,如果之前所读取任
何一个makefile文件被更新,那么make就清除本次执行的状态重新读取一遍所有的
makefile文件(此过程中,同样在读取完成以后也会去试图更新所有的已经读取的
makefile文件,但是一般这些文件不会再次被重建,因为它们在时间戳上已经是最新
的)。读取完成以后再开始解析已经读取的makefile文件并开始执行必要的动作。

实际应用中,我们会明确给出makefile文件,而并不需要来由make自动重建它们。但是
make在每一次执行时总会自动地试图重建那些已经存在的makefile文件,如果需要处于
效率考虑,可以采用一些办法来避免make在执行过程时查找重建makefile的隐含规则。
例如我们可以书写一个明确的规则,以makefile文件作为目标,规则的命令定义为空。

Makefile规则中,如果使用一个没有依赖只有命令行的双冒号规则去更新一个文件,那
么每次执行make时,此规则的目标文件将会被无条件的更新(此规则定义的命令会被无
条件执行)。如果这样一个规则的目标是makefile文件,那么执行make时,这个
makefile文件(双冒号规则的目标)就会被无条件更新,而使得make的执行陷入到一个
死循环(此makefile文件被不断的更新、重新读取、更新再重新读取的过程)。为了防
止这种情况的发生,make在遇到一个目标是makefile文件的双冒号规则时,将忽略对这
个规则的执行(其中包括了使用“MAKEFILES”指定、命令行选项指定、指示符“include”
指定的需要make读取的所有makefile文件中定义的这一类双冒号规则)。

执行make时,如果没有使用“-f(--file)”选项指定一个文件,make程序将读取缺省的
文件。和使用“-f(--file)”选项不同,make无法确定工作目录下是否存在缺省名称的
makefile文件。如果缺省makefile文件不存在,但可以通过一个规则来创建它(此规则
是隐含规则),则会自动创建缺省makefile文件,之后重新读取它并开始执行。

因此,如果在当前目录下不存在一个缺省的makefile文件,make将会按照搜索makefile
文件的名称顺序去试图创建它,直到创建成功或者超越其缺省的命名顺序。需要明确的
一点是:执行make时,如果不能成功地创建缺省的makefile文件,并不一定会导致错
误。一个存在(缺省命名的或者可被创建的)的makefile文件并不是make正确运行的前
提(关于这一点大家会在后续的阅读过程中体会到)。

当使用“-t(--touch)”选项来更新Makefile的目标文件(更新规则目标文件的时间戳)
时,对于哪些是makefile文件的目标是无效的,这些目标文件(makefile文件)的时间
戳并不会被更新。就是说即使在执行make时使用了选项“-t”,那些目标是makefile文件
的规则同样也会被执行(重建这些makefile文件,而其它的规则不会被执行,make只是
简单的更新规则目标文件的时间戳);类似还有选项“-q(—question)”和“-n(—
just-print) ”,这主要是因为一个过时的makefile文件对其它目标的重建规则在当前
看来可能是错误的。

正因为如此,执行命令“make –f mfile –n foo”首先会试图重建“mfile文件”、并重新读
取它,之后会打印出更新目标“foo”所要执行的命令(但不会真正的执行这些命令)。在
这种情况时,如果不希望重建makefile文件。我们需要在执行make时,在命令行中将这
个makefile文件作为一个最终目标,这样选项“–t”和其它的选项就对这个makefile文件
目标有效,防止执行这个makefile作为目标的规则(如果是“-t”参数,则是简单的更新
这个makefile文件的时间戳)。同样,命令“make –f mfile –n mfile foo”会读取文件
“mfile”,打印出重建文件“mfile”的命令、重建“foo”的命令而实际不执行任何命令。并
且所打印的用于更新“foo”目标的命令是选项“-f”指定的、没有被重建的“mfile”文件中
所定义的命令。


\section{重载另外一个makefile}
有些情况下,存在两个比较类似的makefile文件。其中一个(makefile-A)需要使用另
外一个(makefile-B)中所定义的变量和规则。通常我们会想到在“makefile-A”中使用
指示符“include”包含“mkaefile-B”来达到目的。但使用这种方式,如果在两个makefile
文件中存在相同目标,而在不同的文件中其描述规则使用不同的命令。这样,相同的目
标文件就同时存在两个不同的规则命令,这是makefile所不允许的。遇到这种情况,使
用指示符“include”显然是行不通的。GNU make提供另外一种途径来实现此目的。具体的
做法如下:

在需要包含的makefile文件(makefile-A)中,定义一个称之为“所有匹配模式”的规
则,它用来述那些在“makefile-A”中没有给出明确创建规则的目标的重建规则。就是
说,如果在当前makefile文件中不能找到重建一个目标的规则时,就使用“所有匹配模
式”所在的规则来重建这个目标。

看一个例子,如果存在一个命名为“Makefile”的makefile文件,其中描述目标“foo”的规
则和其他的一些规,我们也可以书写一个内容如下命名为“GNUmakefile”的文件。

\begin{Verbatim}[]
#sample GNUmakefile
foo:
    frobnicate > foo
%: force
    @$(MAKE) -f Makefile $@
force: ;
\end{Verbatim}

执行命令“make foo”,make将使用工作目录下命名为“GNUmakefile”的文件并执行目标
“foo”所在的规则,创建目标“foo”的命令是:“\verb"frobnicate > foo"”。如果执行另
外一个命令“make bar”,因为在“GUNmakefile”中没有此目标的更新规则。make将使用
“所有匹配模式”规则,执行命令“\verb"$(MAKE) -f Makefile bar"”。如果文件
“Makefile”中存在此目标更新规则的定义,那么这个规则会被执行。此过程同样适用于
其它“GNUmakefile”中没有给出的目标更新规则。此方式的灵活之处在于:如果在
“Makefile”文件中存在同样一个目标“foo”的重建规则,由于make执行时首先读取文件
“GUNmakefile”并在其中能够找到目标“foo”的重建规则,所以make就不会去执行这个“所
有模式匹配规则”(上例中目标“\%”所在的规则)。这样就避免了使用指示符“include”
包含一个makefile文件时所带来的目标规则的重复定义问题。

此种方式,模式规则的模式只使用了单独的“\%”(我们称他为“所有模式匹配规则”),
它可以匹配任何一个目标;它的依赖是“force”,保证了即使目标文件已经存在也会执行
这个规则(文件已存在时,需要根据它的依赖文件的修改情况决定是否需要重建这个目
标文件);“force”规则中使用空命令是为了防止make程序试图寻找一个规则去创建目标
“force”时,又使用了模式规则“\%: force”而陷入无限循环。

\section{make如何解析makefile文件}
GUN make的执行过程分为两个阶段。

\begin{dinglist}{226}
\itemsep=4pt \parskip=0pt

\item \textbf{第一阶段}:读取所有的makefile文件(包括“MAKIFILES”变量指定的、
    指
    示符“include”指定的、以及命令行选项“-f(--file)”指定的makefile文件),内建
    所有的变量、明确规则和隐含规则,并建立所有目标和依赖之间的依赖关系结构链
    表。

\item \textbf{第二阶段}:根据第一阶段已经建立的依赖关系结构链表决定哪些目标需
    要更新,并使用对应的规则来重建这些目标。
\end{dinglist}

理解make执行过程的两个阶段是很重要的。它能帮助我们更深入的了解执行过程中变量
以及函数是如何被展开的。变量和函数的展开问题是书写Makefile时容易犯错和引起大
家迷惑的地方之一。本节将对这些不同的结构的展开阶段进行简单的总结(明确变量和
函数的展开阶段,对正确的使用变量非常有帮助)。首先,明确以下基本的概念;在
make执行的第一阶段中如果变量和函数被展开,那么称此展开是“立即”的,此时所有的
变量和函数被展开在需要构建的结构链表的对应规则中(此规则在建立链表是需要使
用)。其他的展开称之为“延后”的。这些变量和函数不会被“立即”展开,而是直到后续
某些规则须要使用时或者在make处理的第二阶段它们才会被展开。

可能现在讲述的这些还不能完全理解。不过没有关系,通过后续章节内容的学习,我们
会一步一步的熟悉make的执行过程。学习过程中可以回过头来参考本节的内容。相信在
看完本书之后,会对make的整个过程有全面深入的理解。

\subsection{变量取值}变量定义解析的规则如下:
\begin{Verbatim}[]
IMMEDIATE = DEFERRED
IMMEDIATE ?= DEFERRED
IMMEDIATE := IMMEDIATE
IMMEDIATE += DEFERRED or IMMEDIATE
define IMMEDIATE
DEFERRED
Endef
\end{Verbatim}
当变量使用追加符(+=)时,如果此前这个变量是一个简单变量(使用 :=定义的)则认
为它是立即展开的,其它情况时都被认为是“延后”展开的变量。


\subsection{条件语句}
所有使用到条件语句在产生分支的地方,make程序会根据预设条件将正确地分支展开。
就是说条件分支的展开是“立即”的。其中包括:“ifdef”、“ifeq”、“ifndef”和“ifneq”
所确定的所有分支命令。

\subsection{规则的定义}
所有的规则在make执行时,都按照如下的模式展开:
\begin{Verbatim}[]
IMMEDIATE : IMMEDIATE ; DEFERRED DEFERRED
\end{Verbatim}


其中,规则中目标和依赖如果引用其他的变量,则被立即展开。而规则的命令行中的变
量引用会被延后展开。此模板适合所有的规则,包括明确规则、模式规则、后缀规则、
静态模式规则。

\section{总结}
make的执行过程如下:
\begin{enumerate}
\itemsep=0pt \parskip=0pt
\item       依次读取变量“MAKEFILES”定义的makefile文件列表。
\item 读取工作目录下的
    makefile文件(根据命名的查找顺序“GNUmakefile”,“makefile”,“Makefile”,
    首先找到那个就读取那个)。
\item        依次读取工作目录makefile文件中使用指
    示符“include”包含的文件。
    \item       查找重建所有已读取的makefile文件的规则
    (如果存在一个目标是当前读取的某一个makefile文件,则执行此规则重建此
    makefile文件,完成以后从第一步开始重新执行)。
    \item      初始化变量值并展
    开那些需要立即展开的变量和函数并根据预设条件确定执行分支。
    \item 根据“终极目
    标”以及其他目标的依赖关系建立依赖关系链表。
    \item        执行除“终极目标”以外
    的所有的目标的规则(规则中如果依赖文件中任一个文件的时间戳比目标文件
    新,则使用规则所定义的命令重建目标文件)。
    \item      执行“终极目标”所在的
    规则。
\end{enumerate}

执行一个规则的过程是这样的:对于一个存在的规则(明确规则和隐含规则)首先,
make程序将比较目标文件和所有的依赖文件的时间戳。如果目标的时间戳比所有依赖文
件的时间戳更新(依赖文件在上一次执行make之后没有被修改),那么什么也不做。否
则(依赖文件中的某一个或者全部在上一次执行make后已经被修改过),规则所定义的
重建目标的命令将会被执行。这就是make工作的基础,也是其执行规制所定义命令的依
据。(后续讨论规则时将会对此详细地说明)。
