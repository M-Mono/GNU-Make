\chapter{make的隐含规则}
在Makefile中重建一类目标的标准规则在很多场合需要用到。例如:根据.c源文件创建
对应的.o文件,传统方式是使用GNU 的C编译器。

“隐含规则”为make提供了重建一类目标文件通用方法,不需要在Makefile中明确地给出
重建特定目标文件所需要的细节描述。例如:典型地;make对C文件的编译过程是由.c源
文件编译生成.o目标文件。当Makefile中出现一个.o文件目标时,make会使用这个通用
的方式将后缀为.c的文件编译称为目标的.o文件。

另外,在make执行时根据需要也可能是用多个隐含规则。比如:make将从一个.y文件生
成对应的.c文件,最后再生成最终的.o文件。就是说,只要目标文件名中除后缀以外其
它部分相同,make都能够使用若干个隐含规则来最终产生这个目标文件(当然最原始的
那个文件必须存在)。例如;可以在Makefile中这样来实现一个规则:“foo : foo.h”,
只要在当前目录下存在“foo.c”这个文件,就可以生成“foo”可执行文件。本文前边的很
多例子中已经使用到了隐含规则。

内嵌的“隐含规则”在其所定义的命令行中,会使用到一些变量(通常也是内嵌变量)。
我们可以通过改变这些变量的值来控制隐含规则命令的执行情况。例如:内嵌变量
“CFLAGS”代表了gcc编译器编译源文件的编译选项,我们就可以在Makefile中重新定义
它,来改变编译源文件所要使用的参数。

尽管我们不能改变make内嵌的隐含规则,但是我们可以使用模式规则重新定义自己的隐
含规则,也可以使用后追规则来重新定义隐含规则。后缀规则存在某些限制(目前版本
make保存它的原因是为了兼容以前版本)。使用模式规则更加清晰明了。

\section{隐含规则的使用}
使用make内嵌的隐含规则,在Makefile中就不需要明确给出重建某一个目标的命令,甚
至可以不需要规则。make会自动根据已存在(或者可以被创建)的源文件类型来启动相
应的隐含规则。例如:

\begin{Verbatim}[]
foo : foo.o bar.o
    cc -o foo foo.o bar.o $(CFLAGS) $(LDFLAGS)
\end{Verbatim}

这里并没有给出重建文件“foo.o”的规则,make执行这条规则时,无论文件“foo.o”存在
与否,都会试图根据隐含规则来重建这个文件(就是试图重新编译文件“foo.c”或者其它
类型的源文件)。

make执行过程中找到的隐含规则,提供了此目标的基本依赖关系,确定了目标的依赖文
件(通常是源文件,不包含对应的头文件依赖)和重建目标需要使用的命令行。隐含规
则所提供的依赖文件只是一个最基本的(通常它们之间的对应关系为:“EXENAME.o”对应
“EXENAME.c”、“EXENAME”对应于“EXENAME.o”)。当需要增加这个目标的依赖文件时,要
在Makefile中使用没有命令行的规则给出。

每一个内嵌的隐含规则中都存在一个目标模式和依赖模式,而且同一个目标模式可以对
应多个依赖模式。例如:一个.o文件的目标可以由c编译器编译对应的.c源文件得到、
Pascal编译器编译.p的源文件得到,等等。make会根据不同的源文件来使用不同的编译
器。对于“foo.c”就是用c编译,对于“foo.p”就使用Pascal编译器编译。

上边提到,make会自动根据已存在(或者可以被创建)的源文件类型来启动相应的隐含
规则。这里的“可被创建”文件是指:这个文件在Makefile中被作为目标或者依赖明确的
提及,或者可以根据已存在的文件使用其它的隐含规则来创建它。当一个隐含规则的目
标是另外一个隐含规则的依赖时,我们称它们是一个隐含规则链。

通常,make会对那些没有命令行的规则、双冒号规则寻找一个隐含规则来执行。作为一
个规则的依赖文件,在没有一个规则明确描述它的依赖关系的情况下;make会将其作为
一个目标并为它搜索一个隐含规则,试图重建它。

注意:给目标文件指定明确的依赖文件并不会影响隐含规则的搜索。我们来看一个例
子:

\begin{Verbatim}[]
foo.o: foo.p
\end{Verbatim}

这个规则指定了“foo”的依赖文件是“foo.p”。但是如果在工作目录下存在同名.c源文件
“foo.c”。执行make的结果就不是用“pc”编译“foo.p”来生成“foo”,而是用“cc”编译
“foo.c”来生成目标文件。这是因为在隐含规则列表中对.c文件的隐含规则处于.p文件隐
含规则之前。

当需要给目标指定明确的重建规则时,规则描述中就不能省略命令行,这个规则必须提
供明确的重建命令来说明目标需要重建所需要的动作。为了能够在存在“foo.c”的情况下
编译“foo.p”。规则可以这样写:
\begin{Verbatim}[]
foo.o: foo.p
    pc $< -o $@
\end{Verbatim}

这一点在多语言实现的工程编译中,需要特别注意!否则编译出来的可能就不是你想要
得程序。

另外:当我们不想让make为一个没有命令行的规则中的目标搜索隐含规则时,我们需要
使用空命令来实现。

最后让我们来看一个简单的例子,之前在目标指定变量 一节的例子我们就可以简化为:

\begin{Verbatim}[]
# sample Makefile

CUR_DIR = $(shell pwd)
INCS := $(CUR_DIR)/include
CFLAGS := -Wall –I$(INCS)

EXEF := foo bar

.PHONY : all clean
all : $(EXEF)

foo : CFLAGS+=-O2
bar : CFLAGS+=-g

clean :
    $(RM) *.o *.d $(EXES)
\end{Verbatim}

例子中没有出现任何关于源文件的描述。所有剩余工作全部交给了make去处理,它会自
动寻找到相应规则并执行、最终完成目标文件的重建。

隐含规则为我们提供了一个编译整个工程非常高效的手段,一个大的工程中毫无例外的
会用到隐含规则。实际工作中,灵活运用GNU make所提供的隐含规则功能,可以大大提
供效率。

\section{make的隐含规则一览}
本节罗列出了GUN make常见的一些内嵌隐含规则,除非在Makefile有名确定义、或者使%
用命令行“-r”或者“-R”参数取消隐含规则,否则这些隐含规则将有效。%

需要说明的是:即使我们没有使用命令行参数“-r”,在make中也并不是所有的这些隐含%
规则都被定义了。其实,很多的这些看似预定义的隐含规则在make执行时,实际是用后%
缀规则来实现的;因此,它们依赖于make中的“后缀列表”(也就是目标.SUFFIXES的后缀%
列表)。make的默认后缀列表为:%
.out、.a、.ln、.o、.c、.cc、.C、.p、.f、.F、.r、.y、.l、.s、.S、.mod、.sym、.def、.h、.info、.dvi、.tex、.texinfo、.texi、txinfo、.w、.ch、.web、.sh、.elc、el。
所有我们下边将提到的隐含规则,如果其依赖文件中某一个满足列表中列出的后缀,则%
是后缀规则。如果修改了可识别后缀列表,那么可能会是许多默认预定义的规则无效%
(因为一些后缀可能不会别识别)。以下是常用的一些隐含规则(对于不常见的隐含规%
则这里没有描述,其余规则查看帮助文档或者通过命令:make -q 查看):

\begin{dinglist}{47}
\itemsep=4pt

\item \textbf{编译C程序}

“N.o”自动由“N.c” 生成,执行命令为“\$(CC) -c \$(CPPFLAGS) \$(CFLAGS)”。

\item \textbf{编译C++程序}

“N.o”自动由“N.cc”或者“N.C” 生成,执行命令为“\$(CXX) -c \$(CPPFLAGS)
\$(CFLAGS)”。建议使用“.cc”作为C++源文件的后缀,而不是“.C”。

\item \textbf{编译Pascal程序}

“N.o”自动由“N.p”创建,执行命令时“\$(PC) -c \$(PFLAGS)”。

\item \textbf{编译Fortran/Ratfor程序}

“N.o”自动由“N.r”、“N.F”或者“N.f” 生成,根据源文件后缀执行对应的命令:

\begin{Verbatim}[]
.f — “$(FC) –c  $(FFLAGS)”
.F — “$(FC) –c  $(FFLAGS) $(CPPFLAGS)”
.r — “$(FC) –c  $(FFLAGS) $(RFLAGS)”
\end{Verbatim}

\item \textbf{预处理Fortran/Ratfor程序}

“N.f”自动由“N.r”或者“N.F” 生成。此规则只是转换Ratfor或有预处理的Fortran程序到
一个标准的Fortran程序。根据源文件后缀执行对应的命令:

\begin{Verbatim}[]
.F — “$(FC) –F $(CPPFLAGS) $(FFLAGS)”
.r — “$(FC) –F $(FFLAGS) $(RFLAGS)”
\end{Verbatim}

\item \textbf{编译Modula-2程序}

“N.sym”自动由“N.def” 生成,执行的命令是:“\$(M2C) \$(M2FLAGS)
\$(DEFFLAGS)”。“N.o”自动由“N.mod”生成,执行的命令是:“\$(M2C) \$(M2FLAGS)
\$(MODFLAGS)”。

\item \textbf{汇编和需要预处理的汇编程序}

“N.s”是不需要预处理的汇编源文件,“N.S”是需要预处理的汇编源文件。汇编器为
“as”。 “N.o” 可自动由“N.s”生成,执行命令是:“\$(AS) \$(ASFLAGS)”。 “N.s” 可由
“N.S”生成,C预编译器“cpp”,执行命令是:“\$(CPP) \$(CPPFLAGS)”。

\item  \textbf{链接单一的object文件}

“N”自动由“N.o”生成,通过C编译器使用链接器(GUN ld),执行命令是:“\$(CC)
\$(LDFLAGS) N.o \$(LOADLIBES) \$(LDLIBS)”。

此规则仅适用:由一个源文件直接产生可执行文件的情况。当需要有多个源文件共同来
创建一个可执行文件时,需要在Makefile中增加隐含规则的依赖文件。例如:

\begin{Verbatim}[]
x : y.o z.o
\end{Verbatim}

当“x.c”、“y.c”和“z.c”都存在时,规则执行如下命令:
\begin{Verbatim}[]
cc -c x.c -o x.o
cc -c y.c -o y.o
cc -c z.c -o z.o
cc x.o y.o z.o -o x
rm -f x.o
rm -f y.o
rm -f z.o
\end{Verbatim}

在复杂的场合,目标文件和源文件之间不存在向上边那样的名字对应关系时(“x”和
“x.c”对应,因此隐含规则在进行链接时,自动将“x.c”作为其依赖文件)。这时,需要
在Makefile中明确给出描述目标依赖关系的规则。

通常,gcc在编译源文件时(根据源文件的后缀名来启动相应的编译器),如果没有指定
“-c”选项,gcc会在编译完成之后调用“ld”连接成为可执行文件。

\item \textbf{Yacc C程序}

“N.c”自动由“N.y”,执行的命令:“\$(YACC) \$(YFALGS)”。(“Yacc”是一个语法分析工
具)。

\item \textbf{Lex C程序时的隐含规则}

“N.c”自动由“N.l”,执行的命令是:“\$(LEX) \$(LFALGS)”。(关于“Lex”的细节请查看
相关资料)。


\end{dinglist}

在隐含规则中,命令行中的实际命令是使用一个变量计算得到,诸如:
``COMPILE.c"、``LINK.o"(这个在前面也看到过)和``PREPROCESS.S"等。这些变量被
展开之后就是对应的命令(包括了命令行选项),例如:变量``COMPILE.c"的定义为
``cc -c"(如果Makefile中存在``CFLAGS"的定义,它的值会存在于这个变量中)。

make会根据默认的约定,使用“COMPILE.x”来编译一个“.x”的文件。类似地使用“LINK.x”
来连接“.x”文件;使用“PREPROCESS.x”对“.x”文件进行预处理。

每一个隐含规则在创建一个文件时都使用了变量“OUTPUT\_OPTION”。make执行命令时根
据命令行参数来决定它的值,当命令行中没有包含“-o”选项时,它的值为:“-o \$@”,
否则为空。建议在规则的命令行中明确使用“-o”选项执行输出文件路径。这是因为在编
译一个多目录的工程时,如果我们的Makefile中使用了“VPATH”指定搜索目录 时,编译
后的.o文件或者其它文件会出现在和源文件不同的目录中。在有些系统的编译器不接受
命令行的“-o”参数,而Makefile中包含“VPAT”的情况时,输出文件可能会出现在错误的
目录下。解决这个问题的方式就是将“OUTPUT\_OPTION”的值赋为“;mv \$*.o \$@”,其功
能是将编译完成的.o文件改变为规则中的目标文件。

\section{隐含变量}
内嵌隐含规则的命令中,所使用的变量都是预定义的变量。我们将这些变量称为“隐含变
量”。这些变量允许对它进行修改:在Makefile中、通过命令行参数或者设置系统环境变
量的方式来对它进行重定义。无论是用那种方式,只要make在运行时它的定义有效,
make的隐含规则都会使用这些变量。当然,也可以使用“-R”或
“--no–builtin-variables”选项来取消所有的隐含变量(同时将取消了所有的隐含规
则)。

例如,编译.c源文件的隐含规则为:“\$(CC) -c \$(CFLAGS) \$(CPPFLAGS)”。默认的编
译命令是“cc”,执行的命令是:“cc –c”。我们可以同上述的任何一种方式将变量“CC”定
义为“ncc”,那么编译.c源文件所执行的命令将是“ncc -c”。同样我们可以对变量
“CFLAGS”进行重定义。对这些变量重定义后如果需要整个工程的各个子目录有效,同样
需要使用关键字“export”将他们导出;否则目录间编译命令可能出现不一致。编译.c源
文件时,隐含规则使用“\$(CC)”来引用编译器;“\$(CFLAGS)”引用编译选项。

隐含规则中所使用的变量(隐含变量)分为两类:1. 代表一个程序的名字(例如:“CC”
代表了编译器这个可执行程序)。2. 代表执行这个程序使用的参数(例如:变量
“CFLAGS”),多个参数使用空格分开。当然也允许在程序的名字中包含参数。但是这种
方式建议不要使用。

隐含规则中所使用的变量(隐含变量)分为两类:1. 代表一个程序的名字(例如:“CC”
代表了编译器这个可执行程序)。2. 代表执行这个程序使用的参数(例如:变量
“CFLAGS”),多个参数使用空格分开。当然也允许在程序的名字中包含参数。但是这种
方式建议不要使用。

以下是一些作为程序名的隐含变量定义:

\subsection{代表命令的变量}
\begin{dinglist}{226}
\itemsep=4pt

\item \textbf{AR}

函数库打包程序,可创建静态库.a文档。默认是“ar”。

\item \textbf{AS}

汇编程序。默认是“as”。

\item \textbf{CC}

C编译程序。默认是“cc”。

\item \textbf{CXX}

C++编译程序。默认是“g++”。

\item \textbf{CO}

从 RCS中提取文件的程序。默认是“co”。

\item \textbf{CPP}

C程序的预处理器(输出是标准输出设备)。默认是“\$(CC) -E”。

\item \textbf{FC}

编译器和预处理Fortran 和 Ratfor 源文件的编译器。默认是“f77”。

\item \textbf{GET}

从SCCS中提取文件程序。默认是“get”。

\item \textbf{LEX}

将 Lex 语言转变为 C 或 Ratfo 的程序。默认是“lex”。

\item \textbf{PC}

Pascal语言编译器。默认是“pc”。

\item \textbf{YACC}

Yacc文法分析器(针对于C程序)。默认命令是“yacc”。

\item \textbf{YACCR}

Yacc文法分析器(针对于Ratfor程序)。默认是“yacc -r”。

\item \textbf{MAKEINFO}

转换Texinfo源文件(.texi)到Info文件程序。默认是“makeinfo”。

\item \textbf{TEX}

从TeX源文件创建TeX DVI文件的程序。默认是“tex”。

\item \textbf{TEXI2DVI}

从Texinfo源文件创建TeX DVI 文件的程序。默认是“texi2dvi”。

\item \textbf{WEAVE}

转换Web到TeX的程序。默认是“weave”。

\item \textbf{CWEAVE}

转换C Web 到 TeX的程序。默认是“cweave”。

\item \textbf{TANGLE}

转换Web到Pascal语言的程序。默认是“tangle”。

\item \textbf{CTANGLE}

转换C Web 到 C。默认是“ctangle”。

\item \textbf{RM}

删除命令。默认是“rm -f”。


\end{dinglist}


\subsection{命令参数的变量}
下边的是代表命令执行参数的变量。如果没有给出默认值则默认值为空。

\begin{dinglist}{226}
\itemsep=4pt

\item \textbf{ARFLAGS}

执行“AR”命令的命令行参数。默认值是“rv”。

\item \textbf{ASFLAGS}

执行汇编语器“AS”的命令行参数(明确指定“.s”或“.S”文件时)。

\item \textbf{CFLAGS}

执行“CC”编译器的命令行参数(编译.c源文件的选项)。

\item \textbf{CXXFLAGS}

执行“g++”编译器的命令行参数(编译.cc源文件的选项)。

\item \textbf{COFLAGS}

执行“co”的命令行参数(在RCS中提取文件的选项)。

\item \textbf{CPPFLAGS}

执行C预处理器“cc -E”的命令行参数(C 和 Fortran 编译器会用到)。

\item \textbf{FFLAGS}

Fortran语言编译器“f77”执行的命令行参数(编译Fortran源文件的选项)。

\item \textbf{GFLAGS}

SCCS “get”程序参数。

\item \textbf{LDFLAGS}

链接器(如:“ld”)参数。

\item \textbf{LFLAGS}

Lex文法分析器参数。

\item \textbf{PFLAGS}

Pascal语言编译器参数。

\item \textbf{RFLAGS}

Ratfor 程序的Fortran 编译器参数。

\item \textbf{YFLAGS}

Yacc文法分析器参数。

\end{dinglist}


\section{make隐含规则链}
有时,一个目标文件需要多个(一系列)隐含规则来创建。例如:创建文件“N.o”的过程
可能是:首先执行“yacc”由“N.y”生成文件“N.c”,之后由编译器将“N.c”编译成为
“N.o”。如果一个目标文件需要一系列隐含规则才能完成它的创建,我们就把这个系列称
为一个“链”。

我们来看上边例子的执行过程。有两种情况:

\vspace{-5pt}
\begin{enumerate}
\itemsep=0pt

  \item 如果文件“N.c”存在或者它在Makefile中被提及,那就不需要进行其它搜索,
      make处理的过程是:首先,make可以确定出“N.o”可由“N.c”创建;之后,make
      试图使用隐含规则来重建“N.c”。它会寻找“N.y”这个文件,如果“N.y”存在,则
      执行隐含规则来重建“N.c”这个文件。之后再由“N.c”重建“N.o”;当不存在
      “N.y”文件时,直接编译“N.c”生成“N.o”。
  \item 文件“N.c”不存在也没有在Makefile中提及的情况,只要存在“N.y”这个文
      件,那么make也会经过这两个步骤来重建“N.o”(N.y → N.c → N.o)。这种情
      况下,文件“N.c”作为一个中间过程文件。Make在执行规则时,如果需要一个中
      间文件才能完成目标的重建,那么这个文件将会自动地加入到依赖关系链中
      (和Makefile中明确提及的目标作相同处理),并使用合适的隐含规则对它进
      行重建。
\end{enumerate}
\vspace{-5pt}

make的中间过程文件和那些明确指定的文件在规则中的地位完全相同。但make在处理时
两者之间存在一些差异:

第一:中间文件不存在时,make处理两者的方式不同。对于一个普通文件来说,因为
Makefile中有明确的提及,此文件可能是作为一个目标的依赖,make在执行它所在的规
则前会试图重建它。但是对于中间文件,因为没有明确提及,make不会去试图重建它。
除非这个中间文件所依赖的文件(上例第二种情况中的文件“N.y”;N.c是中间过程文
件)被更新。

第二:如果make在执行时需要用到一个中间过程文件,那么默认的动作将是:这个过程
文件在make执行结束后会被删除(make会在删除中间过程文件时打印出执行的命令以显
示那些文件被删除了)。因此一个中间过程文件在make执行结束后就不再存在了。

在Makefile中明确提及的所有文件都不被作为中间过程文件来处理,这是缺省地。不过
我们可以在Makefile中使用特殊目标“.INTERMEDIATE”来指除将那些文件作为中间过程文
件来处理(这些文件作为目标“.INTERMEDIATE”的依赖文件罗列),即使它们在Makefile
中被明确提及,这些作为特殊目标“.INTERMEDIATE”依赖的文件在make执行结束之后会被
自动删除。

另一方面,如果我们希望保留某些中间过程文件(它没有在Makefile中被提及),不希
望make结束时自动删除它们。可以在Makefile中使用特使目标“.SECONDARY”来指出这些
文件(这些文件将被作为“secondary”文件;需要保留的文件作为特殊目标“.SECONDARY”
的依赖文件罗列)。注意:“secondary”文件也同时被作为中间过程文件来对待。


需要保留中间过程文件还存在另外一种实现方式。例如需要保留所有.o的中间过程文
件,我们可以将.o文件的模式(\%.o)作为特殊目标“.PRECIOUS”的依赖。

一个“链”可以包含两个以上隐含规则的调用过程。同一个隐含规则在一个“链”中只能出
现一次。否则就会出现像“foo”依赖“foo.o.o”甚至“foo.o.o.o.o…”这样不合逻辑的情况
发生。因为,如果允许在一个“链”中多次调用同一隐含规则(N : N.o; \$(LINK.o)
\$(LDFLAGS) N.o \$(LOADLIBES) \$(LDLIBS) ),将会导致make进入到无限的循环中
去。

隐含规则链中的某些隐含规则,在一些情况会被优化处理。例如:从文件“foo.c”创建可
执行文件“foo”,这一过程可以是:使用隐含规则将“foo.c”编译生成“foo.o”文件,之后
再使用另一个隐含规则来完成对“foo.o”的链接,最后生成执行文件“foo”。这个过程中
对源文件的编译和对.o文件的链接分别使用了两个独立的规则(它们组成一个隐含规则
链)。但是实际情况是,对源文件的编译和对.o文件的链接是在一个规则中完成的,规
则使用命令“cc foo.c foo”。make的隐含规则表中,所有可用的优化规则处于首选地
位。

\section{模式规则}
模式规则类似于普通规则。只是在模式规则中,目标名中需要包含有模式字符“\%”(一
个),包含有模式字符“\%”的目标被用来匹配一个文件名,“\%”可以匹配任何非空字符
串。规则的依赖文件中同样可以使用“\%”,依赖文件中模式字符“\%”的取值情况由目标
中的“\%”来决定。例如:对于模式规则“\%.o : \%.c”,它表示的含义是:所有的.o文件
依赖于对应的.c文件。我们可以使用模式规则来定义隐含规则。

要注意的是:模式字符“\%”的匹配和替换发生在规则中所有变量和函数引用展开之后,
变量和函数的展开一般发生在make读取Makefile时(变量和函数的展开可参考 第五 章
使用变量 和  第七章 make的函数 ),而模式规则中的“\%”的匹配和替换则发生在make
执行时。


\subsection{模式规则介绍}
在模式规则中,目标文件是一个带有模式字符“\%”的文件,使用模式来匹配目标文件。
文件名中的模式字符“\%”可以匹配任何非空字符串,除模式字符以外的部分要求一致。
例如:“\%.c”匹配所有以“.c”结尾的文件(匹配的文件名长度最少为3个字母),
“s\%.c”匹配所有第一个字母为“s”,而且必须以“.c”结尾的文件,文件名长度最小为5个
字符(模式字符“\%”至少匹配一个字符)。在目标文件名中“\%”匹配的部分称为“茎”
(前面已经提到过,参考 静态模式 一节)。使用模式规则时,目标文件匹配之后得到
“茎”,依赖根据“茎”产生对应的依赖文件,这个依赖文件必须是存在的或者可被创建
的。

因此,一个模式规则的格式为:

\begin{Verbatim}[]
%.o : %.c ; COMMAND...
\end{Verbatim}
\noindent 这个模式规则指定了如何由文件“N.c”来创建文件“N.o”,文件“N.c”应该是已
存在的或者可被创建的。

模式规则中依赖文件也可以不包含模式字符“\%”。当依赖文件名中不包含模式字符“\%”
时,其含义是所有符合目标模式的目标文件都依赖于一个指定的文件(例如:\%.o :
debug.h,表示所有的.o文件都依赖于头文件“debug.h”)。这样的模式规则在很多场合
是非常有用的。

同样一个模式规则可以存在多个目标。多目标的模式规则和普通多目标规则有些不同,
普通多目标规则的处理是将每一个目标作为一个独立的规则来处理,所以多个目标就就
对应多个独立的规则(这些规则各自有自己的命令行,各个规则的命令行可能相同)。
但对于多目标模式规则来说,所有规则的目标共同拥有依赖文件和规则的命令行,当文
件符合多个目标模式中的任何一个时,规则定义的命令就有可能将会执行;因为多个目
标共同拥有规则的命令行,因此一次命令执行之后,规则不会再去检查是否需要重建符
合其它模式的目标。看一个例子:
\begin{Verbatim}[]
#sample Makefile

Objects = foo.o bar.o
CFLAGS := -Wall

%x : CFLAGS += -g
%.o : CFLAGS += -O2

%.o %.x : %.c
    $(CC) $(CFLAGS) $< -o $@
\end{Verbatim}

当在命令行中执行“make foo.o foo.x”时,会看到只有一个文件“foo.o”被创建了,同时
make会提示“foo.x”文件是最新的(其实“foo.x”并没有被创建)。此过程表明了多目标
的模式规则在make处理时是被作为一个整体来处理的。这是多目标模式规则和多目标的
普通规则的区别之处。大家不妨将上边的例子改为普通多目标规则试试看将会得到什么
样的结果。

最后需要说明的是:

\begin{enumerate}
\itemsep=0pt \parskip=0pt
  \item 模式规则在Makefile中的顺序需要注意,当一个目标文件同时符合多个目标
      模式时,make将会把第一个目标匹配的模式规则作为重建它的规则。
  \item  Makefile中明确指定的模式规则会覆盖隐含模式规则。就是说如果在
      Makefile中出现了一个对目标文件合适可用的模式规则,那么make就不会再为
      这个目标文件寻找其它隐含规则,而直接使用在Makefile中出现的这个规则。
      在使用时,明确规则永远优先于隐含规则。
  \item 另外,依赖文件存在或者被提及的规则,优先于那些需要使用隐含规则来创
      建其依赖文件的规则。
\end{enumerate}

\subsection{模式规则示例}
本小节来看一些使用模式规则的例子,这些模式规则在GNU make中已经被预定义。首先
看编译.c文件到.o文件的隐含模式规则:

\begin{Verbatim}[]
%.o : %.c
    $(CC) -c $(CFLAGS) $(CPPFLAGS) $< -o $@
\end{Verbatim}
\noindent 此规则描述了一个.o文件如何由对应的.c文件创建。规则的命令行中使用了
自动化变量“\$<”和“\$@”,其中自动化变量“\$<”代表规则的依赖,“\$@”代表规则的目
标。此规则在执行时,命令行中的自动化变量将根据实际的目标和依赖文件取对应值。

make中第二个内嵌模式规则是:
\begin{Verbatim}[]
% :: RCS/%,v
    $(CO) $(COFLAGS) $<
\end{Verbatim}
\noindent
这个规则的含义是:任何一个文件“X”都可以由目录“RCS”下的相应文件“x.v”来生成。规
则的目标为“\%”,它匹配任何文件名,因此只要存在相对应的依赖文件(N.v),目标
(N)都可被创建。双冒号表示该规则是最终规则,意味着规则的依赖文件不是中间过程
文件。

另外,一个具有多目标的隐含规则是:
\begin{Verbatim}[]
%.tab.c %.tab.h: %.y
    bison -d $<
\end{Verbatim}
\noindent 它是一个多目标模式规则,关于多目标的特征可参考 模式规则介绍 一小节
最后一个例子。

\subsection{自动化变量}
模式规则中,规则的目标和依赖文件名代表了一类文件名;规则的命令是对所有这一类
文件重建过程的描述,显然,在命令中不能出现具体的文件名,否则模式规则失去意
义。那么在模式规则的命令行中该如何表示文件,将是本小节的讨论的重点。

假如你需要书写一个将.c文件编译到.o文件的模式规则,那么你该如何为gcc书写正确的
源文件名?当然了,不能使用任何具体的文件名,因为在每一次执行模式规则时源文件
名都是不一样的。为了解决这个问题,就需要使用“自动环变量”,自动化变量的取值是
根据具体所执行的规则来决定的,取决于所执行规则的目标和依赖文件名。

下面对所有的自动化变量进行说明:
\begin{dinglist}{226}
\itemsep=4pt \parskip=0pt

\item \textbf{\$@}

表示规则的目标文件名。如果目标是一个文档文件(Linux中,一般称.a文件为文档文
件,也称为静态库文件),那么它代表这个文档的文件名。在多目标模式规则中,它代
表的是哪个触发规则被执行的目标文件名。

\item \textbf{\$\%}

当规则的目标文件是一个静态库文件时,代表静态库的一个成员名。例如,规则的目标
是“foo.a(bar.o)”,那么,“\$\%”的值就为“bar.o”,“\$@”的值为“foo.a”。如果目标不
是静态库文件,其值为空。

\item \textbf{\$<}

规则的第一个依赖文件名。如果是一个目标文件使用隐含规则来重建,则它代表由隐含
规则加入的第一个依赖文件。

\item \textbf{\$?}

所有比目标文件更新的依赖文件列表,空格分割。如果目标是静态库文件名,代表的是
库成员(.o文件)。

\item \textbf{\$}\verb"^"

规则的所有依赖文件列表,使用空格分隔。如果目标是静态库文件,它所代表的只能是
所有库成员(.o文件)名。一个文件可重复的出现在目标的依赖中,变量“\$\verb"^"”
只记录它的一次引用情况。就是说变量“\$\verb"^"”会去掉重复的依赖文件。

\item \textbf{\$+}

类似“\$\verb"^"”,但是它保留了依赖文件中重复出现的文件。主要用在程序链接时库
的交叉引用场合。

\item \textbf{\$*}

在模式规则和静态模式规则中,代表“茎”。“茎”是目标模式中“\%”所代表的部分(当文
件名中存在目录时,“茎”也包含目录(斜杠之前)部分)。例如:文件“dir/a.foo.b”,
当目标的模式为“a.\%.b”时,“\$*”的值为“dir/a.foo”。“茎”对于构造相关文件名非常
有用。自动化变量“\$*”需要两点说明:

\begin{itemize}
\itemsep=0pt \parskip=2pt
  \item 对于一个明确指定的规则来说不存在“茎”,这种情况下“\$*”的含义发生改
      变。此时,如果目标文件名带有一个可识别的后缀,那么“\$*”表示文件中除后
      缀以外的部分。例如:“foo.c”则“\$*”的值为:“foo”,因为.c是一个可识别的
      文件后缀名。GUN make对明确规则的这种奇怪的处理行为是为了和其它版本的
      make兼容。通常,在除静态规则和模式规则以外,明确指定目标文件的规则中
      应该避免使用这个变量。
  \item 当明确指定文件名的规则中目标文件名包含不可识别的后缀时,此变量为
      空。
\end{itemize}

\end{dinglist}

自动化变量“\$?”在显式规则中也是非常有用的,使用它规则可以指定只对更新以后的依
赖文件进行操作。例如,静态库文件“libN.a”,它由一些.o文件组成。这个规则实现了
只将更新后的.o文件加入到库中:

\begin{Verbatim}[]
lib: foo.o bar.o lose.o win.o
    ar r lib $?
\end{Verbatim}

以上罗列的自动量变量中。其中有四个在规则中代表文件名(\$@、\$<、\$\%、\$*)。
而其它三个的在规则中代表一个文件名列表。GUN make中,还可以通过这七个自动化变
量来获取一个完整文件名中的目录部分和具体文件名部分。在这些变量中加入“D”或者
“F”字符就形成了一系列变种的自动环变量。这些变量会出现在以前版本的make中,在当
前版本的make中,可以使用“dir”或者“notdir”函数来实现同样的功能。


\begin{dinglist}{226}
\itemsep=4pt

\item \textbf{\$(@D)}

表示目标文件的目录部分(不包括斜杠)。如果“\$@”是“dir/foo.o”,那么“\$(@D)”的
值为“dir”。如果“\$@”不存在斜杠,其值就是“.”(当前目录)。注意它和函数“dir”的
区别!

\item \textbf{\$(@F)}

目标文件的完整文件名中除目录以外的部分(实际文件名)。如果“\$@”为
“dir/foo.o”,那么“\$(@F)”只就是“foo.o”。“\$(@F)”等价于函数“\$(notdir \$@)”。


  \item \begin{minipage}[t]{\linewidth}
          \textbf{\$(*D)} \\
          \textbf{\$(*F)}
        \end{minipage}


分别代表目标“茎”中的目录部分和文件名部分。

  \item \begin{minipage}[t]{\linewidth}
          \textbf{\$(\%D)} \\
          \textbf{\$(\%F)}
        \end{minipage}



当以如“archive(member)”形式静态库为目标时,分别表示库文件成员“member”名中的目
录部分和文件名部分。它仅对这种形式的规则目标有效。

  \item \begin{minipage}[t]{\linewidth}
          \textbf{\$(<D)} \\
          \textbf{\$(<F)}
        \end{minipage}


分别表示规则中第一个依赖文件的目录部分和文件名部分。

  \item \begin{minipage}[t]{\linewidth}
          \textbf{\$(}\verb"^"\textbf{D)} \\
          \textbf{\$(}\verb"^"\textbf{F)}
        \end{minipage}

分别表示所有依赖文件的目录部分和文件部分(不存在同一文件)。

  \item \begin{minipage}[t]{\linewidth}
          \textbf{\$(+D)} \\
          \textbf{\$(+F)}
        \end{minipage}

分别表示所有依赖文件的目录部分和文件部分(可存在重复文件)。

  \item \begin{minipage}[t]{\linewidth}
          \textbf{\$(?D)} \\
          \textbf{\$(?F)}
        \end{minipage}

分别表示被更新的依赖文件的目录部分和文件名部分。

\end{dinglist}

在讨论自动化变量时,为了和普通变量(如:“CFLAGS”)区别,我们直接使用了“\$<”的
形式。这种形式仅仅是为了和普通变量进行区别,没有别的目的。其实对于自动环变量
和普通变量一样,代表规则第一个依赖文件名的变量名实际上是“<”,我们完全可以使用
“\$(<)”来替代“\$<”。但是在引用自动化变量时通常的做法是“\$<”,因为自动化变量本
身是一个特殊字符。

GUN make同时支持“Sysv”特性,允许在规则的依赖列表中使用特殊的变量引用(一般的
自动化变量只能在规则的命令行中被引用)“\$\$@”、“\$\$(@D)”和“\$\$(@F)”(注意:
要使用“\$\$”),它们分别代表了“目标的完整文件名”、“目标文件名中的目录部分”和
“目标的实际文件名部分”。这三个特殊的变量只能用在明确指定目标文件名的规则中或
者是静态模式规则中,不用于隐含规则中。另外Sysv make和GNU make对规则依赖的处理
也不尽相同。Sysv make对规则的依赖进行两次替换展开,而GUN make对依赖列表的处理
只有一次,对其中的变量和函数引用直接进行展开。

自动化变量的这个古怪的特性完全是为了兼容Sysv 版本的makefile文件。在使用GNU
make时可以不考虑这个,也可以在Makefile中使用伪目标“.POSIX”来禁止这一特性。

\subsection{模式的匹配}
通常,模式规则中目标模式由前缀、后缀、模式字符“\%”组成,这三个部分允许两个同
时为空。实际文件名应该是以模式指定的前缀开始、后缀结束的任何文件名。文件名中
除前缀和后缀以外的所有部分称之为“茎”(模式字符“\%”可以代表若干字符。因此:模
式“\%.o”所匹配的文件“test.c”中“test”就是“茎”)。模式规则中依赖文件名的确定过
程是:首先根据规则定义的目标模式匹配实际的目标文件,确定“茎”,之后使用 “茎”替
代规则依赖文件名中的模式字符“\%”,生成依赖文件名。这样就产生了一个明确指定了
目标和依赖文件的规则。例如模式规则:“\%.o : \%.c”,当“test.o”需要重建时将形成
规则“test.o : test.c”。

当目标模式中包含斜杠(目录部分)。在进行目标文件匹配时,文件名中包含的目录字
符串在匹配之前被移除,只进行基本文件名的匹配;匹配成功后,再将目录部分加入到
匹配之后的字符串之前形成“茎”。来看一个例子:例如目标模式为“e\%t”,文件
“src/eat”匹配这个模式,那么“茎”就是“src/a”;模式规则中依赖文件的产生:首先使
用“茎”中的非目录部分(“a”)替代依赖文件中的模式字符“\%”,之后再将目录部分(“s
rc/”)加入到形成的依赖文件名之前构成依赖文件的全路径名。这里如果模式规则的依
赖模式为“c\%r”,则那么目标“src/eat”对应的依赖文件就为“src/car”。

\subsection{万用规则}
当模式规则的目标只是一个模式字符“\%”(它可以匹配任何文件名)时,我们称这个规
则为万用规则。万用规则在书写Makefile时非常有用,但它会影响make的执行效率,因
为make在执行时将会使用万用规则来试图重建其它规则的目标和依赖文件。

假如在一个存在万用规则的Makefile中提及了文件“foo.c”。为了创建这个目标,make会
试图使用以下规则来创建这个目标:1.对一个.o文件“foo.c.o”进行链接并产生文件
“foo.c”;2.使用c编译和连接程器由文件“foo.c.c”来创建这个文件;3. 编译并链接
Pascal程序“foo.c.p”来创建;等等。总之make会试图使用所有可能的隐含规则来完成对
这个文件的创建。

当然,我们很清楚以上这样的过程是没有必要的,我们知道“foo.c”是一个.c原文件,而
不是一个可执行程序。make在执行时都会试图根据可能的隐含规则来创建这个文件,但
由于其依赖的文件(“foo.c.o”、“foo.c.c”等)不存在,最终这些可能的隐含规则都会
被否定。但是如果在Makefile中存在一个万用规则,那么make执行时所要考虑的情况就
比较复杂,也很多(它会试图功过隐含规则来创建那些依赖文件,虽然最终这些文件不
可能被创建,也无从创建),从而导致make的执行效率会很低。

为了避免万用规则带来的效率问题,我们可以对万用规则的使用加以限制。通常有两种
方式,需要在定义一个万用规则时对其进行限制。

\begin{enumerate}
\itemsep=4pt \parskip=2pt
  \item 将万用规则设置为最终规则,定义时使用双冒号规则。作为最终规则,此规
      则只有在它的依赖文件存在时才能被应用,即使它的依赖可以由隐含规则创建
      也不行。就是说,最终规则中没有进一步的“链”。

      例如,从RCS和SCCS文件中提取源文件的内嵌隐含规则就是一个最终规则。因此
  如果文件“foo.c,v”不存在,make就不会试图从一个中间文件“foo.c,v.o”或
  “RCS/SCCS/s.foo.c,v”来创建它。 RCS 和 SCCS 文件一般都是最终的源文件,它不
  能由其它任何文件重建;make可以记录它的时间戳,但不会寻找重建它们的方式。

  如果万用规则没有定义为最终规则,那么它就是一个非最终规则。非最终的万用规
  则不会被用来创建那些符合某一个明确模式规则的目标和依赖文件。就是说如果在
  Makefile中存在匹配此文件的模式规则(非万用规则),那么对于这个文件来说其
  重建规则只会是它所匹配的这个模式,而不是这个非最终的万用规则。例如,文件
  “foo.c”,如果在Makefile中同时存在一个万用规则和模式规则 “\%.c : \%.y”(该
  规则运行Yacc)。无论该规则是否会被使用(如果存在文件“foo.y”,那么规则将被
  执行)。那么make试图重建“foo.c”的规则都是“\%.c : \%.y”,而不是万用规则。
  这样做是为了避免make执行时试图使用非最终的万用规则来重建文件“foo.c”的情况
  发生。
  \item 定义一个特殊的内嵌哑模式规则给出如何重建某一类文件,避免使用非最终
      万用规则。哑模式规则没有依赖,也没有命令行,在make的其它场合被忽略。
      例如,内嵌的哑模式规则:“\%.p :”为Pascal源程序如“foo.p”指定了重建规则
      (规则不存在依赖文件、也不存在任何命令),这样就避免了make试图“foo.p”
      而去寻找“foo.p.o”或“foo.p.c”的过程。
\end{enumerate}

我们可以为所有的make可识别的后缀创建一个形如“\%.p :”的哑模式规则。

\subsection{重建内嵌隐含规则}
一个隐含规则,我们可以对它进行重建。重建一个隐含规则时,需要使用相同的目标和
依赖模式,命令可以不同(重新指定规则的命令)。这样就可以替代有相同目标和依赖
的那个make内嵌规则,替代之后隐含规则可能被使用的顺序由它在Makefile中的位置决
定。例如通常Makefile中可能会包含这样一个规则:

\begin{Verbatim}[]
%.o : %.c
    $(CC) $(CFLAGS) –D__DEBUG__  $< -o $@
\end{Verbatim}
\noindent 它替代了编译.c文件的内嵌隐含规则。

也可以取消一个内嵌的隐含规则。同样需要定义一个和隐含规则具有相同目标和依赖的
规则,但这个规则没有命令行。例如下边的这个规则取消了编译.s文件的内嵌规则。
\begin{Verbatim}[]
%.o : %.s
\end{Verbatim}

\section{缺省规则}
有时make会需要一个缺省的规则,在执行过程中无法为一个文件找到合适的重建规则
(在Makefile中没有给出重建它的明确规则,同时也没有合适可用的隐含规则)。那么
make就使用这个规则来重建它。就是说,当所需要的重建的目标文件没有可用的命令
时、就执行这个缺省规则命令。

这样一个规则,我们可以使用最终万用规则。例如:调试Makefile时(可能一些源文件
还没有完成),我们关心的是Makefile中所有的规则是否可正确执行,而源文件的具体
内容却不需要关心。基于这一点我们就可以使用空文件(和源文件同名的文件),在
Makefile中定义这样一个规则:

\begin{Verbatim}[]
%::
    touch $@
\end{Verbatim}
\noindent 执行make过程中,对所有不存在的.c文件将会使用“touch”命令创建这样一个
空的源文件。

实现一个缺省规则的方式也可以不使用万用规则,而使用伪目标“.DEFAULT”。上边的例
子也可以这样实现:
\begin{Verbatim}[]
.DEFAULT :
    touch $@
\end{Verbatim}
\noindent 需要注意:没有指定命令行的伪目标“.DEFAULT”,含义是取消前边所有使用
“.DEFAULT”指定的缺省执行命令。

同样,也可以让这个缺省的规则不执行任何命令(给它定义个一个空命令)。另外缺省
规则也可用来实现在一个Makefile中重载另一个makefile文件。

\section{后缀规则}
后缀规则是一种古老定义隐含规则的方式,在新版本的make中使用模式规则作为对它的
替代,模式规则相比后缀规则更加清晰明了。在现在版本中保留它的原因是为了能够兼
容旧的makefile文件。后缀规则有两种类型:“双后缀”和“单后缀”。

双后缀规则定义一对后缀:目标文件的后缀和依赖目标的后缀。它匹配所有后缀为目标
后缀的文件。对于一个匹配的目标文件,它的依赖文件这样形成:将匹配的目标文件名
中的后缀替换为依赖文件的后缀得到。如:一个描述目标和依赖后缀的“.o”和“.c”的规
则就等价于模式规则“\%o : \%c”。

单后缀规则只定义一个后缀:此后缀是源文件名的后缀。它可以匹配任何文件,其依赖
文件这样形成:将后缀直接追加到目标文件名之后得到。例如:单后缀“.c”就等价于模
式规则“\% : \%.c”。

判断一个后缀规则是单后缀还是双后缀的过程:判断后缀规则的目标,如果其中只存在
一个可被make识别的后缀,则规则是一个“单后缀”规则;当规则目标中存在两个可被
make识别的后缀时,这个规则就是一个“双后缀”规则。

例如:“.c”和“.o”都是make可识别的后缀。因此当定义了一个目标是“.c.o”的规则时。
make会将它作为一个双后缀规则来处理,它的含义是所有“.o”文件的依赖文件是对应的
“.c”文件。下边是使用后追规则定义的编译.c源文件的规则:

\begin{Verbatim}[]
.c.o:
    $(CC) -c $(CFLAGS) $(CPPFLAGS) -o $@ $<
\end{Verbatim}

注意:一个后缀规则中不存在任何依赖文件。否则,此规则将被作为一个普通规则对
待。因此规则:
\begin{Verbatim}[]
.c.o: foo.h
    $(CC) -c $(CFLAGS) $(CPPFLAGS) -o $@ $<
\end{Verbatim}
\noindent 就不是一个后缀规则。它是一个目标文件为“.c.o”、依赖文件是“foo.h”的普
通规则。它也不等价于规则:
\begin{Verbatim}[]
%.o: %.c foo.h
    $(CC) -c $(CFLAGS) $(CPPFLAGS) -o $@ $<
\end{Verbatim}

需要注意的是:没有命令行的后缀规则是没有任何意义的。它和没有命令行的模式规则
不同,它也不能取消之前使用后追规则定义的规则。它所实现的仅仅是将这个后缀规则
作为目标加入到make的数据库中。

可识别的后缀指的是特殊目标“.SUFFIXES”所有依赖的名字。通过给特殊目标“SUFFIXES”
添加依赖来增加一个可被识别的后缀。像下边这样:
\begin{Verbatim}[]
.SUFFIXES: .hack .win
\end{Verbatim}
\noindent 它所实现的功能是把后缀“.hack”和“.win”加入到可识别后缀列表的末尾。

如果需要重设默认的可识别后缀,因该这样来实现:
\begin{Verbatim}[]
.SUFFIXES:            #删除所有已定义的可识别后缀
.SUFFIXES: .c .o .h   #重新定义
\end{Verbatim}
\noindent 首先使用没有依赖的特殊目标“.SUFFIXES”来删除所有已定义的可识别后缀;
之后再重新定义。

注意:make的“-r”或“-no-builtin-rules”可以清空所有已定义的可识别后缀。

在make读取所有的makefile文件之前,变量“SUFFIXE”被定义为默认的可识别后缀列表。
虽然存在这样一个变量,但是请不要通过修改这个变量值的方式来改变可识别的后缀列
表,应该使用特殊目标“.SUFFIXES”来实现。

\section{隐含规则搜索算法}
对于目标“T”,make为它搜索隐含规则的算法如下。此算法适合于:1. 任何没有命令行
的双冒号规则;2. 任何没有命令行的普通规则;3. 那些不是任何规则的目标、但它是
另外某些规则的依赖文件;4. 在递归搜索过程中,隐含规则链中前一个规则的依赖文
件。

在搜索过程中没有提到后缀规则,因为所有的后缀规则在make读取Makefile时,都被转
换为对应的模式规则。

对于形式为“ARCHIVE(MEMBER)”的目标,下边的算法会执行两次,第一次的目标是整个目
标名“T”(“ARCHIVE(MEMBER)”),如果搜索失败,进行第二次搜索,第二次以“member”
作为目标来搜索。

搜索过程如下:

\begin{enumerate}
\itemsep=4pt \parskip=0pt
  \item 将目标“T”的目录部分分离,分离后目录部分称为“D”,其它部分称“N”。例
      如:“T”为“src/foo.o”时,D就是“src/”,“N”就为“foo.o”。
  \item 列出所有和“T”或者“N”匹配的模式规则。如果模式规则的目标中包含斜杠,
      则认为和“T”相匹配,否则认为此模式规则和“N”相匹配。
  \item 只要这个模式规则列表中包含一个非万用规则的规则,那么将列表中所有的
      非最终万用规则删除。
  \item 删除这个模式规则列表中的所有没有命令行的规则。
  \item 对于这个模式规则列表中的所有规则:

\begin{enumerate}
  \item 计算出模式规则的“茎”S,S应该是“T”或“N”中匹配“\%”的非空的部分。
  \item 计算依赖文件。把依赖中的“\%”用“S”替换。如果目标模式中不包含斜杠,
      则把“D”加在替换之后的每一个依赖文件开头,构成完整的依赖文件名。
  \item 测试规则的所有依赖文件是否存在或是应该存在(一个文件,如果在
      Makefile中它作为一个明确规则的目标,或者依赖文件被提及,我们就说这
      个文件是一个“应该存在”的文件)。如果所有的依赖文件都存在、应该存在
      或是这个规则没有依赖。退出查找,使用该规则。
\end{enumerate}

  \item 截止到第5步,合适的规则还是没有被找到,进一步搜索。对于这个模式规则
      列表中的每一规则:

\begin{enumerate}
  \item 如果规则是一个终止规则,则忽略它,继续下一条模式规则。
  \item 计算依赖文件(同第5步)。
  \item 测试此规则的依赖文件是否存在或是应该存在。
  \item 对于此规则中不存在的依赖文件,递归的调用这个算法查找它是否可由隐
      含规则来创建。
  \item 如果所有的依赖文件存在、应该存在、或是它可以由一个隐含规则来创
      建。退出查找,使用该规则。
\end{enumerate}


  \item 如果没有隐含规则可以创建这个规则的依赖文件,则执行特殊目标
      “.DEFAULT”所指定的命令(可以创建这个文件,或者给出一个错误提示)。如
      果在Makefile中没有定义特殊目标“DEFAULT”,就没有可用的命令来完成“T”的
      创建。make退出。
\end{enumerate}


一旦为一类目标查找到合适的模式规则。除匹配“T”或者“N”的模式以外,对其它模式规
则中的目标模式进行配置,使用“茎”S替换其中的模式字符“\%”,将得到的文件名保存直
到执行命令更新这个目标文件(“T”)。在命令执行以后,把每一个储存的文件名放入数
据库,并且标志为已更新,其时间戳和文件“T”相同。

在执行更新文件“T”的命令时,使用自动化变量表示规则中的依赖文件和目标文件。
