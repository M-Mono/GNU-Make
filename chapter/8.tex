\chapter{make的内嵌函数}

GNU make的函数提供了处理文件名、变量、文本和命令的方法。使用函数我们的
Makefile可以书写的更加灵活和健壮。可以在需要的地方地调用函数来处理指定的文本
(需要处理的文本作为函数的参数),函数的在调用它的地方被替换为它的处理结果。
函数调用(引用)的展开和变量引用的展开方式相同。

\section{函数的调用语法}
GNU make函数的调用格式类似于变量的引用,以“\$”开始表示一个引用。语法格式如
下:

\begin{Verbatim}[]
$(FUNCTION ARGUMENTS)
\end{Verbatim}
\noindent 或者:
\begin{Verbatim}[]
${FUNCTION ARGUMENTS}
\end{Verbatim}

于函数调用的格式有以下几点说明:

\textbf{1.} 调用语法格式中“FUNCTION”是需要调用的函数名,它应该是make内嵌的函
数名。对于用户自己的函数需要通过make的“call”函数来间接调用。

\textbf{2.} “ARGUMENTS”是函数的参数,参数和函数名之间使用若干个空格或者[tab]
字符分割(建议使用一个空格,这样不仅使在书写上比较直观,更重要的是当你不能确
定是否可以使用[Tab]的时候,避免不必要的麻烦);如果存在多个参数时,参数之间使
用逗号“,”分开。

\textbf{3.} 以“\$”开头,使用成对的圆括号或花括号把函数名和参数括起(在
Makefile中,圆括号和花括号在任何地方必须成对出现)。参数中存在变量或者函数的
引用时,对它们所使用的分界符(圆括号或者花括号)建议和引用函数的相同,不使用
两种不同的括号。推荐在变量引用和函数引用中统一使用圆括号;这样在使用“vim”编辑
器书写Makefile时,使用圆括它可以亮度显式make的内嵌函数名,避免函数名的拼写错
误。在Makefile中应该这样来书写“\$(sort \$(x))”;而不是“\$(sort \$\{x\})”和其
它几种。

\textbf{4.} 函数处理参数时,参数中如果存在对其它变量或者函数的引用,首先对这
些引用进行展开得到参数的实际内容。而后才对它们进行处理。参数的展开顺序是按照
参数的先后顺序来进行的。

\textbf{5.} 书写时,函数的参数不能出现逗号“,”和空格。这是因为逗号被作为多个参
数的分隔符,前导空格会被忽略。在实际书写Makefile时,当有逗号或者空格作为函数
的参数时,需要把它们赋值给一个变量,在函数的参数中引用这个变量来实现。我们来
看一个这样的例子:

\begin{Verbatim}[]
comma:= ,
empty:=
space:= $(empty) $(empty)
foo:= a b c
bar:= $(subst $(space),$(comma),$(foo))
\end{Verbatim}

这样我们就实现了“bar”的值是“a,b,c”。

\section{文本处理函数}
以下是GNU make内嵌的文本(字符串)处理函数。

\subsection{\$(subst FROM,TO,TEXT)}
函数名称:字符串替换函数—subst。

函数功能:把字串“TEXT”中的“FROM”字符替换为“TO”。

返回值:替换后的新字符串。

示例:
\begin{Verbatim}[]
$(subst ee,EE,feet on the street)
\end{Verbatim}

替换“feet on the street”中的“ee”为“EE”,结果得到字符串“fEEt on the strEEt”。

\subsection{\$(patsubst PATTERN,REPLACEMENT,TEXT)}
函数名称:模式替换函数—patsubst。

函数功能:搜索“TEXT”中以空格分开的单词,将否符合模式“TATTERN”替换为
“REPLACEMENT”。参数“PATTERN”中可以使用模式通配符“\%”来代表一个单词中的若干字
符。如果参数“REPLACEMENT”中也包含一个“\%”,那么“REPLACEMENT”中的“\%”将是
“TATTERN”中的那个“\%”所代表的字符串。在“TATTERN”和“REPLACEMENT”中,只有第一个
“\%”被作为模式字符来处理,之后出现的不再作模式字符(作为一个字符)。在参数中
如果需要将第一个出现的“\%”作为字符本身而不作为模式字符时,可使用反斜杠
“\verb"\"”进行转义处理(转义处理的机制和使用静态模式的转义一致,具体可参考
5.12.1 静态模式规则的语法 一小节)。

返回值:替换后的新字符串。

函数说明:参数“TEXT”单词之间的多个空格在处理时被合并为一个空格,并忽略前导和结尾空格。

示例:
\begin{Verbatim}[]
$(patsubst %.c,%.o,x.c.c bar.c)
\end{Verbatim}

把字串“x.c.c bar.c”中以.c结尾的单词替换成以.o结尾的字符。函数的返回结果是
“x.c.o bar.o”。

本文的第六章在 变量的高级用法的第一小节 中曾经讨论过变量的替换引用,它是一个
简化版的“patsubst”函数在变量引用过程的实现。变量替换引用中:
\begin{Verbatim}[]
$(VAR:PATTERN=REPLACEMENT)
\end{Verbatim}
就相当于:
\begin{Verbatim}[]
$(patsubst PATTERN,REPLACEMENT,$(VAR))
\end{Verbatim}

而另外一种更为简单的替换字符后缀的实现:
\begin{Verbatim}[]
$(VAR:SUFFIX=REPLACEMENT)
\end{Verbatim}
它等于:
\begin{Verbatim}[]
$(patsubst %SUFFIX,%REPLACEMENT,$(VAR))
\end{Verbatim}

例如我们存在一个代表所有.o文件的变量。定义为“objects = foo.o bar.o baz.o”。为
了得到这些.o文件所对应的.c源文件。我们可以使用以下两种方式的任意一个:
\begin{Verbatim}[]
$(objects:.o=.c)
$(patsubst %.o,%.c,$(objects))
\end{Verbatim}

\subsection{\$(strip STRINT)}

函数名称:去空格函数—strip。

函数功能:去掉字串(若干单词,使用若干空字符分割)“STRINT”开头和结尾的空字符,并将其中多个连续空字符合并为一个空字符。

返回值:无前导和结尾空字符、使用单一空格分割的多单词字符串。函数说明:空字符

包括空格、[Tab]等不可显示字符。

示例:

\begin{Verbatim}[]
STR =        a    b c
LOSTR = $(strip $(STR))
\end{Verbatim}

结果是“a b c”。 “strip”函数经常用在条件判断语句的表达式中,确保表达式比较的可
靠和健壮!

\subsection{\$(findstring FIND,IN)}
函数名称:查找字符串函数—findstring。

函数功能:搜索字串“IN”,查找“FIND”字串。

返回值:如果在“IN”之中存在“FIND”,则返回“FIND”,否则返回空。

函数说明:字串“IN”之中可以包含空格、[Tab]。搜索需要是严格的文本匹配。

示例:
\begin{Verbatim}[]
$(findstring a,a b c)
$(findstring a,b c)
\end{Verbatim}
第一个函数结果是字“a”;第二个值为空字符。

\subsection{\$(filter PATTERN…,TEXT)}
函数名称:过滤函数—filter。

函数功能:过滤掉字串“TEXT”中所有不符合模式“PATTERN”的单词,保留所有符合此模式的单词。可以使用多个模式。模式中一般需要包含模式字符“%”。存在多个模式时,模式表达式之间使用空格分割。

返回值:空格分割的“TEXT”字串中所有符合模式“PATTERN”的字串。函数说明:“filter”

函数可以用来去除一个变量中的某些字符串,我们下边的例子中就是用到了此函数。

示例:
\begin{Verbatim}[]
sources := foo.c bar.c baz.s ugh.h foo: $(sources)
cc $(filter %.c %.s,$(sources)) -o foo
\end{Verbatim}
使用“\verb"$(filter %.c %.s,$(sources))"”的返回值给cc来编译生成目标“foo”,函
数返回值为“foo.c bar.c baz.s”。

\subsection{\$(filter-out PATTERN...,TEXT)}

函数名称:反过滤函数—filter-out。

函数功能:和“filter”函数实现的功能相反。过滤掉字串“TEXT”中所有符合模式
“PATTERN”的单词,保留所有不符合此模式的单词。可以有多个模式。存在多个模式时,
模式表达式之间使用空格分割。

返回值:空格分割的“TEXT”字串中所有不符合模式“PATTERN”的字串。

函数说明:“filter-out”函数也可以用来去除一个变量中的某些字符串,(实现和
“filter”函数相反)。

示例:
\begin{Verbatim}[]
objects=main1.o foo.o main2.o bar.o
mains=main1.o main2.o

$(filter-out $(mains),$(objects))
\end{Verbatim}

实现了去除变量“objects”中“mains”定义的字串(文件名)功能。它的返回值为“foo.o
bar.o”。

\subsection{\$(sort LIST)}

函数名称:排序函数—sort。

函数功能:给字串“LIST”中的单词以首字母为准进行排序
(升序),并取掉重复的单词。

返回值:空格分割的没有重复单词的字串。函数说明:
两个功能,排序和去字串中的重复单词。可以单独使用其中一个功能。

示例:
\begin{Verbatim}[]
$(sort foo bar lose foo)
\end{Verbatim}

返回值为:“bar foo lose” 。

\subsection{\$(word N,TEXT)}

函数名称:取单词函数—word。

函数功能:取字串“TEXT”中第“N”个单词(“N”的值从1开始)。

返回值:返回字串“TEXT”中第“N”个单词。

函数说明:如果“N”值大于字串“TEXT”中单词的数目,返回空字符串。如果“N”为0,出错!

示例:

\begin{Verbatim}[]
$(word 2, foo bar baz)
\end{Verbatim}
返回值为“bar”。

\subsection{\$(wordlist S,E,TEXT)}

函数名称:取字串函数—wordlist。

函数功能:从字串“TEXT”中取出从“S”开始到“E”的单词串。“S”和“E”表示单词在字串中位置的数字。

返回值:字串“TEXT”中从第“S”到“E”(包括“E”)的单词字串。
函数说明:“S”和“E”都是从1开始的数字。
当“S”比“TEXT”中的字数大时,返回空。如果“E”大于“TEXT”字数,返回从“S”开始,到“TEXT”结束的单词串。如果“S”大于“E”,返回空。

示例:

\begin{Verbatim}[]
$(wordlist 2, 3, foo bar baz)
\end{Verbatim}

返回值为:“bar baz”。


\subsection{\$(words TEXT)}

函数名称:统计单词数目函数—words。

函数功能:字算字串“TEXT”中单词的数目。

返回值:“TEXT”字串中的单词数。

示例:


\begin{Verbatim}[]
$(words, foo bar)
\end{Verbatim}

返回值是“2”。所以字串“TEXT”的最后一个单词就是:\$(word \$(words TEXT),TEXT)。

\subsection{\$(firstword NAMES…)}

函数名称:取首单词函数—firstword。

函数功能:取字串“NAMES…”中的第一个单词。

返回值:字串“NAMES…”的第一个单词。

函数说明:“NAMES”被认为是使用空格分割的多个单词(名字)的序列。函数忽略“NAMES…”中除第一个单词以外的所有的单词。

示例:
\begin{Verbatim}[]
$(firstword foo bar)
\end{Verbatim}

返回值为“foo”。函数“firstword”实现的功能等效于“\$(word 1, NAMES…)”。

以上11个函数是make内嵌的的文本处理函数。书写Makefile时可搭配使用来实现复杂功
能。最后我们使用这些函数来实现一个实际应用。例子中我们使用函数“subst”和
“patsbust”。Makefile中可以使用变量“VPATH”来指定搜索路径。对于源代码所包含的头
文件的搜索路径需要使用gcc的“-I”参数指定目录来实现,“VPATH”罗列的目录是用冒号
“:”分割的。如下就是Makefile的片段:

\begin{Verbatim}[]
……
VPATH = src:../includes
override CFLAGS += $(patsubst %,-I%,$(subst :, ,$(VPATH)))
……
\end{Verbatim}

那么第二条语句所实现的功能就是“CFLAGS += -Isrc –I../includes”。

\section{文件名处理函数}
GNU make除支持上一节所介绍的文本处理函数之外,还支持一些针对于文件名的处理函
数。这些函数主要用来对一系列空格分割的文件名进行转换,这些函数的参数被作为若
干个文件名来对待。函数对作为参数的一组文件名按照一定方式进行处理并返回空格分
割的多个文件名序列。


\subsection{\$(dir NAMES…)}

函数名称:取目录函数—dir。

函数功能:从文件名序列“NAMES…”中取出各个文件名的目录部分。文件名的目录部分就是包含在文件名中的最后一个斜线(“/”)(包括斜线)之前的部分。

返回值:空格分割的文件名序列“NAMES…”中每一个文件的目录部分。

函数说明:如果文件名中没有斜线,认为此文件为当前目录(“./”)下的文件。

示例:
\begin{Verbatim}[]
$(dir src/foo.c hacks)
\end{Verbatim}

返回值为“src/ ./”。

\subsection{\$(notdir NAMES…)}

函数名称:取文件名函数——notdir。

函数功能:从文件名序列“NAMES…”中取出非目录部分。目录部分是指最后一个斜线(“/”)(包括斜线)之前的部分。删除所有文件名中的目录部分,只保留非目录部分。

返回值:文件名序列“NAMES…”中每一个文件的非目录部分。

函数说明:如果“NAMES…”中存在不包含斜线的文件名,则不改变这个文件名。以反斜线结尾的文件名,是用空串代替,因此当“NAMES…”中存在多个这样的文件名时,返回结果中分割各个文件名的空格数目将不确定!这是此函数的一个缺陷。

示例:
\begin{Verbatim}[]
$(notdir src/foo.c hacks)
\end{Verbatim}

返回值为:“foo.c hacks”。

\subsection{\$(suffix NAMES…)}

函数名称:取后缀函数—suffix。

函数功能:从文件名序列“NAMES…”中取出各个文件名的后缀。后缀是文件名中最后一个以点“.”开始的(包含点号)部分,如果文件名中不包含一个点号,则为空。

返回值:以空格分割的文件名序列“NAMES…”中每一个文件的后缀序列。

函数说明:“NAMES…”是多个文件名时,返回值是多个以空格分割的单词序列。如果文件名没有后缀部分,则返回空。

示例:
\begin{Verbatim}[]
$(suffix src/foo.c src-1.0/bar.c hacks)
\end{Verbatim}

返回值为“.c .c”。

\subsection{\$(basename NAMES…)}

函数名称:取前缀函数—basename。

函数功能:从文件名序列“NAMES…”中取出各个文件名的前缀部分(点号之后的部分)。前缀部分指的是文件名中最后一个点号之前的部分。

返回值:空格分割的文件名序列“NAMES…”中各个文件的前缀序列。如果文件没有前缀,则返回空字串。

函数说明:如果“NAMES…”中包含没有后缀的文件名,此文件名不改变。如果一个文件名中存在多个点号,则返回值为此文件名的最后一个点号之前的文件名部分。

示例:
\begin{Verbatim}[]
$(basename src/foo.c src-1.0/bar.c /home/jack/.font.cache-1 hacks)
\end{Verbatim}

返回值为:“src/foo src-1.0/bar /home/jack/.font hacks”。

\subsection{\$(addsuffix SUFFIX,NAMES…)}

函数名称:加后缀函数—addsuffix。

函数功能:为“NAMES…”中的每一个文件名添加后缀“SUFFIX”。参数“NAMES…”为空格分割的文件名序列,将“SUFFIX”追加到此序列的每一个文件名的末尾。

返回值:以单空格分割的添加了后缀“SUFFIX”的文件名序列。

函数说明:

示例:
\begin{Verbatim}[]
$(addsuffix .c,foo bar)
\end{Verbatim}

返回值为“foo.c bar.c”。

\subsection{\$(addprefix PREFIX,NAMES…)}

函数名称:加前缀函数—addprefix。

函数功能:为“NAMES…”中的每一个文件名添加前缀“PREFIX”。参数“NAMES…”是空格分割的文件名序列,将“SUFFIX”添加到此序列的每一个文件名之前。

返回值:以单空格分割的添加了前缀“PREFIX”的文件名序列。

函数说明:

示例:

\begin{Verbatim}[]
$(addprefix src/,foo bar)
\end{Verbatim}

返回值为“src/foo src/bar”。

\subsection{\$(join LIST1,LIST2)}

函数名称:单词连接函数——join。

函数功能:将字串“LIST1”和字串“LIST2”各单词进行对应连接。就是将“LIST2”中的第一个单词追加“LIST1”第一个单词字后合并为一个单词;将“LIST2”中的第二个单词追加到“LIST1”的第一个单词之后并合并为一个单词,……依次列推。

返回值:单空格分割的合并后的字(文件名)序列。

函数说明:如果“LIST1”和“LIST2”中的字数目不一致时,两者中多余部分将被作为返回序列的一部分。

示例1:
\begin{Verbatim}[]
$(join a b , .c .o)
\end{Verbatim}

返回值为:“a.c b.o”。

示例2:
\begin{Verbatim}[]
$(join a b c , .c .o)
\end{Verbatim}

返回值为:“a.c b.o c”。

\subsection{\$(wildcard PATTERN)}

函数名称:获取匹配模式文件名函数—wildcard

函数功能:列出当前目录下所有符合模式“PATTERN”格式的文件名。

返回值:空格分割的、存在当前目录下的所有符合模式“PATTERN”的文件名。

函数说明:“PATTERN”使用shell可识别的通配符,包括“?”(单字符)、“*”(多字符)等。

示例:
\begin{Verbatim}[]
$(wildcard *.c)
\end{Verbatim}

返回值为当前目录下所有.c源文件列表。

\section{foreach 函数}
函数“foreach”不同于其它函数。它是一个循环函数。类似于Linux的shell中的for语句。

\begin{dinglist}{226}
\itemsep=4pt \parskip=0pt

\item “foreach”函数的\textbf{语法:}

\begin{Verbatim}[]
$(foreach VAR,LIST,TEXT)
\end{Verbatim}


\item \textbf{函数功能:}这个函数的工作过程是这样的:如果需要(存在变量或者函
    数的引
    用),首先展开变量“VAR”和“LIST”的引用;而表达式“TEXT”中的变量引用不展开。
    执行时把“LIST”中使用空格分割的单词依次取出赋值给变量“VAR”,然后执行“TEXT”
    表达式。重复直到“LIST”的最后一个单词(为空时结束)。“TEXT”中的变量或者函
    数引用在执行时才被展开,因此如果在“TEXT”中存在对“VAR”的引用,那么“VAR”的
    值在每一次展开式将会到的不同的值。

\item \textbf{返回值:}空格分割的多次表达式“TEXT”的计算的结果。

\end{dinglist}

我们来看一个例子,定义变量“files”,它的值为四个目录(变量“dirs”代表的a、b、c、d四个目录)下的文件列表:

\begin{Verbatim}[]
dirs := a b c d
files := $(foreach dir,$(dirs),$(wildcard $(dir)/*))
\end{Verbatim}

例子中,“TEXT”的表达式为“\verb"$(wildcard $(dir)/*)"”。表达式第一次执行时将
展开为“\verb"$(wildcard a/*)"”;第二次执行时将展开为
“\verb"$(wildcard b/*)"”;第三次展开为“\verb"$(wildcard c/*)";以此类推。所以此函数所实现的功能
就和一下语句等价:

\begin{Verbatim}[]
files := $(wildcard a/* b/* c/* d/*)
\end{Verbatim}

当函数的“TEXT”表达式过于复杂时,我们可以通过定义一个中间变量,此变量代表表达式的一部分。并在函数的“TEXT”中引用这个变量。上边的例子也可以这样来实现:

\begin{Verbatim}[]
find_files = $(wildcard $(dir)/*)
dirs := a b c d
files := $(foreach dir,$(dirs),$(find_files))
\end{Verbatim}

在这里我们定义了一个变量(也可以称之为表达式),需要注意,在这里定义的是“递归
展开”时的变量“find\_files”。保证了定义时变量值中的引用不展开,而是在表达式被
函数处理时才展开(如果这里使用直接展开式的定义将是无效的表达式)。

\begin{dinglist}{226}
\itemsep=4pt \parskip=0pt

\item \textbf{函数说明:}函数中参数“VAR”是一个局部的临时变量,它只在“foreach”
    函数的上
    下文中有效,它的定义不会影响其它部分定义的同名“VAR”变量的值。在函数的执行
    过程中它是一个“直接展开”式变量。

\end{dinglist}


\begin{quote}\kaishu
\textbf{注意:}在使用函数“foreach”时,变量“VAR”的名字。我们建议使用一个单
词、最好能够表达其含义的名字,不要使用一个奇怪的字符串作为变量名。虽然执行
是不会发生错误,但是会让人很费解。没有人会喜欢这种方式,尽管可能它可以正常
工作:

\verb"files := $(foreach Esta escrito en espanol!,b c ch,$(find_files))"
\end{quote}

\section{if 函数}
函数“if”提供了一个在函数上下文中实现条件判断的功能。就像make所支持的条件语句
—ifeq一样。

\begin{dinglist}{226}
\itemsep=4pt \parskip=0pt

\item \textbf{函数语法:}

\begin{Verbatim}[]
$(if CONDITION,THEN-PART[,ELSE-PART])
\end{Verbatim}

\item \textbf{函数功能:}第一个参数“CONDITION”,在函数执行时忽略其前导和结尾
    空字符,如果包含对其他变量或者函数的引用则进行展开。如果“CONDITION”的展开
    结果非空,则条件为真,就将第二个参数“THEN\_PATR”作为函数的计算表达式;
    “CONDITION”的展开结果为空,将第三个参数“ELSE-PART”作为函数的表达式,函数
    的返回结果为有效表达式的计算结果。


\item \textbf{返回值:}根据条件决定函数的返回值是第一个或者第二个参数表达式的
    计算结果。当不存在第三个参数“ELSE-PART”,并且“CONDITION”展开为空,函数返
    回空。

\item \textbf{函数说明:}函数的条件表达式“CONDITION”决定了函数的返回值只能是
    “THEN-PART”或者“ELSE-PART”两个之一的计算结果。

\item \textbf{函数示例:}
\begin{Verbatim}[]
SUBDIR += $(if $(SRC_DIR) $(SRC_DIR),/home/src)
\end{Verbatim}

函数的结果是:如果“SRC\_DIR”变量值不为空,则将变量“SRC\_DIR”指定的目录作为一
个子目录;否则将目录“/home/src”作为一个子目录。
\end{dinglist}

\section{call函数}
“call”函数是唯一一个可以创建定制化参数函数的引用函数。使用这个函数可以实现对
用户自己定义函数引用。我们可以将一个变量定义为一个复杂的表达式,用“call”函数
根据不同的参数对它进行展开来获得不同的结果。

\begin{dinglist}{226}
\itemsep=4pt \parskip=0pt

\item \textbf{函数语法:}

\begin{Verbatim}[]
$(call VARIABLE,PARAM,PARAM,...)
\end{Verbatim}

\item \textbf{函数功能:}在执行时,将它的参数“PARAM”依次赋值给临时变量
    “\$(1)”、“\$(2)”(这些临时变量定义在“VARIABLE”的值中,参考下边的例子)……
    call函数对参数的数目没有限制,也可以没有参数值,没有参数值的“call”没有任
    何实际存在的意义。执行时变量“VARIABLE”被展开为在函数上下文有效的临时变
    量,变量定义中的“\$(1)”作为第一个参数,并将函数参数值中的第一个参数赋值给
    它;变量中的“\$(2)”一样被赋值为函数的第二个参数值;依此类推(变量\$(0)代表
    变量“VARIABLE”本身)。之后对变量“VARIABLE” 表达式的计算值。


\item \textbf{返回值:}参数值“PARAM”依次替换“\$(1)”、“\$(2)”…… 之后变量
    “VARIABLE”定义的表达式的计算值。

\item \textbf{函数说明:}1. 函数中“VARIBLE”是一个变量名,而不是变量引用。因
    此,通常“call”函数中的“VARIABLE”中不包含“\$”(当然,除非此变量名是一个计算
    的变量名)。2. 当变量“VARIBLE”是一个make内嵌的函数名时(如
    “if”、“foreach”、“strip”等),对“PARAM”参数的使用需要注意,因为不合适或者
    不正确的参数将会导致函数的返回值难以预料。3. 函数中多个“PARAM”之间使用逗
    号分割。4. 变量“VARIABLE”在定义时不能定义为直接展开式!只能定义为递归展开
    式。

\item \textbf{函数示例:}首先,来看一个简单的例子。
\end{dinglist}
\textbf{示例1:}

\begin{Verbatim}[]
reverse =  $(2) $(1)
foo = $(call reverse,a,b)
\end{Verbatim}

变量“foo”的值为“ba”。这里变量“reverse”中的参数定义顺序可以根据需要来调整,并
不是需要按照“\$(1)”、“\$(2)”、“\$(3)”…… 这样的顺序来定义。

看一个稍微复杂一些的例子。我们定义了一个宏“pathsearch”来在“PATH”路径中搜索第
一个指定的程序。

\textbf{示例2:}

\begin{Verbatim}[]
pathsearch = $(firstword $(wildcard $(addsuffix /$(1),$(subst :, ,$(PATH)))))
LS := $(call pathsearch,ls)
\end{Verbatim}

变量“LS”的结果为“/bin/sh”。执行过程:函数“subst”将环境变量“PATH”转换为空格分
割的搜索路径列表;“addsuffix”构造出可能的可执行程序“\$(1)”(这里是“ls”)带路
径的完整文件名(如:“/bin/\$(1)”),之后使用函数“wildcard”匹配,最后
“firstword”函数取第一个文件名。

函数“call”以可以套嵌使用。每一层“call”函数的调用都为它自己的局部变量“\$(1)”等
赋值,覆盖上一层函数为它所赋的值。

\textbf{示例3:}
\begin{Verbatim}[]
map = $(foreach a,$(2),$(call $(1),$(a)))
o = $(call map,origin,o map MAKE)
\end{Verbatim}

那么变量“o”的值就为“file file default”。我们这里使用了“origin”函数。我们分析
函数的执行过程:首先,“o=\$(call map,origin, o map MAKE)”这个函数调用使用了变
量“map”所定义的表达式;使用内嵌函数名“origin”作为它的第一个参数值,使用
Makefile中的变量“o map MAKE”作为他的第二个参数值。当使用“call”函数展开后等价
于“\$(foreach a,o map MAKE,\$(origin \$(a)))”。


\begin{dinglist}{226}
\itemsep=4pt \parskip=0pt

\item \textbf{注意:}和其它函数一样,“call”函数会保留出现在其参数值列表中的空
    字符。因
    此在使用参数值时对空格处理要格外小心。如果参数中存在多余的空格,函数可能
    会返回一个莫名奇妙的值。为了安全,在变量作为“call”函数参数值之前,应去掉
    其值中的多余空格。

\end{dinglist}

\section{value函数}
函数“value”提供了一种在不对变量进行展开的情况下获取变量值的方法。注意:并不是
说函数会取消之前已经执行过的替换扩展。比如:定义了一个直接展开式的变量,此变
量在定义过程中对其它变量的引用进行替换而得到自身的值。在使用“value”函数取这个
变量进行取值时,得到的是不包含任何引用值。而不是将定义过程中的替换展开动作取
消后包含引用的定义值。就是说此过程不能取消此变量在定义时已经发生了的替换展开
动作。

\begin{dinglist}{226}
\itemsep=4pt \parskip=0pt

\item \textbf{函数语法:}

\begin{Verbatim}[]
$(value VARIABLE)
\end{Verbatim}

\item \textbf{函数功能:}不对变量“VARIBLE”进行任何展开操作,直接返回变量
    “VARIBALE”的值。这里“VARIABLE”是一个变量名,一般不包含“\$”(除非计算的变量
    名)。


\item \textbf{返回值:}变量“VARIBALE”所定义文本值(如果变量定义为递归展开式,
    其中包含对其他变量或者函数的引用,那么函数不对这些引用进行展开。函数的返
    回值是包含有引用值)。

\item \textbf{函数示例:}
\begin{Verbatim}[]
# sample Makefile
FOO = $PATH

all:
    @echo $(FOO)
    @echo $(value FOO)
\end{Verbatim}

执行make,可以看到的结果是:第一行为:“ATH”。这是因为变量“FOO”定义为
“\$PATH”,所以展开为“ATH”(“\$P”为空)。第二行才是我们需要显示的系统环境变量
“PATH”的值(value函数得到变量“FOO”的值为“\$PATH”)。

\end{dinglist}

\section{eval函数}

\begin{dinglist}{226}
\itemsep=4pt \parskip=0pt

\item \textbf{函数功能:}函数“eval”是一个比较特殊的函数。使用它可以在Makefile
    中构造一个可变的规则结构关系(依赖关系链),其中可以使用其它变量和函数。
    函数“eval”对它的参数进行展开,展开的结果作为Makefile的一部分,make可以对
    展开内容进行语法解析。展开的结果可以包含一个新变量、目标、隐含规则或者是
    明确规则等。也就是说此函数的功能主要是:根据其参数的关系、结构,对它们进
    行替换展开。


\item \textbf{返回值:}函数“eval”的返回值时空,也可以说没有返回值。


\item \textbf{函数说明:}“eval”函数执行时会对它的参数进行两次展开。第一次展开
    过
    程发是由函数本身完成的,第二次是函数展开后的结果被作为Makefile内容时由
    make解析时展开的。明确这一过程对于使用“eval”函数非常重要。理解了函数
    “eval”二次展开的过程后。实际使用时,如果在函数的展开结果中存在引用(格式
    为:\$(x)),那么在函数的参数中应该使用“\$\$”来代替“\$”。因为这一点,所以通常
    它的参数中会使用函数“value”来取一个变量的文本值。



\end{dinglist}

我们看一个例子。例子看起来似乎非常复杂,因为它综合了其它的一些概念和函数。不
过我们可以考虑两点:其一,通常实际一个模板的定义可能比例子中的更为复杂;其
二,我们可以实现一个复杂通用的模板,在所有Makefile中包含它,亦可作到一劳永
逸。相信这一点可能是大多数程序员所推崇的。

示例:

\begin{Verbatim}[]
# sample Makefile

PROGRAMS    = server client

server_OBJS = server.o server_priv.o server_access.o
server_LIBS = priv protocol

client_OBJS = client.o client_api.o client_mem.o
client_LIBS = protocol

# Everything after this is generic
.PHONY: all
all: $(PROGRAMS)

define PROGRAM_template
$(1): $$($(1)_OBJ) $$($(1)_LIBS:%=-l%)
ALL_OBJS   += $$($(1)_OBJS)
endef

$(foreach prog,$(PROGRAMS),$(eval $(call PROGRAM_template,$(prog))))

$(PROGRAMS):
    $(LINK.o) $^ $(LDLIBS) -o $@

clean:
    rm -f $(ALL_OBJS) $(PROGRAMS)
\end{Verbatim}

来看一下这个例子:它实现的功能是完成“PROGRAMS”的编译链接。例子中“\$(LINK.o)”
为“\$(CC) \$(LDFLAGS)”,意思是对所有的.o文件和指定的库文件进行链接。

\section{origin函数}

函数“origin”和其他函数不同,函数“origin”的动作不是操作变量(它的参数)。它只
是获取此变量(参数)相关的信息,告诉我们这个变量的出处(定义方式)。
\begin{dinglist}{226}
\itemsep=4pt \parskip=0pt

\item \textbf{函数语法:}

\begin{Verbatim}[]
$(origin VARIABLE)
\end{Verbatim}

\item \textbf{函数功能:}函数“origin”查询参数“VARIABLE”(一个变量名)的出处。

\item \textbf{函数说明:}“VARIABLE”是一个变量名而不是一个变量的引用。因此通常
    它
    不包含“\$”(当然,计算的变量名例外)。

\item \textbf{返回值:}返回“VARIABLE”的定义方式。用字符串表示。

\end{dinglist}

函数的返回情况有以下几种:
\begin{dinglist}{241}
\itemsep=4pt \parskip=0pt
\item \textbf{undefined}

变量“VARIABLE”没有被定义。

\item \textbf{default}

变量“VARIABLE”是一个默认定义(内嵌变量)。如“CC”、“MAKE”、“RM”等%
变量。如果在Makefile中重新定义这些变量,函数返回值将相应发生变化。

\item \textbf{environment}

变量“VARIABLE”是一个系统环境变量,并且make没有使用命令行选项“-e”(Makefile中不存在同名的变量定义,此变量没有被替代)。

\item \textbf{environment override} %

变量“VARIABLE”是一个系统环境变量,并且make使用了命令行选项“-e”。Makefile中存在一个同名的变量定义,使用“make -e”时环境变量值替代了文件中的变量定义。

\item \textbf{file}

变量“VARIABLE”在某一个makefile文件中定义。

\item \textbf{command line} %

变量“VARIABLE”在命令行中定义。

\item \textbf{override} %

变量“VARIABLE”在makefile文件中定义并使用“override”指示符声明。

\item \textbf{automatic} %

变量“VARIABLE”是自动化变量。

\end{dinglist}

函数“origin”返回的变量信息对我们书写Makefile是相当有用的,可以使我们在使用一
个变量之前对它值的合法性进行判断。假设在Makefile其包了另外一个名为bar.mk的
makefile文件。我们需要在bar.mk中定义变量“bletch”(无论它是否是一个环境变
量),保证“make –f bar.mk”能够正确执行。另外一种情况,当Makefile包含bar.mk,
在Makefile包含bar.mk之前有同样的变量定义,但是我们不希望覆盖bar.mk中的
“bletch”的定义。一种方式是:我们在bar.mk中使用指示符“override”声明这个变量。
但是它所存在的问题时,此变量不能被任何方式定义的同名变量覆盖,包括命令行定
义。另外一种比较灵活的实现就是在bar.mk中使用“origin”函数,如下:

\begin{Verbatim}[]
ifdef bletch
ifeq "$(origin bletch)" "environment"
bletch = barf, gag, etc.
endif
endif
\end{Verbatim}

这里,如果存在环境变量“bletch”,则对它进行重定义。
\begin{Verbatim}[]
ifneq "$(findstring environment,$(origin bletch))" ""
bletch = barf, gag, etc.
endif
\end{Verbatim}

这个例子实现了:即使环境变量中已经存在变量“bletch”,无论是否使用“make -e”来执
行Makefile,变量“bletch”的值都是“barf,gag,etc”(在Makefile中所定义的)。环境
变量不能替代文件中的定义。

如果“\$(origin bletch)”返回“environment”或“environment override”,都将对变量
“bletch”重新定义。


\section{shell函数}
shell函数不同于除“wildcard”函数之外的其它函数。make可以使用它来和外部通信。

\begin{dinglist}{226}
\itemsep=4pt \parskip=0pt
\item \textbf{函数功能:}函数“shell”所实现的功能和shell中的引用(\verb"``")
    相同。实
    现对命令
    的扩展。这就意味着需要一个shell 命令作为此函数的参数,函数的返回结果是此
    命令在shell中的执行结果。make仅仅对它的回返结果进行处理;make将函数返回结
    果中的所有换行符(“\verb"\"n”)或者一对“\verb"\"n\verb"\"r”替换为单空格;并去掉末尾的回车符号
    (“\verb"\"n”)或者“\verb"\"n\verb"\"r”。进行函数展开式时,它所调用的命令(它的参数)得到执
    行。除对它的引用出现在规则的命令行和递归变量的定义中以外,其它决大多数情
    况下,make是在读取解析Makefile时完成对函数shell的展开。

\item \textbf{返回值:}函数“shell”的参数(一个shell命令)在shell环境中的执行
    结果。

\item \textbf{函数说明:}函数本身的返回值是其参数的执行结果,没有进行任何处
    理。对结果
    的处理是由make进行的。当对函数的引用出现在规则的命令行中,命令行在执行时
    函数才被展开。展开时函数参数(shell命令)的执行是在另外一个shell进程中完
    成的,因此需要对出现在规则命令行的多级“shell”函数引用需要谨慎处理,否则会
    影响效率(每一级的“shell”函数的参数都会有各自的shell进程)。
\end{dinglist}

示例1:
\begin{Verbatim}[]
contents := $(shell cat foo)
\end{Verbatim}

\noindent 将变量“contents”赋值为文件“foo”的内容,文件中的换行符在变量中使用空
格代替。

示例2:
\begin{Verbatim}[]
files := $(shell echo *.c)
\end{Verbatim}

\noindent 将变量“files”赋值为当前目录下所有.c文件的列表(文件名之间使用空格分
割)。在shell中之行的命令是“echo *.c”,此命令返回当前目录下的所有.c文件列表。
上例的执行结果和函数“\$(wildcard *.c)”的结果相同,除非你使用的是一个奇怪的
shell。

\begin{quote}\kaishu
\textbf{注意:}通过上边的两个例子我们可以看到,当引用“shell”函数的变量定义
使用直接展开式定义时可以保证函数的展开是在make读入Makefile时完成。后续对此
变量的引用就不会有展开过程。这样就可以避免规则命令行中的变量引用在命令行执
行时展开的情况发生(因为展开“shell”函数需要另外的shell进程完成,影响命令的
执行效率)。这也是我们建议的方式。
\end{quote}

\section{make的控制函数}
make提供了两个控制make运行方式的函数。通常它们用在Makefile中,当make执行过程
中检测到某些错误是为用户提供消息,并且可以控制make过程是否继续。

\subsection{\$(error TEXT…)}
\begin{dinglist}{226}
\itemsep=4pt \parskip=0pt
\item \textbf{函数功能:}产生致命错误,并提示“TEXT…”信息给用户,并退出make的
    执行。需要说明的是:“error”函数是在函数展开式(函数被调用时)才提示信息并
    结束make进程。因此如果函数出现在命令中或者一个递归的变量定义中时,在读取
    Makefile时不会出现错误。而只有包含“error”函数引用的命令被执行,或者定义中
    引用此函数的递归变量被展开时,才会提示致命信息“TEXT…”同时退出make。

\item \textbf{返回值:}空。

\item \textbf{函数说明:}“error”函数一般不出现在直接展开式的变量定义中,否则
    在make读取Makefile时将会提示致命错误。
\end{dinglist}

示例1:
\begin{Verbatim}[]
ifdef ERROR1
$(error error is $(ERROR1))
endif
\end{Verbatim}

\noindent make读取解析Makefile时,如果只起那已经定义变量“EROOR1”,make将会提
示致命错误信息“\$(ERROR1)”并退出。

示例2:
\begin{Verbatim}[]
ERR = $(error found an error!)

.PHONY: err
err: ; $(ERR)
\end{Verbatim}

\noindent 这个例子,在make读取Makefile时不会出现致命错误。只有目标“err”被作为
一个目标被执行时才会出现。

\subsection{\$(warning TEXT…)}
\begin{dinglist}{226}
\itemsep=4pt \parskip=0pt
\item \textbf{函数功能:}函数“warning”类似于函数“error”,区别在于它不会导致致
    命错误(make不退出),而只是提示“TEXT…”,make的执行过程继续。

\item \textbf{返回值:}空。

\item \textbf{函数说明:}用法和“error”类似,展开过程相同。
\end{dinglist}
