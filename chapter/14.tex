\chapter{Makefile的约定}

本章讨论书写Makefile时需要遵循的约定。工具“Automake”可以帮助我们创建一个遵循
这些约定的Makefile。所有GNU发布的软件包中的Makefile都是按照这些标准的约定来书
写的。因此理解本章的内容,可帮助很快的熟悉那些开源代码的结构。而对于我们,在
为一个工程书写Makefile时,也尽量遵循这些约定。虽然并没有强求你这么做,但是建
议你还是按照已约定的规则来书写自己的Makefile。

\section{基本的约定}
本节讨论书写Makefile时应遵循的一些基本约定,由于不同版本make之间的差异。可能
在GNU make环境中正常工作的Makefile,换成其它版本的make执行时会发生错误。为了
最大可能的兼容不同版本的make,这里给出了一些基本的约定。

\section{基本的约定}
本节讨论书写Makefile时应遵循的一些基本约定,由于不同版本make之间的差异。可能
在GNU make环境中正常工作的Makefile,换成其它版本的make执行时会发生错误。为了
最大可能的兼容不同版本的make,这里给出了一些基本的约定。

\begin{enumerate}
  \item 所有的Makefile中应该包含这样一行:

\begin{Verbatim}[]
SHELL = /bin/sh
\end{Verbatim}


  其目的是为了避免变量“SHELL”在有些系统上可能继承同名的系统环境变量而导致错
      误。虽然在GNU 版本的make中不会出现这种问题(GNU make 中变量“SHELL”的
      默认值是“/bin/sh”,它不同于系统环境变量“SHELL”)。

  \item 小心处理后缀和隐含规则。不同make可识别后缀和隐含规则可能不同,它可
      能会导致混乱或者错误。因此在特定Makefile中明确限定可识别的后缀是一个
      不错的主意。在Makefile中应该这样做:

\begin{Verbatim}[]
.SUFFIXES:
.SUFFIXES: .c .o
\end{Verbatim}

第一行首先取消掉make默认的可识别后缀列表,第二行重新指定可识别的后缀列表。

  \item 小心处理规则中的路径。当需要处理指定目录的的文件时,应该明确给出路
      径。如“./”代表当前目录,“\$(srcdir)”代表源代码目录。没有指定明确路
      径,那么就意味着是当前目录。

      目录“./”(当前目录,GNU的发布软件包中的“build”目录)和“\$(srcdir)”的
      区别和重要,我们可以通过“configure”脚本的选项“--srcdir”指定源代码所在
      的目录(可参考GNU发布的软件包中的configure脚本)。当源代码目录和build
      目录不同时,规则:

\begin{Verbatim}[]
foo.1 : foo.man sedscript
    sed –e sedscript foo.man > $@
\end{Verbatim}

将执行失败,是因为“foo.man”和“sedscript”并不在当前目录(当然,处理这种错误
的手段可能有很多,诸如使用变量“VPATH”等)。当前目录只是build目录,并不是软
件包目录。

  \item 使用GNU make的变量“VPATH”指定搜索目录。当规则只有一个依赖文件时。应
      该使用自动化变量“\$<”和“\$@”代替出现在命令的依赖文件和目标文件(其它
      版本的make,只在隐含规则中设置自动化变量“\$<”)。对于一个这样的规则:

\begin{Verbatim}[]
foo.o : bar.c
    $(CC) –I. –I$(srcdir) $(CFLAGS) –c bar.o –o foo.o
\end{Verbatim}

我们在Makefile中应该以这种方式来书写:

\begin{Verbatim}[]
foo.o : bar.c
    $(CC) –I. –I$(srcdir) $(CFLAGS) –c $< –o $@
\end{Verbatim}

另外,对于有多个依赖的规则,为了规则能被正确执行。应该在规则的命令行中明确
的指定文件的完整路径名。例如第一个例子就可以这样写(需要在规则之前使用
“VPATH”指定搜索目录):

\begin{Verbatim}[]
foo.1 : foo.man sedscript
    sed –e $(srcdir)/sedscript $(srcdir)/foo.man > $@
\end{Verbatim}

\end{enumerate}

在GNU的发布软件包中,包括了很多非源代码的文件。诸如:“info”文件、“Autoconf”的
输出文件、“Automake”、“Bison”或者“Flex”等文件。这些文件在发布时也存在于源代码
的目录中。因此Makefile中对它们的重建也应该是在源代码目录下,而不应该在build目
录。

相反的,对于那些本来就不存在于源代码目录下的文件,也不应该将它们创建在源代码
的目录下。\textbf{要记住,make的过程不应该以任何方式修改源代码,或者改变源代
码目录的结构}。

\section{规则命令行的约定}
本节讨论书写规则命令的一些约定,在书写多系统兼容的Makefile时,需要注意不同系
统的命令存在不兼容。这里对规则命令行做出了一些书写的基本约定:

\begin{enumerate}
  \item 书写Makefile时,规则的命令(包括其他的脚本文件,如:configure)应该
      是“sh”而不是“csh”所支持的。
  \item 用于创建和安装的“configure”脚本以及Makefile中的命令,除使用下面所列
      出的之外,避免使用其它命令:
\begin{Verbatim}[]
cat cmp cp diff echo egrep expr false grep install-info ln ls mkdir mv pwd
rm rmdir sed sleep sort tar test touch true
\end{Verbatim}

  \item 在目标“dist”的命令行中可以使用压缩工具“gzip”。
  \item 对于可使用的这些工具命令,尽量使用它的通用选项。不要使用那些只在特
      定系统上有效的选项。如:“mkdir -p”这个命令在Linux系统上能够很好的工
      作,但是其它很多系统却并不支持“mkdir”的“-p”选项。
  \item 尽量不要在规则的命令行中创建符号连接文件(使用“ln”命令)。因为有些
      系统不支持符号连接文件(对于类Unix的系统我们基本上没有问题,可能这里
      所说的是MS-DOS系统的系统。我想大家也没有兴趣或者说没有必要在MS-DOS下
      写Makefile,所以这个限制基本可以不考虑)。
  \item 重建或者安装目标(一般是伪目标)的命令行可使用编译器或者相关工具程
      序,这些命令使用一个变量来表示。这样做的好处是:当修改一个命令时,只
      需要更改代表命令的变量的值就可以了。对于以下的这些命令程序:
\begin{Verbatim}[]
ar  bison  cc  flex  install  ld  ldconfig  lex make  makeinfo  ranlib
texi2dvi  yacc
\end{Verbatim}

在规则中的命令中,使用以下这些变量来代表它们:

\begin{Verbatim}[]
$(AR) $(BISON) $(CC) $(FLEX) $(INSTALL) $(LD) $(LDCONFIG)$(LEX) $(MAKE)
$(MAKEINFO) $(RANLIB) $(TEXI2DVI) $(YACC)
\end{Verbatim}



\end{enumerate}

如果规则的命令行需要使用“ranlib”或者“ldconfig”等这些工具时,需要考虑当前的系
统是否支持这些工具。当在不支持它的系统中执行包含此命令的规则时,要能够给出提
示信息(提示原因是告诉用户系统不支持此命令,但不应该出现错误而退出执行)。

对以上没有列出的其它命令,在规则的命令行使用时也应该都是通过变量的形式。例如
如下的这些命令:

\begin{Verbatim}[]
chgrp chmod chown mknod
\end{Verbatim}

我们可以在Makefile中为这些命令组件定义一个代表它的变量(如:CHRP、CHMOD等,在
命令行中就可以使用\$(CHMOD)来引用)。

书写多系统兼容Makefile时需要遵循以上的约定。如果只考虑一种系统时,以上的这些
约定中可以灵活处理,比如:在命令组件的使用上,我们就可以使用这个系统独具的那
些命令组件;系统支持符号链接时,我们就可以在命令行重创建一个符号链接文件。对
于上边的第6个约定,强烈建议大家都遵循。

\section{代表命令变量}
在我们书写的Makefile中应该讲所有的命令、选项作为变量定义,方便后期对命令的修
改和对选项的修改。就是说用户可以通过修改一个变量值来重新指定所要执行的命令,
或者来控制命令执行的选项、参数等。

当使用变量来表示命令时,如果规则中需要使用此命令时,可通过引用代表此命令的变
量来实现。例如:定义变量“CC = gcc”,规则中就可使用“\$(CC)”来引用“gcc”。对于一
些件管理器工具如“ln”,“rm”“mv”等,可以不为它们定义变量,而直接使用。

所有命令执行的参数选项也应该定义一个变量(可称为命令的选项变量)。在命令变量
(代表一个命令的变量)后添加“FLAGS”来命名这个选项变量。例如:变量“CFLAGS”是c
编译器(命令变量为“CC”)的命令行选项变量;变量YFLAGS时命令“yacc”(命令变量为
“YACC”)选项变量;变量“LDFLAGS”是命令“ld”(命令变量为“LD”)的选项变量等。在所
有需要执行预处理的命令行应该使用变量“CCFLAGS”作为gcc的执行参数;同样任何需要
执行链接的命令行中使用“LDFLAGS”作为命令的执行参数。

c编译器的编译选项变量“CFLAGS”在Makefile中通常是为编译所有的源文件提供选项变
量。为编译一个特定文件增加的选项,不应包含在变量“CFLAGS”中。编译特定文件(或
者一类特定文件)时,如果需要使用特定的选项参数,可以将这些选项写在编译它所执
行规则的命令行中(也可以使用目标指定变量或者模式指定变量)。例如:
\begin{Verbatim}[]
CFLAGS = -g
ALL_CFLAGS = -I $(CFLAGS)
.c.o:
    $(CC) -c $(CPPFLAGS) $(ALL_CFLAGS) $<
\end{Verbatim}

例子中,变量“CFLAGS”指定编译选项为“-g”,在本例中它作为缺省的编译选项。对于所
有源文件的编译都使用“-g”选项。双后缀规则的命令行中为编译生成“.o”文件指定了另
外的选项“-I. -g”。

在所有编译命令行中,变量“CFLAGS”应该放在编译选项列表的最后。这样可以保证当命
令行参数出现重复时,“CFLAGS”始终效的。另外,在任何调用c编译器的命令行中都应该
使用选项变量“CFLAGS”,无论是进行编译还是连接。

如果需要在Makefile中实现文件安装的规则,那么就需要在Makefile中定义变量
“INSTALL”。此变量代表安装命令(install)。同时在Makefile中也需要定义变量
“INSTALL\_PROGRAM”和“INSTALL\_DATA”(“INSTALL\_PROGRAM”的缺省值都是
“\$(INSTALL)”;“INSTALL\_DATA”的缺省值是“\$\{INSTALL\} –m 644”)。可以使用这
些变量,来安装可执行程序或者非可执行程序到指定位置。例如:

\begin{Verbatim}[]
$(INSTALL_PROGRAM) foo $(bindir)/foo $(INSTALL_DATA) libfoo.a
$(libdir)/libfoo.a
\end{Verbatim}

另外,也可以使用变量“DESTDIR”来指定目标需要安装的目录。通常也可以不在Makefile
定义变量“DESTDIR”,可通过make命令行参数的形式来指定。例如:“make
DESTDIR=exec/ install”。因此上边的命令就可以这样实现:

\begin{Verbatim}[]
$(INSTALL_PROGRAM) foo $(DESTDIR)$(bindir)/foo $(INSTALL_DATA) libfoo.a
$(DESTDIR)$(libdir)/libfoo.a
\end{Verbatim}

安装命令中使用文件名而不是目录名作为第二个参数。每一个需要安装的文件使用单独
的命令(包括安装一个目录)。

\section{安装目录变量}
在Makefile中,安装目录同样需要使用变量来指定,这样就可以很方便的修改文件的安
装路径。安装目录的标准命名下边将一一介绍。这些变量基于标准的文件系统结构,这
些变量的变种在SVR4、4.4BSD、Linux、Ultrix v4以及其它现代操作系统中都有使用。

以下所罗列的两个变量是指定安装文件的根目录。所有其它安装目录都是它们的子目
录。注意:文件不能直接安装在这两个目录下。

\begin{dinglist}{226}
\itemsep=4pt \parskip=0pt

\item \textbf{prefix}

这个变量(通常作为实际文件安装目录的父目录,可以理解为其它实际文件安装目录的
前缀)用于构造下列(除这两个安装根目录以外的其它目录变量)变量的缺省值。变量
“prefix”缺省值是“/usr/local”。创建完整的GNU系统时,变量prefix的缺省值是空值,
“/usr”是“/”的符号连接符文件。(如果使用“Autoconf”工具,它应该写成
“@prefix@”)。注意:当更改了变量“prefix”以后重新执行“make install”,不会导致
可执行程序(终极目标)的重建。

\item \textbf{exec\_prefix}

这个前缀用于构造下列变量的缺省值。变量“exec\_prefix”缺省值是“\$(prefix)”(如
果使用“Autoconf”工具,它应该写为“@exec\_prefix@”)。通常,“\$(exec\_prefix)”
目录中的子目录下存放和机器相关文件(例如可执行文件和例程库)。“\$(prefix)”目
录的子目录存放通用的一般文件。同样:改变“exec\_prefix”的值之后执行“make
install”,不会重建可执行程序(终极目标)。

\end{dinglist}

文件(包括可执行程序、说明文档等)的安装目录:

\begin{dinglist}{226}
\itemsep=4pt \parskip=0pt

\item \textbf{bindir}

用于安装一般用户可运行的可执行程序。通常它的值为:“/usr/local/bin”,使用时应
写为:“\$(exec\_prefix)/bin”。 (使用“Autoconf”工具时,应该为“@bindir@”)。

\item \textbf{sbindir}

安装可在shell中直接调用执行的程序。这些命令仅对系统管理员有用(系统管理工
具)。通常它的值为:“/usr/local/sbin”,要求在使用时应写为:
“\$(exec\_prefix)/sbin”。(使用“Autoconf”工具时,应该为“@sbindir@”)。

\item \textbf{libexecdir}

用于安装那些通常不是由用户直接使用,而是由其它程序调用的可执行程序。通常它的
值为:“/usr/local/libexec”,要求在使用时应写为:“\$(exec\_prefix)/libexec”。
(使用“Autoconf”工具时,应该为“@libexecdir@”)。

\end{dinglist}

程序执行时使用的数据文件可从以下两个方面来分类:

\begin{enumerate}
  \item 是否可由程序更改。分为两类:程序可修改和不可修改的文件(虽然用户可
      编辑其中某些文件)。
  \item 是否和体系结构相关。分为两类:体系结构无关文件,可被所有类型的机器
      共享;体系结构相关文件,仅可被相同类型机器、操作系统共享;其它的就是
      那些不能被任何两个机器共享的文件。
\end{enumerate}

这样就存在六种不同的可能。除编译生成的目标文件(.o文件)和库文件以外,不推荐
使用那些和特定机器体系结构相关的文件,使用和体系无关的数据文件更加简洁,而
且,它的实现也并不非常困难。

在Makefile中应该使用以下变量为不同类型的文件指定对应的安装目录:
\begin{dinglist}{226}
\itemsep=4pt \parskip=0pt

\item \textbf{datadir}

用于安装和机器体系结构无关的只读数据文件。通常它的值为:“/usr/local/share”,
使用时应写为:“\$(prefix)/share”。(使用“Autoconf”工具时,应该为
“@datadir@”)。“\$(infodir)”和“\$(includedir)”作为例外情况,参考后续对它们的
详细描述。

\item \textbf{sysconfdir}

用于安装从属于特定机器的只读数据文件,包括:主机配置文件、邮件服务、网络配置
文件、“/etc/passwd”文件等。所有该目录下的文件都应该是普通文本文件(可识别的
“ASCII”码文本文件)。通常它的值为:“/usr/local/etc”,在使用时应写为:
“\$(prefix)/etc”。(使用“Autoconf”工具时,应该为“@sysconfdir@”)。

不要将可执行文件安装在这个目录下(可执行文件的安装目录应该是“\$(libexecdir)”
或者“\$(sbindir)”)。也不要在这个目录下安装那些需要更改的文件(系统的配置文件
等)。这些文件应该安装在目录“\$(localstatedir)”下。

\item \textbf{sharedstatedir}

用于安装那些可由程序运行时修改的文件,这些文件与体系结构无关。通常它的值为:
“/usr/local/com”,要求在使用时应写为:“\$(prefix)/com”。 (使用“Autoconf”工具
时,应该为“@sharedstatedir@”)。

\item \textbf{localstatedir}

用于安装那些可由程序运行时修改的文件,但这些文件和体系结构相关。用户没有必要
通过直接修改这些文件来配置软件包,对于不同的配置文件,将它们放在“\$(datadir)”
或者“\$(sysconfdir)”目录中。“\$(localstatedir)”值通常为:“/usr/local/var”,在
使用时应写为:“\$(prefix)/var”。(使用“Autoconf”工具时,应该为
“@localstatedir@”)。


\item \textbf{libdir}

用于存放编译后的目标文件(.o)文件库文件(文档文件或者执行的共享库文件)。不
要在此目录下安装可执行文件(可执行文件应该安装在目录“\$(libexecdir)”下)。变
量libdir值通常为:“/usr/local/lib”,使用时应写为:“\$(exec\_prefix)/lib”。
(使用“Autoconf”工具时,应该为“@libdir@”)。


\item \textbf{infodir}

用于安装软件包的 Info 文件。它的缺省值为:“/usr/local/info”,使用时应写为:
“\$(prefix)/info”。(使用“Autoconf”工具时,应该为“@infodir@”)。

\item \textbf{lispdir}

用于安装软件包的Emacs Lisp 文件的目录。它的缺省值为:
“/usr/local/share/emacs/site-lisp”,使用时应写为:
“\$(prefix)/share/emacs/site-lisp”。当使用Autoconf工具时,应将写为
“@lispdir@”。为了保证“@lispdir@”能够正常工作,需要在“configure.in”文件中包含
如下部分:

\begin{Verbatim}[]
lispdir='${datadir}/emacs/site-lisp'
AC_SUBST(lispdir)
\end{Verbatim}

\item \textbf{includedir}

用于安装用户程序源代码使用“\#include”包含的头文件。它的缺省值为:
“/usr/local/include”,使用时应写为:“\$(prefix)/include”。 (使用“Autoconf”工
具时,应该为“@includedir@”)。

除gcc外的大多数编译器不会在目录“/usr/local/include”中搜寻头文件,因此这种方式
只适用gcc编译器。这一点应该不是一个问题,因为很多情况下一些库需要gcc才能工
作。对那些依靠其它编译器的库文件,需要将头文件安装在两个地方,一个由变量
“includedir”指定,另一个由变量“oldincludedir”指定。

\item \textbf{oldincludedir}

它所指定的目录也同样用于安装头文件,这些头文件用于非gcc的编译器。它的缺省值
为:“/usr/include”。(使用“Autoconf”工具时,应该为“@oldincludedir@”)。

Makefile在安装头文件时,需要判断变量“oldincludedir”的值是否为空。如果为空,就
不使用它进行头文件的安装(一般是安装完成“/usr/local/include”下的头文件之后才
安装此目录下的头文件)。

一个软件包的安装不能替换该目录下已经存在的头文件,除非是同一个软件包(重新使
用相同的软件包在此目录下安装头文件)。例如,软件包“Foo”需要在“oldincludedir”
指定的目录下安装一个头文件“foo.h”时,可安装的条件为:1. 目录
“\$(oldincludedir)”目录下不存在头文件“foo.h”;2. 已经存在头文件“foo.h”,存在
的头文件“foo.h”是之前软件包“Foo”安装的。

检查头文件“foo.h”是否来自于软件包Foo,需要在头文件的注释中包含一个“magic”字符
串,使用命令“grep”来在该文件中查找这个magic。
\end{dinglist}

Unix风格的帮助文件需要安装在以下目录中:

\begin{dinglist}{226}
\itemsep=4pt \parskip=0pt

\item \textbf{mandir}

安装该软件包的帮助文档(如果有)的顶层目录。它的缺省值为:“/usr/local/man”,
要求在使用时应写为:“\$(prefix)/man”。 (使用“Autoconf”工具时,应该为
“@@mandir@@”)。

\item \textbf{mandir}

安装该软件包的帮助文档(如果有)的顶层目录。它的缺省值为:“/usr/local/man”,
要求在使用时应写为:“\$(prefix)/man”。 (使用“Autoconf”工具时,应该为
“@@mandir@@”)。

\item \textbf{man1dir}

用于安装帮助文档的第一节(man 1)。它的缺省值为:“\$(mandir)/man1”。。

\item \textbf{man2dir}

用于安装帮助文档的第二节(man 2)。它的缺省值为:“\$(mandir)/man2”。

\item \textbf{manext}

文件名扩展字,它是对安装手册的扩展。以点号(.)开始的十进制数。缺省值为:
“.1”。

\item \textbf{man1ext}

帮助文档的第一节(man 1)的文件名扩展字。

\item \textbf{man2ext}

帮助文档的第一节(man 2)的文件名扩展字。

当一个软件包的帮助手册有多个章节时,使用这些变量代替“manext”。(第一节
“man1ext”,第二节“man2ext”,第三节“man3ext”……)。
\end{dinglist}

而且下列这些变量也应该在Makefile中定义:

\begin{dinglist}{226}
\itemsep=4pt \parskip=0pt

\item \textbf{srcdir}

安装该软件包的帮助文档(如果有)的顶层目录。它的缺省值为:“/usr/local/man”,
要求在使用时应写为:“\$(prefix)/man”。 (使用“Autoconf”工具时,应该为
“@@mandir@@”)。

\item \textbf{mandir}

此变量指定的目录是需要编译的源文件所在的目录。该变量的值
    在使用“configure”脚本对软件包进行配置时产生的。(使用“Autoconf”工具,应该
    书写为“srcdir = @srcdir@”)。

例如:
\begin{Verbatim}[]
# 安装的普通目录路径前缀。
# 注意:该目录在开始安装前必须存在
prefix = /usr/local
exec_prefix = $(prefix)
# 放置“gcc”命令使用的可执行程序
bindir = $(exec_prefix)/bin
# 编译器需要的目录
libexecdir = $(exec_prefix)/libexec
# 软件包的Info文件所在目录
infodir = $(prefix)/info
\end{Verbatim}

\end{dinglist}

在用户标准指定的目录下安装大量文件时,可以将这些文件分类安装在指定目录的多个
子目录下。可以在Makefile中实现一个“install”伪目标来描述安装这些文件的命令(包
括创建子目录,安装文件到对应的子目录中)。

在发布的软件包中,不能强制要求用户必须指定这些安装目录的变量。使用一套标准的
安装目录变量来指定安装目录,当用户需要指定安装目录时,通过修改变量定义来指定
具体的目录,在用户没有指定的情况下,使用默认的目录。

\section{Makefile的标准目标名}
所有GNU发布的软件包的Makefile中,必须包含以下这些目标:

\begin{dinglist}{226}
\itemsep=4pt \parskip=2pt

\item \textbf{all}

此目标的动作是编译整个软件包。“all”应该为Makefile的终极目标。该目标的动作不重
建任何文档(只编译所有的源代码,生成可执行程序);Info文件应该作为发布文件的
一部分,DVI文件只在明确指定的时候才应该被重建。

缺省情况下,对所有源程序的编译和连接应该使用选项“-g”,是最终的可执行程序中包
含调试信息。当最终的可执行程序不需要包含调试信息时,可使用“strip”去掉可执行程
序中的调试符号以减小最终的程序大小。

\item \textbf{install}

此目标的动作是完成程序的编译并将最终的可执行程序、库文件等拷贝到安装的目录。
如果只是验证这些程序是否可被正确安装,它的动作应该是一个测试安装动作。

安装时一般不要对可执行程序进行strip(去掉可执行程序内部的调试信息)。存在另外
一个目标“install-strip”,它实现安装的同时完成对可执行程序strip。

保证目标“install”的动作不更改程序创建目录(builid目录)下的任何文件,对这个目
录下文件的修改(重建或者更新)是目标“all”所要定义的动作。

“install”目标定义的动作在安装目录不存在时,能够创建这些不存在的安装目录。这些
目录包括:变量“prefix”和“exec\_prefix”指定的目录和所有必要的子目录。完成此任务
的方式可以使用下边介绍的“installdirs”目标。

在安装man文档的命令前使用“-”忽略这安装命令的错误,这样可以避免在没有Unix man
文档的系统上执行安装时出现错误。

安装Info文档的方法是使用变量“INSTALL\_DATA”将Info文档拷贝到“\$(infodir)”目录下
去,如果存在“install-info”命令则执行它。“install-info”是一个编辑Info“dir”文件
的程序,更新或者修改“info”文档的入口和目录;它是Texinfo软件包的一部分。这里有
一个安装Info文档的例子:

\begin{Verbatim}[]
$(DESTDIR)$(infodir)/foo.info: foo.info
    $(POST_INSTALL)
# 可能在“.”(当前目录)存在一个新的文档,而不是“srcdir”。
    -if test -f foo.info; then d=.; \
    else d=$(srcdir); fi; \
    $(INSTALL_DATA) $$d/foo.info $(DESTDIR)$@; \
#如果install-info命令存在则运行它
# 使用“if”代替在命令行前的“-”
# 这样,就可以看到运行install-info产生的真正错误
# 我们使用“$(SHELL) -c”是因为在一些shell中
# 遇到未知的命令不会运行失败
    if $(SHELL) -c 'install-info --version' \
        >/dev/null 2>&1; then \
        install-info --dir-file=$(DESTDIR)$(infodir)/dir \
            $(DESTDIR)$(infodir)/foo.info; \
    else true; fi
\end{Verbatim}

目标install的命令需要分为三类:正常命令、预安装命令和安装后命令。

\item \textbf{uninstall}

删除所有已安装文件——由install创建的文件拷贝。规则所定义的命令不能修改编译目录
下的文件,仅仅是删除安装目录下的文件。像install目标的命令一样,uninstall目标
的命令也分为三类。参考 14.6 安装命令分类 一节。

\item \textbf{install-strip}

和目标install的动作类似,但是install-strip指定的命令在安装时对可执行文件进行
strip(去掉程序内部的调试信息)。它的定义如下:

\begin{Verbatim}[]
install-strip:
    $(MAKE) INSTALL_PROGRAM='$(INSTALL_PROGRAM) -s'  install
\end{Verbatim}

如果软件包的存在安装脚本时,目标install-strip所定义的命令就不能是对目标
“install”的引用,它仅仅完成对可执行文件的strip。

“install-strip”不应该直接在build目录下对可执行文件进行strip,应该是对安装目录
下的可执行文件进行strip。就是说“install-strip”所定义的命令不能对build目录下的
文件产生影响。

一般不建议安装时对可执行文件进行strip,因为去掉可执行文件的调试信息后,如果在
程序中存在bug,就不能通过gdb对程序进行调试。


\item \textbf{clean}

清除当前目录下编译生成的所有文件,这些文件在make过程中产生。注意,clean动作不
能删除软件包的配置文件,同时也不能删除build时创建的那些文件(诸如:目录、
build生成的信息记录文件等)。因为这些文件都是发布版本的一部分。

对于.dvi文件,当它不作为发布版本的一部分时,可以删除。

\item \textbf{distclean}

类似于目标clean,但增加删除当前目录下的的配置文件、build过程产生的文件。目标
“distclean”指定的删除命令应该删除软件包中所有非发布文件。

\item \textbf{mostlyclean}

类似于目标“clean”,但是可保留一些编译生成的文件,避免在下次编译时对这些文件重
建。例如,对于gcc来说,此目标指定的命令不删除文件“libgcc.a”,因为在绝大多数情
况下它都不需要重新编译。

\item \textbf{maintainer-clean}

此目标所定义的命令几乎会删除所有当前目录下能够由Makefile重建的文件。典型的,
包括目标“distclean”删除的文件、由Bison生成的.c源文件、tags记录文件、Ifon文件
等。但是有一个例外,就是执行“make maintainer-clean”不能删除“configure”这个配
置脚本文件,即使“configure”可以由Makefile生成。因为“configure”是软件包的配置
脚本。

目标“maintainer-clean”应该只能由维护软件包的用户使用,而不能被普通用户使用。
因为它会删除一些软件包的发布文件,而重建这些文件可能需要专门的工具。因此我们
在使用此目标是需要小心。为了让用户能够在执行前得到提示,通常目标
“maintainer-clean”的命令以下两行为开始:

@echo“该命令用于维护此软件包的用户使用”;

@echo“它删除的文件可能需要使用特殊的工具来重建。”

\item \textbf{TAGS}

此目标所定义的命令完成对该程序的tags记录文件的更新。tags文件通常可被编辑器作
为符号记录文件,例如vim,Emacs等。

\item \textbf{info}

产生必要的Info文档。此目标应该按照如下书写:

\begin{Verbatim}[]
info: foo.info

foo.info: foo.texi chap1.texi chap2.texi
$(MAKEINFO) $(srcdir)/foo.texi
\end{Verbatim}


必须在Makefile中定义变量“MAKEINFO”,代表命令工具makeinfo,该工具是发布软件
Texinfo的一部分。

通常,GNU的发布程序会和Info文档会被一同创建,这意味着Info文档是在源文件的目录
下。用户在创建发布软件时,一般情况下,make不更新Info文档,因为它们已经更新到
最新了。

\item \textbf{dvi}

为所有的Texinfo文件创建对应的DVI文件。例如:

\begin{Verbatim}[]
dvi: foo.dvi

foo.dvi: foo.texi chap1.texi chap2.texi
    $(TEXI2DVI) $(srcdir)/foo.texi
\end{Verbatim}

必须在Makefile中定义变量“TEXI2DVI”。它代表命令工具texi2dvi,该工具是发布软件
Texinfo一部分。

规则中也可以没有命令行,这样make程序会自动为它推导对应的命令。

\item \textbf{dist}

此目标指定的命令创建发布程序的tar文件。创建的tar文件应该是这个软件包的目录,
文件名中也可以包含版本号(就是说创建的tar文件在解包之后应该是一个目录)。例
如,发布的gcc 1.40版的tar文件解包的目录为“gcc-1.40”。

通常的做法是是创建一个空目录,如使用ln或cp将所需要的文件加入到这个目录中,之
后对这个目录使用tar进行打包。打包之后的tar文件使用gzip压缩。例如,实际的gcc
1.40版的发布文件叫“gcc-1.40.tar.gz”。

目标“dist”的依赖文件为软件包中所有的非源代码的文件,因此在使用目标进行发布软
件打包压缩之前必须保证这些文件是最新的。


\item \textbf{check}

此目标指定的命令完成所有的自检功能。在执行检查之前,应确保所有程序已经被创
建,可以不安装。为了对它们进行测试,需要实现在程序没有安装的情况下被执行的规
则命令。

\item \textbf{installcheck}

执行安装检查。在执行安装检查之前,确保所有程序已经被创建并且被安装。需要注意
的是:安装目录“\$(bindir)”是否在搜索路径中。

\item \textbf{installdirs}

使用目标“installdirs”创建安装目录以及它的子目录在很多场合是非常有用的。脚本
“mkinstalldirs”就是为了实现这个目的而编写的;发布的Texinfo软件包中就包含了这
个脚本文件。Makefile中的规则可以这样书写:

\begin{Verbatim}[]
# 确保所有安装目录(例如 $(bindir))存在,如有必要则创建这些目录
installdirs: mkinstalldirs
    $(srcdir)/mkinstalldirs $(bindir) $(datadir) \
        $(libdir) $(infodir) \
        $(mandir)
\end{Verbatim}

或者可以使用变量“DESTDIR”:

\begin{Verbatim}[]
installdirs: mkinstalldirs
    $(srcdir)/mkinstalldirs \
        $(DESTDIR)$(bindir)  $(DESTDIR)$(datadir) \
        $(DESTDIR)$(libdir)  $(DESTDIR)$(infodir) \
        $(DESTDIR)$(mandir)
\end{Verbatim}

该规则不能更改软件的编译目录,仅仅是创建程序的安装目录。

\end{dinglist}

\section{安装命令分类}
在为Makefile书写“install”目标时,需要将其命令分为三类:正常命令、安装前命令和
安装后命令。

正常命令是把文件移动到合适的地方,并设置它们的模式。这个过程不修改任何文件,
仅仅是把需要安装的文件从软件包中拷贝到安装目录。

安装前命令和安装后命令可能修改某些文件;通常,修改一些配置文件和系统的数据库
文件。典型地,安装前命令在正常命令之前执行,安装后命令在正常命令执行后执行。

大多数情况是,安装后命令是运行“install-info”程序。它所完成的工作不能由正常命
令完成,因为它更新了一个文件(Info的目录),该文件不能单独的或者完整的从软件
包来进行安装,因为它只能在正常安装命令完成安装软件包的info文档之后才可正确执
行。

大多数程序不需要安装前命令,但应该在Makefile中提供。

将“install”规则的命令分为这三类时,应该在命令之间插入分类行(category
lines)。分类行说明了后续需要执行的命令的类别。

分类行是由一个[Tab]字符开始的对make特殊变量的引用,行尾是可选的注释内容。可以
使用三个特殊的变量,每一个代表一种类别。分类行执行一个空动作,因为这三个特殊
的make变量没有定义(我们书写的Makefile中也不能定义它们)。

以下是三种可能的分类行,以及它们的解释:
\begin{Verbatim}[]
$(PRE_INSTALL)       # 以下是安装前命令
$(POST_INSTALL)      # 以下是安装后命令
$(NORMAL_INSTALL)    # 以下是正常命令
\end{Verbatim}

如果安装规则(install所在的规则)的命令行没有使用分类行,那么在第一个出现的分
类行之前的所有的命令行都被认为是正常命令。如果命令行中没有分类行,那么规则的
所有命令行都都被认为是正常命令行。

对应的,以下这三个是“uninstall”命令的分类行:
\begin{Verbatim}[]
$(PRE_UNINSTALL)         #以下是卸载前命令
$(POST_UNINSTALL)        #以下是卸载后命令
$(NORMAL_UNINSTALL)      #以下是正常命令
\end{Verbatim}

卸载前命令应该是取消软件包Info文档的入口。

如果目标install或 uninstall存在依赖,其作为安装的子例程,那么就应该在每一个依
赖目标的命令行开始使用分类行,同时目标insall和uninstall的命令行开始也需要使用
分类行。这样就可以确保任何调用方式时每一条命令都被正确的分类。

除下列命令外,安装前命令和安装后命令不应该使用其它命令:
\begin{Verbatim}[]
basename bash cat chgrp chmod chown cmp cp dd diff echo egrep expand expr
false fgrep find getopt grep gunzip gzip hostname install install-info kill
ldconfig ln ls md5sum mkdir mkfifo mknod mv printenv pwd rm rmdir sed sort
tee test touch true uname xargs yes
\end{Verbatim}

按照这种方式分类命令的原因是为了创建二进制的软件包。典型的二进制软件包包括可
执行文件、必须安装的其它文件以及它自己的安装文件,因此二进制软件包就不需要任
何正常命令、只需要安装前命令和安装后命令。

创建二进制软件包的程序通过提取安装前命令和安装后命令工作。这里有一个抽取安装
前命令的方法:

\begin{Verbatim}[]
make -n install -o all \
    PRE_INSTALL=pre-install \
    POST_INSTALL=post-install \
    NORMAL_INSTALL=normal-install \
    | gawk -f pre-install.awk
\end{Verbatim}

文件“pre-install.awk”可能包括以下内容:
\begin{Verbatim}[]
$0 ~ /^\t[ \t]*(normal_install|post_install)[ \t]*$/ {on = 0}
on {print $0}
$0 ~ /^\t[ \t]*pre_install[ \t]*$/ {on = 1}
\end{Verbatim}

安装前命令的执行结果和软件包的安装shell脚本的结果一样。
