\chapter{make的常见错误信息}

本章对make执行时可能出现常见错误进行汇总、分析,并给出修正的可能方法。

make执行过程中所产生错误并不都是致命的;特别是在命令行之前存在“-”、或者make使
用“-k”选项执行时。make执行过程的致命错误都带有前缀字符串“***”。

错误信息都有前缀,一种是执行程序名作为错误前缀(通常是“make”);另外一种是当
Makefile本身存在语法错误无法被make解析并执行时,前缀包含了makefile文件名和出
现错误的行号。

在下述的错误列表中,省略了普通前缀:

\begin{dinglist}{226}
\itemsep=4pt \parskip=2pt

  \item \begin{minipage}[t]{\linewidth}
          \textbf{[FOO] Error NN} \\
          \textbf{[FOO] signal description}
        \end{minipage}

这类错误并不是make的真正错误。它表示make检测到make所调用的作为执行命令的程序
返回一个非零状态(Error NN),或者此命令程序以非正常方式退出(携带某种信
号),参考 5.4 命令的错误 一节。

如果错误信息中没有附加“***”字符串,则是子过程的调用失败,如果Makefile中此命令
有前缀“-”,make会忽略这个错误。

  \item \begin{minipage}[t]{\linewidth}
          \textbf{missing separator. Stop.} \\
          \textbf{missing separator (did you mean TAB instead of 8 spaces?). Stop.}
        \end{minipage}

不可识别的命令行,make在读取Makefile过程中不能解析其中包含的内容。GNU make在
读取Makefile时根据各种分隔符(:, =, [TAB]字符等)来识别Makefile的每一行内容。
这些错误意味着make不能发现一个合法的分隔符。

出现这些错误信息的可能的原因是(或许是编辑器,绝大部分是ms-windows的编辑器)
在Makefile中的命令之前使用了4个(或者8个)空格代替了[Tab]字符。这种情况,将产
生上述的第二种形式产生错误信息。且记,所有的命令行都应该是以[Tab]字符开始的。

  \item \begin{minipage}[t]{\linewidth}
          \textbf{commands commence before first target. Stop.} \\
          \textbf{missing rule before commands. Stop.}
        \end{minipage}

Makefile可能是以命令行开始:以[Tab]字符开始,但不是一个合法的命令行(例如,一
个变量的赋值)。命令行必须和规则一一对应。

产生第二种的错误的原因可能是一行的第一个非空字符为分号,make会认为此处遗漏了
规则的“target: prerequisite”部分。

  \item \begin{minipage}[t]{\linewidth}
          \textbf{No rule to make target `XXX'.} \\
          \textbf{No rule to make target ` XXX ', needed by `yyy'.}
        \end{minipage}

无法为重建目标“XXX”找到合适的规则,包括明确规则和隐含规则。

修正这个错误的方法是:在Makefile中添加一个重建目标的规则。其它可能导致这些错
误的原因是Makefile中文件名拼写错误,或者破坏了源文件树(一个文件不能被重建,
可能是由于依赖文件的问题)。

  \item \begin{minipage}[t]{\linewidth}
          \textbf{No targets specified and no makefile found. Stop.} \\
          \textbf{No targets. Stop.}
        \end{minipage}

第一个错误表示在命令行中没有指定需要重建的目标,并且make不能读入任何makefile
文件。第二个错误表示能够找到makefile文件,但没有终极目标或者没有在命令行中指
出需要重建的目标。这种情况下,make什么也不做。参考 第九章 执行make。

  \item \begin{minipage}[t]{\linewidth}
          \textbf{Makefile `XXX' was not found.} \\
          \textbf{Included makefile `XXX' was not found.}
        \end{minipage}

没有使用“-f”指定makefile文件,make不能在当前目录下找到默认Makefile(makefile
或者GNUmakefile)。使用“-f”指定文件,但不能读取这个指定的makefile文件。

  \item \begin{minipage}[t]{\linewidth}
          \textbf{warning: overriding commands for target `XXX'} \\
          \textbf{warning: overriding commands for target `XXX'}
        \end{minipage}

对同一目标“XXX”存在一个以上的重建命令。GNU make规定:当同一个文件作为多个规则
的目标时,只能有一个规则定义重建它的命令(双冒号规则除外)。如果为一个目标多
次指定了相同或者不同的命令,就会产生第一个告警;第二个告警信息说新指定的命令
覆盖了上一次指定的命令。

  \item \begin{minipage}[t]{\linewidth}
          \textbf{Circular XXX <- YYY dependency dropped.}
        \end{minipage}

规则的依赖关系产生了循环:目标“XXX”的依赖文件为“YYY”,而依赖“YYY”的依赖列表中
又包含“XXX”。

  \item \begin{minipage}[t]{\linewidth}
          \textbf{Recursive variable `XXX' references itself (eventually). Stop.}
        \end{minipage}

make的变量“XXX”(递归展开式)在替换展开时,引用它自身。无论对于直接展开式变量
(通过:=定义的)或追加定义(+=),这都是不允许的。

  \item \begin{minipage}[t]{\linewidth}
          \textbf{Unterminated variable reference. Stop.}
        \end{minipage}

变量或者函数引用语法不正确,没有使用完整的的括号(缺少左括号或者右括号)。

  \item \begin{minipage}[t]{\linewidth}
          \textbf{insufficient arguments to function `XXX'. Stop.}
        \end{minipage}

函数“XXX”引用时参数数目不正确。函数缺少参数。

  \item \begin{minipage}[t]{\linewidth}
          \textbf{missing target pattern. Stop.} \\
          \textbf{multiple target patterns. Stop.} \\
          \textbf{target pattern contains no `\%'. Stop.} \\
          \textbf{mixed implicit and static pattern rules.  Stop.}
        \end{minipage}

不正确的静态模式规则。

第一条错误的原因是:静态模式规则的目标段中没有模式目标;

第二条错误的原因是:静态模式规则的目标段中存在多个模式目标;

第三条错误的原因是:静态模式规则的目标段目标模式中没有包含模式字符“\%”;

第四条错误的原因是:静态模式规则的三部分都包含了模式字符“\%”。正确的应该是只
有后两个才可以包含模式字符“\%”。

关于静态模式规则可参考 4.12 静态模式 一节。

  \item \begin{minipage}[t]{\linewidth}
          \textbf{warning: -jN forced in submake: disabling jobserver mode.}
        \end{minipage}

这一条告警和下条告警信息发生在:make检测到递归的make调用时,可通信的子make进
程出现并行处理的错误。递归执行的make的命令行参数中存在“-jN”参数(N的值大于
1),在有些情况下可能导致此错误,例如:Makefile中变量“MAKE”被赋值为“make
–j2”,并且递归调用的命令行中使用变量“MAKE”。在这种情况下,被调用make进程不能
和其它make进程进行通信,其只能简单的独立的并行处理两个任务”。

  \item \begin{minipage}[t]{\linewidth}
          \textbf{warning: jobserver unavailable: using -j1. Add `+' to parent make rule.}
        \end{minipage}

为了现实make进程之间的通信,上层make进程将传递信息给子make进程。在传递信息过
程中可能存在这种情况,子make进程不是一个实际的make进程,而上层make却不能确定
子进程是否是真实的make进程。它只是将所有信息传递下去。上层make采用正常的算法
来决定这些。当出现这种情况,子进程只会接受父进程传递的部分有用的信息。子进程
会产生该警告信息,之后按照其内建的顺序方式进行处理。

\end{dinglist}
