\chapter{Makefile中的变量}

在Makefile中,变量是一个名字(像是C语言中的宏),代表一个文本字符串(变量的
值)。在Makefile的目标、依赖、命令中引用变量的地方,变量会被它的值所取代(与C
语言中宏引用的方式相同,因此其他版本的make也把变量称之为“宏”)。在Makefile中
变量有以下几个特征。

\begin{enumerate}
\itemsep=4pt \parskip=0pt
  \item Makefile中变量和函数的展开(除规则命令行中的变量和函数以外),是在
      make读取makefile文件时进行的,这里的变量包括了使用“=”定义和使用指示符
      “define”定义的。
  \item 变量可以用来代表一个文件名列表、编译选项列表、程序运行的选项参数列
      表、搜索源文件的目录列表、编译输出的目录列表和所有我们能够想到的事
      物。
  \item 变量名是不包括“:”、“\#”、“=”、前置空白和尾空白的任何字符串。需要注
      意的是,尽管在GNU make中没有对变量的命名有其它的限制,但定义一个包含
      除字母、数字和下划线以外的变量的做法也是不可取的,因为除字母、数字和
      下划线以外的其它字符可能会在make的后续版本中被赋予特殊含义,并且这样
      命名的变量对于一些shell来说是不能被作为环境变量来使用的。
  \item 变量名是大小写敏感的。变量“foo”、“Foo”和“FOO”指的是三个不同的变量。
      Makefile传统做法是变量名是全采用大写的方式。推荐的做法是在对于内部定
      义定义的一般变量(例如:目标文件列表objects)使用小写方式,而对于一些
      参数列表(例如:编译选项CFLAGS)采用大写方式,但这并不是要求的。但需
      要强调一点:对于一个工程,所有Makefile中的变量命名应保持一种风格,否
      则会显得你是一个蹩脚的程序员(就像代码的变量命名风格一样)。
  \item 另外有一些变量名只包含了一个或者很少的几个特殊的字符(符号)。称它
      们为自动化变量。像“\$<”、“\$@”、“\$?”、“\$*”
      等。
\end{enumerate}

\section{变量的引用}
当我们定义了一个变量之后,就可以在Makefile的很多地方使用这个变量。变量的引用
方式是:“\$(VARIABLE\_NAME)"”或者“\$\{VARIABLE\_NAME\}”来引用一个
变量的定义。例如:“\$(foo)”或者“\$\{foo\}”就是取变量“foo”的值。美
元符号“\$”在Makefile中有特殊的含义,所有在命令或者文件名中使用
“\$”时需要用两个美元符号“\$\$”来表示。对一个变量的引用可以在
Makefile的任何上下文中,目标、依赖、命令、绝大多数指示符和新变量的赋值中。这
里有一个例子,其中变量保存了所有.o文件的列表:

\begin{Verbatim}[]
objects = program.o foo.o utils.o
program : $(objects) cc -o program
    $(objects)

$(objects) : defs.h
\end{Verbatim}

\noindent 变量引用的展开过程是严格的文本替换过程,就是说变量值的字符串被精确
的展开在变量被引用的地方。因此规则:

\begin{Verbatim}[]
foo = c
prog.o : prog.$(foo)
    $(foo) $(foo) -$(foo) prog.$(foo)
\end{Verbatim}

\noindent 被展开后就是:

\begin{Verbatim}[]
prog.c : prog.c
    cc -c prog.c
\end{Verbatim}

\noindent 通过这个例子会发现变量的展开过程和c语言中的宏展开的过程相同,是一个
严格的文本替换过程。上例中变量“foo”被展开的过程中,变量值中的前导空格会忽略。
举这个例子的目的是为了让我们更清楚地了解变量的展开过程,而不是建议大家按照这
样的方式来书写Makefile。在实际书写时,最好不要这么干。否则将会给你带来很多不
必要的麻烦。


\begin{quote}\kaishu
\textbf{注意}:Makefile中在对一些简单变量的引用,我们也可以不使用“()”和“\verb"{}"”
来标记变量名,而直接使用“\verb"$x"”的格式来实现,此种用法仅限于变量名为单字
符的情况。另外自动化变量也使用这种格式。对于一般多字符变量的引用必须使用括
号了标记,否则make将把变量名的首字母作为作为变量而不是整个字符串
(“\verb"$PATH"”在Makefile中实际上是“\verb"$(P)ATH"”)。这一点和shell中变量的引
用方式不同。shell中变量的引用可以是“\verb"${xx}"”或者“\verb"$xx"”格式。但在Makefile中多字
符变量名的引用只能是“\verb"$(xx)"”或者“\verb"${xx}"”格式。
\end{quote}

一般在我们书写Makefile时,各部分变量引用的格式我们建议如下:

\begin{enumerate}
\itemsep=4pt \parskip=0pt
  \item make变量(Makefile中定义的或者是make的环境变量)的引用使用
      “\$(VAR)” 格式,无论“VAR”是单字符变量名还是多字符变量
      名。
  \item  出现在规则命令行中shell变量(一般为执行命令过程中的临时变量,它不
      属于Makefile变量,而是一个shell变量)引用使用shell的“\$tmp”格
      式。
  \item 对出现在命令行中的make变量我们同样使用“\$(CMDVAR)” 格式来引
      用。

\end{enumerate}

例如:
\begin{Verbatim}[]
# sample Makefile
……
SUBDIRS := src foo

.PHONY : subdir
Subdir :
    @for dir in $(SUBDIRS); do  \
        $(MAKE) –C $$dir || exit 1; \
    done
……
\end{Verbatim}


\section{两种变量定义(赋值)}
在GNU make中,变量的定义有两种方式(或者称为风格)。我们把使用这两种方式定义
的变量可以看作变量的两种不同风格。变量的这两种不同的风格的区别在于:1. 定义方
式;2. 展开时机。下边我们分别对这两种不同的风格进行详细地讨论。

\subsection{递归展开式变量}
第一种风格的变量是递归方式扩展的变量。这一类型变量的定义是通过“=”或者使用指示
符“define”定义的。这种变量的引用,在引用的地方是严格的文本替换过程,此变量值
的字符串原模原样的出现在引用它的地方。如果此变量定义中存在对其他变量的引用,
这些被引用的变量会在它被展开的同时被展开。就是说在变量定义时,变量值中对其他
变量的引用不会被替换展开;而是变量在引用它的地方替换展开的同时,它所引用的其
它变量才会被一同替换展开。语言的描述可能比较晦涩,让我们来看一个例子:
\begin{Verbatim}[]
foo = $(bar)
bar = $(ugh)
ugh = Huh?
 	
all:;echo $(foo)
\end{Verbatim}

执行“make”将会打印出“Huh?”。整个变量的替换过程时这样的:首先
“\$(foo)”被替换为“\$(bar)”,接下来“\$(bar)”被替换为
“\$(ugh)”,最后“\$(ugh)”被替换为“Hug?”。整个替换的过程是在
执行“echo \$(foo)”时完成的。

这种类型的变量是其它版本的make所支持的类型。我们可以把这种类型的变量称为“递归
展开”式变量。此类型变量存有它的优点同时也存在其缺点。\textbf{其优点是:这种类
型变量在定义时,可以引用其它的之前没有定义的变量(可能在后续部分定义,或者是
通过make的命令行选项传递的变量)}。看一个这样的例子:

\begin{Verbatim}[]
CFLAGS = $(include_dirs) -O
include_dirs = -Ifoo -Ibar
\end{Verbatim}

“CFLAGS”会在命令中被展开为“-Ifoo -Ibar -O”。而在“CFLAGS”
的定义中使用了其后才定义的变量“include\_dirs”。

其缺点是:

   \textbf{缺点一}: 使用此风格的变量定义,可能会由于出现变量的递归定义而导致make陷入到
      无限的变量展开过程中,最终使make执行失败。例如,接上边的例子,我们给
      这个变量追加值:
\begin{Verbatim}[]
CFLAGS = $(CFLAGS) –O
\end{Verbatim}

      它将会导致make对变量“CFLAGS”的无限展过程中去(这种定义就是变量的递归
  定义)。因为一旦后续同样存在对“CLFAGS”定义的追加,展开过程将是套嵌的、不
  能终止的(在发生这种情况时,make会提示错误信息并结束)。一般书写Makefile
  时,这种追加变量值的方法很少使用(也不是我们推荐的方式)。看另外一个例
  子:
\begin{Verbatim}[]
x = $(y)
y = $(x) $(z)
\end{Verbatim}

这种情况下变量在进行展开时,同样会陷入死循环。所以对于此风格的变量,当在一
个变量的定义中需要引用其它的同类型风格的变量时需特别注意,防止变量展开过程
的死循环。

    \textbf{缺点二}:这种风格的变量定义中如果使用了函数,那么包含在变量值中
      的函数总会在变量被引用的地方执行(变量被展开时)。

      这是因为在这种风格变量的定义中,对函数引用的替换展开发生在变量展开的
      过程中,而不是在定义这个变量的时候。这样所带来的问题是:使make的执行
      效率降低(每一次在变量被展开时都要展开他所引用的函数);另外在某些时
      候会出现一些变量和函数的引用出现非预期的结果。特别是当变量定义中引用
      了“shell”和“wildcard”函数的情况,可能出现不可控制或者难以预料的错误,
      因为我们无法确定它在何时会被展开。
%\end{enumerate}

\subsection{直接展开式变量}
为了避免“递归展开式”变量存在的问题和不方便。GNU make支持另外一种风格的变量,
称为“直接展开”式。这种风格的变量使用“:=”定义。在使用“:=”定义变量时,变量值中
对其他量或者函数的引用在定义变量时被展开(对变量进行替换)。所以变量被定义后
就是一个实际需要的文本串,其中不再包含任何变量的引用。因此
\begin{Verbatim}[]
x := foo
y := $(x) bar
x := later
\end{Verbatim}
\noindent 就等价于:
\begin{Verbatim}[]
y := foo bar
x := later
\end{Verbatim}
\noindent 和递归展开式变量不同:此风格变量在定义时就完成了对所引用变量和函数
的展开,因此不能实现对其后定义变量的引用。如:
\begin{Verbatim}[]
CFLAGS := $(include_dirs) -O
include_dirs := -Ifoo -Ibar
\end{Verbatim}
\noindent 由于变量“include\_dirs”的定义出现在“CFLAGS”定义之后。
因此在“CFLAGS”的定义中,“include\_dirs”的值为空。“CFLAGS”
的值为“-O”而不是“-Ifoo -Ibar -O”。这一点也是直接展开式和递归展
开式变量的不同点。注意这里的两个变量都是“直接展开”式的。大家不妨试试将其中某
一个变量使用递归展开式定义后看一下又会出现什么样的结果。

下边我们来看一个复杂一点的例子。分析一下直接展开式变量定义(:=)的用法,这里
也用到了make的shell函数和变量“MAKELEVEL”(此变量在make的递归调用时代表make的
调用深度)。

其中包括了对函数、条件表达式和系统变量“MAKELEVEL”的使用:

\begin{Verbatim}[]
ifeq (0,${MAKELEVEL})
cur-dir   := $(shell pwd)
whoami  := $(shell whoami)
host-type := $(shell arch)
MAKE := ${MAKE} host-type=${host-type} whoami=${whoami}
endif
\end{Verbatim}

第一行是一个条件判断,如果是顶层Makefile,就定义下列变量。否则不定义任何变
量。第二、三、四、五行分别定义了一个变量,在进行变量定义时对引用到的其它变量
和函数展开。最后结束定义。利用直接展开式的特点我们可以书写这样一个规则:

\begin{Verbatim}[]
${subdirs}:
${MAKE} cur-dir=${cur-dir}/$@ -C $@ all
\end{Verbatim}
\noindent 它实现了在不同子目录下变量“cur\_dir”使用不同的值(为当前工作
目录)。

在复杂的Makefile中,推荐使用直接展开式变量。因为这种风格变量的使用方式和大多
数编程语言中的变量使用方式基本上相同。它可以使一个比较复杂的Makefile在一定程
度上具有可预测性。而且这种变量允许我们利用之前所定义的值来重新定义它(比如使
用某一个函数来对它以前的值进行处理并重新赋值),此方式在Makefile中经常用到。
尽量避免和减少递归式变量的使用。

\subsection{如何定义一个空格}
使用直接扩展式变量定义我们可以实现将一个前导空格定义在变量值中。一般变量值中
的前导空格字符在变量引用和函数调用时被丢弃。利用直接展开式变量在定义时对引用
的其它变量或函数进行展开的特点,我们可以实现在一个变量中包含前导空格并在引用
此变量时对空格加以保护。像这样:
\begin{Verbatim}[]
nullstring :=
space := $(nullstring) # end of the line
\end{Verbatim}

这里,变量“space”就表示一个空格。在“space”定义行中的注释使得我
们的目的更清晰(明确地描述一个空格字符比较困难),注释和变量引用
“\$(nullstring)”之间存在一个空格。通过这种方式我们就明确的指定了一个空
格。这是一个很好地实现方式。通过引用变量“nullstring”标明变量值的开始,
采用“\#”注释来结束,中间是一个空格字符。

make对变量进行处理时变量值中尾空格是不被忽略的,因此定义一个包含一个或者多个
空格的变量定义时,上边的实现就是一个简单并且非常直观的方式。但是需要注意:当
定义不包含尾空格的变量时,就不能使用这种方式,将变量定义和注释书写在同一行并
使用若干空格分开。否则,注释之前的空格会被作为变量值的一部分。例如下边的做法
就是不正确的:
\begin{Verbatim}[]
dir := /foo/bar    # directory to put the frobs in
\end{Verbatim}

\noindent 变量“dir”的值是“/foo/bar\qquad”(后面有4个空格),这可
能并不是想要实现的。如果一个文件以它作为路径来表示“\$(dir)/file”,那么
大错特错了。

在书写Makefile时。推荐将注释书写在独立的行或者多行,防止出现上边例子中的意外
情况,而且将注释书写在独立的行也使得Makefile清晰,便于阅读。对于特殊的定义,
比如定义包含一个或者多个空格空格的变量时进行详细地说明和注释。

\subsection{“?=”操作符}

GNU make中,还有一个被称为条件赋值的赋值操作符“?=”。被称为条件赋值是因为:只
有此变量在之前没有赋值的情况下才会对这个变量进行赋值。例如:
\begin{Verbatim}[]
FOO ?= bar
\end{Verbatim}
\noindent 其等价于:
\begin{Verbatim}[]
ifeq ($(origin FOO), undefined)
FOO = bar
endif
\end{Verbatim}
\noindent 含义是:如果变量“\verb"FOO"”在之前没有定义,就给它赋值
“\verb"bar"”。否则不改变它的值。


\section{变量的高级用法}
本节讨论关于变量的高级用法,这些高级的用法使我们可以更灵活的使用变量。

\subsection{变量的替换引用}
对于一个已经定义的变量,可以使用“替换引用”将其值中的后缀字符(串)使用指定的
字符(字符串)替换。格式为“\$(VAR:A=B)”(或者“\$\{VAR:A=B\}”),意
思是,替换变量“VAR”中所有“A”字符结尾的字为“B”结尾的字。
“结尾”的含义是空格之前(变量值多个字之间使用空格分开)。而对于变量其它部分的
“A”字符不进行替换。例如:
\begin{Verbatim}[]
foo := a.o b.o c.o
bar := $(foo:.o=.c)
\end{Verbatim}

在这个定义中,变量“bar”的值就为“a.c b.c c.c”。使用变量的替换引
用将变量“foo”以空格分开的值中的所有的字的尾字符“o”替换为
“c”,其他部分不变。如果在变量“foo”中如果存在“o.o”时,那
么变量“bar”的值为“a.c b.c c.c o.c”而不是
“a.c b.c c.c c.c”。

变量的替换引用其实是函数“patsubst”的一个简化实现。在GNU make中同时提供了这两
种方式来实现同样的目的,以兼容其它版本make。

另外一种引用替换的技术使用功能更强大的“patsubst”函数。%
它的格式和上面“\$(VAR:A=B)”的格式相类似,不过需要%
在“A”和“B”中需要包含模式字符“\%”。%
这时它和“\$(patsubst A,B \$(VAR))”所实现功能相同。例如:
\begin{Verbatim}[]
foo := a.o b.o c.o
bar := $(foo:%.o=%.c)
\end{Verbatim}

这个例子同样使变量“bar”的值为“a.c b.c c.c”。这种格式的替换引用
方式比第一种方式更通用。

\subsection{变量的套嵌引用}
计算的变量名是一个比较复杂的概念,仅用在那些复杂的Makefile中。通常我们不需要
对它的计算过程进行深入地了解,只要知道当一个被引用的变量名之中含有“\$”
时,可得到另外一个值。如果您是一个比较喜欢追根问底的人,或者想弄清楚make计算
变量的过程。那么就可以参考本节的内容。

一个变量名(文本串)之中可以包含对其它变量的引用。这种情况我们称之为“变量的套
嵌引用”或者“计算的变量名”。先看一个例子:
\begin{Verbatim}[]
x = y
y = z
a := $($(x))
\end{Verbatim}

这个例子中,最终定义了“a”的值为“z”。来看一下变量的引用过程:首%
先最里边的变量引用“\$(x)”被替换为变量名“y”(就是“\$(\$(x))”
被替换为了“\$(y)”),之后“\$(y)”被替换为“z”%
(就是a := z)。这个例子中(a:=\$(\$(x)))所引用的变量名%
不是明确声明的,而是由\$(x)扩展得%
到。这里“\$(x)”相对于外层的引用就是套嵌的变量引用。

上个例子我们看到是一个两层的套嵌引用的例子,具有多层的套嵌引用在Makefile中也
是允许的。下边我们在来看一个三层套嵌引用的例子:
\begin{Verbatim}[]
x = y
y = z
z = u
a := $($($(x)))
\end{Verbatim}

这个例子最终是定义了“a”的值为“u”。它的扩展过程和上边第一个例子
的过程相同。首先“\$(x)”被替换为“y”,则“\$(\$(x))”就是
“\$(y)”,“\$(y)”再被替换为“z”,所以就有“a:=\$(z)”;
而“\$(z)”最后被替换为“u”。

以上两个套嵌引用的例子中没有用到递归展开式变量的特点。递归展开式变量的变量名
的计算过程,也是按照相同的方式被扩展的。例如:
\begin{Verbatim}[]
x = $(y)
y = z
z = Hello
a := $($(x))
\end{Verbatim}

此例最终实现了“a:=Hello”这么一个定义。这里\$(\$(x))被替换成了
\$(\$(y)),因为\$(y)值是“z”,所以,最终结果是:%
a:=\$(z),也就是“Hello”。

递归变量的套嵌引用过程,也可以包含变量的修改引用和函数调用。看下边的例子,其中使用了make的文本处理函数:
\begin{Verbatim}[]
x = variable1
variable2 := Hello
y = $(subst 1,2,$(x))
z = y
a := $($($(z)))
\end{Verbatim}

此例同样的实现“a:=Hello”。“\$(\$(\$(z)))”首先被替换为
“\$(\$(y))”,之后再次被替换为“\$(\$(su\-bst 1,2,\$(x)))”(“\$(x)”的值是
“variable1”,所以有“\$(\$(subst 1,2,\$(variable1)))”)。函数处理之后为
“\$(variable2)”。之后对它在进行替换展开。最终,变量“a”的值就是“Hello”。从上边
的例子中我们看到,计算的变量名的引用过程存在多层套嵌,也使用了文本处理函数。
这个复杂的计算变量的过程,会使很多人感到混乱甚至迷惑。上例中所要实现的目的就
没有直接使用“a:=Hello”来的直观。在书写Makefile时,应尽量避免使用套嵌的变量引
用。在一些必需的地方,也最好不要使用高于两级的套嵌引用。使用套嵌的变量引用
时,如果涉及到递归展开式变量的引用时需要特别注意。一旦处理不当就可能导致递归
展开错误,从而导致难以预料的结果。

一个计算的变量名可以不是对一个完整、单一的其他变量的引用。其中可以包含多个变
量的引用,也可以包含一些文本字符串。就是说,计算变量的名字可以由一个或者多个
变量引用同时加上字符串混合组成。例如:

\begin{Verbatim}[]
a_dirs := dira dirb
1_dirs := dir1 dir2

a_files := filea fileb
1_files := file1 file2

ifeq "$(use_a)" "yes"
a1 := a
else
a1 := 1
endif

ifeq "$(use_dirs)" "yes"
df := dirs
else
df := files
endif

dirs := $($(a1)_$(df))
\end{Verbatim}

这个例子对变量“dirs”进行定义,变量的可能取值为
“a\_dirs”、“1\_dirs”、“a\_files”和“a\_files”四个之
一,具体依赖于“use\_a”和“use\_dirs”的定义。

计算的变量名也可以使用上一小节我们讨论过的“变量的替换引用”。例如:
\begin{Verbatim}[]
a_objects := a.o b.o c.o
1_objects := 1.o 2.o 3.o

sources := $($(a1)_objects:.o=.c)
\end{Verbatim}

这个例子实现了变量“sources”的定义,它的可能取值为“a.c b.c c.c”和%
“1.c 2.c 3.c”,具体依赖于“a1”的定义。大家自己分析一下计算变量名的过程。

使用嵌套的变量引用的唯一限制是,不能通过指定部分需要调用的函数名称(调用的函
数包括了函数名本身和执行的参数)来实现对这个函数的调用。这是因为套嵌引用在展
开之前已经完成了对函数名的识别测试。我们来看一个例子,此例子试图将函数执行的
结果赋值给一个变量:
\begin{Verbatim}[]
ifdef do_sort
func := sort
else
func := strip
endif

bar := a d b g q c

foo := $($(func) $(bar))
\end{Verbatim}

此例的本意是将“sort”或者“strip”(依赖于是否定义了变量
“do\_sort”)以“a d b g q c”的执行结果赋值变量“foo”。在这
里使用了套嵌引用方式来实现,但是本例的结果是:变量“foo”的值为字符串
“sort a d b g q c”或者“strip a d g q c”。这是目前版本的make在处
理套嵌变量引用时的限制。

计算的变量名可以用在:1. 一个使用赋值操作符定义变量的左值部分;2. 使用
“define”定义的变量名中。例如:

\begin{Verbatim}[]
dir = foo
$(dir)_sources := $(wildcard $(dir)/*.c)
define $(dir)_print
lpr $($(dir)_sources)
endef
\end{Verbatim}
在这个例子中我们定义了三个变量:“dir”,“foo\_sources”和“foo\_print”。

计算的变量名在进行替换时的顺序是:从最里层的变量引用开始,逐步向外进行替换。
一层层展开直到最后计算出需要应用的具体的变量,之后进行替换展开得到实际的引用
值。

变量的套嵌引用(计算的变量名)在我们的Makefile中应该尽量避免使用。在必需的场
合使用时掌握的原则是:套嵌使用的层数越少越好,使用多个两层套嵌引用代替一个多
层的套嵌引用。如果在你的Makefile中存在一个层次很深的套嵌引用。会给其他人阅读
造成很大的困难。而且变量的多级套嵌引用在某些时候会使简单问题复杂化。

作为一个优秀的程序员,在面对一个复杂问题时,应该是寻求一种尽可能简单、直接并
且高效的处理方式来解决,而不是将一个简单问题在实现上复杂化。如果想在简单问题
上突出自己使用某种语言的熟练程度,是一种非常愚蠢、且不成熟的行为。

\begin{quote}\kaishu
\textbf{注意}:套嵌引用的变量和递归展开的变量在本质上存在区别。套嵌的引用就
是使用一个变量表示另外一个变量,或者更多的层次;而递归展开的变量表示当一个
变量存在对其它变量的引用时,对这变量替换的方式。递归展开在另外一个角度描述
了这个变量在定义是赋予它的一个属性或者风格。并且我们可以在定义个一个递归展
开式的变量时使用套嵌引用的方式,但是建议你的实际编写Makefile时要尽量避免这
种复杂的用法。
\end{quote}

\section{变量取值}
一个变量可以通过以下几种方式来获得值:

\begin{dinglist}{226}
\itemsep=4pt \parskip=0pt

\item 在运行make时通过命令行选项来取代一个已定义的变量值。

\item 在makefile文件中通过赋值的方式或者使用“define”来为一个变量赋值。

\item 将变量设置为系统环境变量。所有系统环境变量都可以被make使用。

\item 自动化变量,在不同的规则中自动化变量会被赋予不同的值。它们每一个都有单
    一的习惯性用法。

\item 一些变量具有固定的值。

\end{dinglist}


\section{如何设置变量}
Makefile中变量的设置(也可以称之为定义)是通过“=”(递归方式)或者“:=”
(静态方式)来实现的。“=”和“:=”左边的是变量名,右边是变量的值。下边就是一个变
量的定义语句:
\begin{Verbatim}[]
objects = main.o foo.o bar.o utils.o
\end{Verbatim}

这个语句定义了一个变量“objects”,其值为一个.o文件的列表。变量名
两边的空格和“=”之后的空格在make处理时被忽略。

使用“=”定义的变量称之为“递归展开”式变量;使用“:=”定义的变量称为
“直接展开”式变量,“直接展开”式的变量如果其值中存其他变量或者函数的引用,在定
义时这些引用将会被替换展开。

定义一个变量时需要明确以下几点:

\begin{enumerate}
\itemsep=4pt \parskip=0pt

\item 变量名之中可以包含函数或者其它变量的引用,make在读入此行时根据已定义
    情况进行替换展开而产生实际的变量名。

\item 变量的定义值在长度上没有限制。不过在使用时还是需要根据实际情况考虑,
    保证你的机器上有足够的可用的交换空间来处理一个超常的变量值。变量定义较
    长时,一个好的做法就是将比较长的行分多个行来书写,除最后一行外行与行之
    间使用反斜杠(\verb"\")连接,表示一个完整的行。这样的书写方式对make的
    处理不会造成任何影响,便于后期修改维护而且使得你的Makefile更清晰。例如
    上边的例子就可以这样写:

\item 当引用一个没有定义的变量时,make默认它的值为空。

\item  一些特殊的变量在make中有内嵌固定的值,不过这些变量允许我们在Makefile
    中显式得重新给它赋值。

\item 还存在一些由两个符号组成的特殊变量,称之为自动环变量。它们的值不能在
    Makefile中进行显式的修改。这些变量使用在规则中时,不同的规则中它们会被
    赋予不同的值。

\item 如果你希望实现这样一个操作,仅对一个之前没有定义过的变量进行赋值。那
    么可以使用速记符“?=”(条件方式)来代替“=”或者“:=”
    来实现。
\end{enumerate}


\section{追加变量值}
通常,一个通用变量在定义之后的其他一个地方,可以对其值进行追加。这是非常有用
的。我们可以在定义时(也可以不定义而直接追加)给它赋一个基本值,后续根据需要
可随时对它的值进行追加(增加它的值)。在Makefile中使用“+=”(追加方式)
来实现对一个变量值的追加操作。像下边那样:
\begin{Verbatim}[]
objects += another.o
\end{Verbatim}

这个操作把字符串“another.o”添加到变量“objects”原有值的末尾,使
用空格和原有值分开。因此我们可以看到:
\begin{Verbatim}[]
objects = main.o foo.o bar.o utils.o
objects += another.o
\end{Verbatim}

上边的两个操作之后变量“objects”的值就为:%
“main.o foo.o bar.o utils.o another.o”。使用“+=”操作符,相当于:
\begin{Verbatim}[]
objects = main.o foo.o bar.o utils.o
objects := $(objects) another.o
\end{Verbatim}

但是,这两种方式可能在简单一些的Makefile有相同的效果,复杂的Makefile中它们之
间的差异就会导致一些问题。为了方便我们调试,了解这两种实现的差异还是很有必要
的。

\begin{enumerate}
\itemsep=4pt \parskip=0pt

\item 如果被追加值的变量之前没有定义,那么,“+=”会自动变成
    “=”,此变量就被定义为一个递归展开式的变量。如果之前存在这个变量
    定义,那么“+=”就继承之前定义时的变量风格。

\item 直接展开式变量的追加过程:变量使用“:=”定义,之后“+=”操
    作将会首先替换展开之前此变量的值,尔后在末尾添加需要追加的值,并使用
    “:=”重新给此变量赋值。实际的过程像下边那样:

\begin{Verbatim}[]
variable := value
variable += more
\end{Verbatim}

就是:

\begin{Verbatim}[]
variable := value
variable := $(variable) more
\end{Verbatim}

\item 递归展开式变量的追加过程:一个变量使用“=”定义,之后“+=”
    操作时不对之前此变量值中的任何引用进行替换展开,而是按照文本的扩展方式
    (之前等号右边的文本未发生变化)替换,尔后在末尾添加需要追加的值,并使
    用“=”给此变量重新赋值。实际的过程和上边的相类似:

\begin{Verbatim}[]
variable = value
variable += more
\end{Verbatim}

就是:

\begin{Verbatim}[]
temp = value
variable = $(temp) more
\end{Verbatim}
\end{enumerate}

当然了,上边的过程并不会存在中间变量:“temp”,使用它的目的时方便描述。
这种情况时如果“value”中存在某种引用,情况就有些不同了。看我们通常一个
会用到的例子:
\begin{Verbatim}[]
CFLAGS = $(includes) -O
...
CFLAGS += -pg # enable profiling
\end{Verbatim}

第一行定义了变量“CFLAGS”,它是一个递归展开式的变量。因此make在处理它的
定义时不会对其值中的引用“\$(includes)”进行展开,它的替换展开是在变量
“CFLAGS”被引用的规则中。因此,变量“include”可以在
“CFLAGS”之前不进行定义,只要它在实际引用“CFLAGS”之前定义就可以
了。但是如果给“CFLAGS”追加值使用“:=”操作符,我们按照下边那样实
现:
\begin{Verbatim}[]
CFLAGS := $(CFLAGS) -pg # enable profiling
\end{Verbatim}

这样似乎好像很正确,但是实际上它在有些情况时却不是你所要实现的。来看看,因为
“:=”操作符定义的是直接展开式变量,因此变量值中对其它变量或者函数的引用
会在定义时进行展开。在这种情况下,如果变量“includes”在之前没有进行定义
的话,变量“CFLAGS”的值为“-O -pg”(\$(includes)被替换展开
为空字符)。而其后出现的“includes”的定义对“CFLAGS”将不产生影
响。相反的情况,如果在这里使用“+=”实现:
\begin{Verbatim}[]
CFLAGS += -pg # enable profiling
\end{Verbatim}
那么变量“CFLAGS”的值就是文本串“\$(includes) –O -pg”,因为之前
“CFLAGS”定义为递归展开式,所以追加值时不会对其值的引用进行替换展开。因
此变量“includes”只要出现在规则对“CFLAGS”的引用之前定义,它都可
以对“CFLAGS”的值起作用。对于递归展开式变量的追加,make程序会同样会按照
递归展开式的定义来实现对变量的重新赋值,不会发生递归展开式变量展开过程的无限
循环。

\section{override 指示符}
通常在执行make时,如果通过命令行定义了一个变量,那么它将替代在Makefile中出现
的同名变量的定义。就是说,对于一个在Makefile中使用常规方式(使用
“=”、“:=”或者“define”)定义的变量,我们可以在执行make时
通过命令行方式重新指定这个变量的值,命令行指定的值将替代出现在Makefile中此变
量的值。如果不希望命令行指定的变量值替代在Makefile中的变量定义,那么我们需要
在Makefile中使用指示符“override”来对这个变量进行声明,像下边那样:
\begin{Verbatim}[]
override VARIABLE = VALUE
\end{Verbatim}
\noindent 或者:
\begin{Verbatim}[]
override VARIABLE := VALUE
\end{Verbatim}
\noindent 也可以对变量使用追加方式:
\begin{Verbatim}[]
override VARIABLE += MORE TEXT
\end{Verbatim}
对于追加方式需要说明的是:变量在定义时使用了“override”,则后续对它值进
行追加时,也需要使用带有“override”指示符的追加方式。否则对此变量值的追
加不会生效。

指示符“override”并不是用来调整Makefile和执行时命令参数的冲突,其存在的
目的是为了使用户可以改变或者追加那些使用make的命令行指定的变量的定义。从另外
一个角度来说,就是实现了在Makefile中增加或者修改命令行参数的一种机制。我们可
能会有这样的需求;可以通过命令行来指定一些附加的编译参数,对一些通用的参数或
者必需的编译参数在Makefile中指定,而在命令行中指定一些特殊的参数。对于这种需
求,我们就需要使用指示符“override"”来实现。

例如:无论命令行指定那些编译参数,编译时必须打开“-g”选项,那么在
Makefile中编译选项“CFLAGS”应该这样定义:
\begin{Verbatim}[]
override CFLAGS += -g
\end{Verbatim}
这样,在执行make时无论在命令行中指定了那些编译选项(“指定CFLAGS”的值),编译
时“-g”参数始终存在。


同样,使用“define”定义变量时同样也可以使用“override”进行声明。
例如:
\begin{Verbatim}[]
override define foo
bar
endef
\end{Verbatim}
最后我们来看一个例子:
\begin{Verbatim}[]
# sample Makefile

EXEF = foo

override CFLAGS += -Wall –g

.PHONY : all debug test
all : $(EXEF)

foo : foo.c
………..
………..

$(EXEF) : debug.h
$(CC) $(CFLAGS) $(addsuffix .c,$@) –o $@

debug :
    @echo ”CFLAGS = $(CFLAGS)”
\end{Verbatim}

执行:make CFLAGS=-O2 将显式编译“foo”的过程是“cc –O2 –Wall –g foo.c –o foo”。
执行“make CFLAGS=-O2 debug”可以查看到变量“CFLAGS”的值为“–O2 –Wall –g”。另外,
这个例子中,如果把变量“CFLAGS”之前的指示符“override”去掉,使用相同的命令将得
到不同的结果。大家试试看!


\section{多行定义}
定义变量的另外一种方式是使用“define”指示符。它定义一个包含多行字符串的变量,
我们就是利用它的这个特点实现了一个完整命令包的定义。使用“define”定义的命令包
可以作为“eval”函数的参数来使用。

本文的前些章节已经不止一次的提到并使用了“define”。相信大家已经有所了解。本节
就“define”定义变量从以下几个方面来讨论:

\textbf{1.} “define”定义变量的语法格式:以指示符“define”开始,“endif”结束,之
间的所有内容就是所定义变量的值。所要定义的变量名字和指示符“define”在同一行,
使用空格分开;指示符所在行的下一行开始一直到“endif”所在行的上一行之间的若干
行,是变量值。

\begin{Verbatim}[]
define two-lines
echo foo
echo $(bar)
endef
\end{Verbatim}

如果将变量“two-lines”作为命令包执行时,其相当于:
\begin{Verbatim}[]
two-lines = echo foo; echo $(bar)
\end{Verbatim}

大家应该对这个命令的执行比较熟悉。它把变量“two-lines”的值作为一个完整的shell
命令行来处理(是使用分号“;”分开的在同一行中的两个命令而不是作为两个命令行来处
理),保证了变量完整。

\textbf{2.} 变量的风格:使用“define”定义的变量和使用“=”定义的变量一样,属于
“递归展开”式的变量,两者只是在语法上不同。因此“define”所定义的变量值中,对其
它变量或者函数引用不会在定义变量时进行替换展开,其展开是在“define”定义的变量
被展开的同时完成的。

\textbf{3.} 可以套嵌引用。因为是递归展开式变量,所以在嵌套引用时“\verb"$(x)"”
将是变量的值的一部分。

\textbf{4.} 变量值中可以包含:换行符、空格等特殊符号(注意如果定义中某一行是
以[Tab]字符开始时,当引用此变量时这一行会被作为命令行来处理)。

\textbf{5.} 可以使用“override”在定义时声明变量:这样可以防止变量的值被命令行
指定的值替代。例如:
\begin{Verbatim}[]
override define two-lines
foo
$(bar)
endef
\end{Verbatim}

\section{系统环境变量}
make在运行时,系统中的所有环境变量对它都是可见的。在Makefile中,可以引用任何
已定义的系统环境变量。(这里我们区分系统环境变量和make的环境变量,系统环境变
量是这个系统所有用户所拥有的,而make的环境变量只是对于make的一次执行过程有
效,以下正文中出现没有限制的“环境变量”时默认指的是“系统环境变量”,在特殊的场
合我们会区分两者)正因为如此,我们就可以设置一个命名为“CFLAGS”的环境变量,用
它来指定一个默认的编译选项。就可以在所有的Makefile中直接使用这个变量来对c源代
码就行编译。通常这种方式是比较安全的,但是它的前提是大家都明白这个变量所代表
的含义,没有人在Makefile中把它作其他的用途。当然了,你也可以在你的Makefile中
根据你的需要对它进行重新定义。

使用环境变量需要注意以下几点:

\textbf{1.} 在Makefile中对一个变量的定义或者以make命令行形式对一个变量的定
义,都将覆盖同名的环境变量(注意:它并不改变系统环境变量定义,被修改的环境变
量只在make执行过程有效)。而make使用“-e”参数时,Makefile和命令行定义的变量不
会覆盖同名的环境变量,make将使用系统环境变量中这些变量的定义值。


\textbf{2.} make的递归调用中,所有的系统环境变量会被传递给下一级
make。\textbf{默认情况下,只有环境变量和通过命令行方式定义的变量才会被传递给
子make进程}。在Makefile中定义的普通变量需要传递给子make时需要使用“export”指示
符来对它声明。


\textbf{3.} 一个比较特殊的是环将变量“SHELL”。在系统中这个环境变量的用途是用来
指定用户和系统的交互接口,显然对于make是不合适的。因此make的执行环境变量
“SHELL”没有使用同名的环境变量定义,而是“/bin/sh”。make默认“/bin/sh”作为它的命
令行解释程序(make在执行之前将变量“SHELL”设置为“/bin/sh”)。

我们不推荐使用环境变量的方式来完成普通变量的工作,特别是在make的递归调用中。
任何一个环境变量的错误定义都对系统上的所有make产生影响,甚至是毁坏性的。因为
环境变量具有全局的特征。所以尽量不要污染环境变量,造成环境变量名字污染。我想
大多数系统管理员都明白环境变量对系统是多么的重要。

我们来看一个例子,结束本节。假如我们的机器名为“server-cc”;我们的Makefile内容
如下:
\begin{Verbatim}[]
# test makefile
HOSTNAME = server-http
…………
…………
.PHONY : debug
debug :
    @echo “hostname is : $( HOSTNAME)”
    @echo “shell is $(SHELL)”
\end{Verbatim}

1. 执行“make debug”将显示
\begin{Verbatim}[]
hostname is : server-http
shell is /bin/sh
\end{Verbatim}

2. 执行“make –e debug”;将显示:
\begin{Verbatim}[]
hostname is : server-cc
shell is /bin/sh
\end{Verbatim}

3. 执行“make –e HOSTNAEM=server-ftp”;将显示:
\begin{Verbatim}[]
hostname is : server-ftp
shell is /bin/sh
\end{Verbatim}

\begin{quote}\kaishu
\textbf{记住:}除非必须,否则在你的Makefile中不要重置环境变量“SHELL”的值。
因为一个不正确的命令行解释程序可能会导致规则定义的命令执行失败,甚至是无法
执行!当需要重置它时,必须有充分的理由和配套的规则命令来适应这个新指定的命
令行解释程序。
\end{quote}

\section{目标指定变量}
在Makefile中定义一个变量,那么这个变量对此Makefile的所有规则都是有效的。它就
像是一个“全局的”变量(仅限于定义它的那个Makefile中的所有规则,如果需要对其它
的Makefile中的规则有效,就需要使用“export”对它进行声明。类似于c语言中的全局静
态变量,使用static声明的全局变量)。当然“自动化变量”除外。

另外一个特殊的变量定义就是所谓的“目标指定变量(Target-specific Variable)”。
此特性允许对于相同变量根据目标指定不同的值,有点类似于自动化变量。目标指定的
变量值只在指定它的目标的上下文中有效,对于其他的目标没有影响。就是说目标指定
的变量具有只对此目标上下文有效的“局部性”。

设置一个目标指定变量的语法为:
\begin{Verbatim}[]
TARGET ... : VARIABLE-ASSIGNMENT
\end{Verbatim}
\noindent 或者:
\begin{Verbatim}[]
TARGET ... : override VARIABLE-ASSIGNMENT
\end{Verbatim}

一个多目标指定的变量的作用域是所有这些目标的上下文,它包括了和这个目标相关的
所有执行过程。

目标指定变量的一些特点:

\textbf{1.} “VARIABLE-ASSIGNMENT”可以使用任何一个有效的赋值方式,“=”(递
归)、“:=”(静态)、“+=”(追加)或者“?=”(条件)。

\textbf{2.}   使用目标指定变量值时,目标指定的变量值不会影响同名的那个全局变
量的值。就是说目标指定一个变量值时,如果在Makefile中之前已经存在此变量的定义
(非目标指定的),那么对于其它目标全局变量的值没有变化。变量值的改变只对指定
的这些目标可见。

\textbf{3.} 标指定变量和普通变量具有相同的优先级。就是说,当我们使用make命令
行的方式定义变量时,命令行中的定义将替代目标指定的同名变量定义(和普通的变量
一样会被覆盖)。另外当使用make的“-e”选项时,同名的环境变量也将覆盖目标指定的
变量定义。因此为了防止目标指定的变量定义被覆盖,可以使用第二种格式,使用指示
符“override”对目标指定的变量进行声明。

\textbf{4.} 目标指定的变量和同名的全局变量属于两个不同的变量,它们在定义的风
格(递归展开式和直接展开式)上可以不同。

\textbf{5.} 目标指定的变量变量会作用到由这个目标所引发的所有的规则中去。例
如:

\begin{Verbatim}[]
prog : CFLAGS = -g
prog : prog.o foo.o bar.o
\end{Verbatim}

这个例子中,无论Makefile中的全局变量“CFLAGS”的定义是什么。对于目标“prog”以及
其所引发的所有(包含目标为“prog.o”、“foo.o”和“bar.o”的所有规则)规则,变量
“CFLAGS”值都是“-g”。

使用目标指定变量可以实现对于不同的目标文件使用不同的编译参数。看一个例子:

\begin{Verbatim}[]
# sample Makefile

CUR_DIR = $(shell pwd)
INCS := $(CUR_DIR)/include
CFLAGS := -Wall –I$(INCS)

EXEF := foo bar

.PHONY : all clean
all : $(EXEF)

foo : foo.c
foo : CFLAGS+=-O2
bar : bar.c
bar : CFLAGS+=-g
………
………

$(EXEF) : debug.h
    $(CC) $(CFLAGS) $(addsuffix .c,$@) –o $@

clean :
    $(RM) *.o *.d $(EXES
\end{Verbatim}

这个Makefile文件实现了在编译程序“foo”使用优化选项“-O2”但不使用调试选项“-g”,
而在编译“bar”时采用了“-g”但没有“-O2”。这就是目标指定变量的灵活之处。目标指定
变量的其它特性大家可以修改这个简单的Makefile来进行验证!

\section{模式指定变量}
GNU make除了支持上一节所讨论的模式指定变量之外,还支持另外一种方式:模式指定
变量(Pattern-specific Variable)。使用目标定变量定义时,此变量被定义在某个具
体目标和由它所引发的规则的目标上。而模式指定变量定义是将一个变量值指定到所有
符合此模式的目标上。对于同一个变量如果使用追加方式,通常对于一个目标,它的局
部变量值是:(为所有规则定义的全局值)+(引发它所在规则被执行的目标所指定的
值)+(它所符合的模式指定值)+(此目标所指定的值)。这个大家也不需要深入了
解。

设置一个模式指定变量的语法和设置目标变量的语法相似:
\begin{Verbatim}[]
PATTERN ... : VARIABLE-ASSIGNMENT
\end{Verbatim}
\noindent 或者
\begin{Verbatim}[]
PATTERN ... : override VARIABLE-ASSIGNMENT
\end{Verbatim}

和目标指定变量语法的唯一区别就是:这里的目标是一个或者多个“模式”目标(包含模
式字符“\%”)。例如我们可以为所有的.o文件指定变量“CFLAGS”的值:
\begin{Verbatim}[]
%.o : CFLAGS += -O
\end{Verbatim}

它指定了所有.o文件的编译选项包含“-O”选项,不改变对其它类型文件的编译选项。

需要说明的是:在使用模式指定的变量定义时。目标文件一般除了模式字符(\%)以外
需要包含某种文件名的特征字符(例如:“a\%”、“\%.o”、“\%.a”等)。当单独使用“\%”
作为目标时,指定的变量会对所有类型的目标文件有效。
